\section{Objectifs}\label{sec:intro:objectives}

Parmi tous les sujets de recherche mentionnés ci dessus, le contexte de cette thèse a été restreint des manières suivantes :

\paragraph{Architecture cible}
La thèse se concentre sur l'exploitation des machines à mémoire partagée, de type NUMA.

\paragraph{Modèle de programmation}
La programmation par flot de données est un des meilleurs moyens pour exploiter pleinement le potentiel de ces architectures.
Dans ce type de modèle de programmation, l'utilisateur décrit son programme à l'aide de \emph{tâche} - la plus petite unité d'exécution possible - qui consomment et produise des données.
Si une tâche consomme une donnée produite par une tâche précédente, cela créer un lien entre elles, que l'ont appelle une \emph{dépendance}.
L'ensemble du programme peut être représenté par un graphe de tâches liées par des dépendances, et l'exécution de ce graphe est confié à un programme spécifique que l'on appelle \emph{support exécutif}.

Un certain nombre de langage de programmation utilise ce modèle de programmation, cette thèse se concentre sur le standard de-facto pour cibler les architectures NUMA : OpenMP.

Avec ces restrictions, les objectifs de la thèse étaient double :
\begin{itemize}
 \item Évaluer le langage OpenMP vis-à-vis des architectures actuelles.
 \item Proposer et évaluer des évolutions au langage et aux différents logiciel, en particulier le support exécutif, afin d'améliorer les performances.
\end{itemize}

La plupart des travaux ont été réalisés sur des programmes par flot de données, même si certaines applications à base de boucles ont été étudiées.

L'évaluation d'OpenMP a été réalisée avec plusieurs support exécutif et à donné lieu à la création d'une suite de benchmark, ainsi qu'un outil, NOMDEL4OUTIL, afin de faciliter l'exécution contrôlée de parties d'applications.

Nous avons montré le manque de support au sein d'OpenMP pour cibler les architectures NUMA.
Pour palier à ce problème, nous avons porté notre attention sur l'amélioration de l'usage des informations disponible au sein du support exécutif. Nous avons également proposé des moyens additionnels pour que le programmeur puisse donner plus d'information au support exécutif.

Cela a donné lieu à deux autres contributions :
\begin{itemize}
 \item Une proposition de clause d'affinité entre tâche et ressource dans OpenMP.
 \item Des extensions du support exécutif pour améliorer la distributions des données, et tirer parti de la localité des données.
\end{itemize}



