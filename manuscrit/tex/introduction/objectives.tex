\section{Objectifs}\label{sec:intro:objectives}

L'objectif principal de cette thèse était d'étudier les améliorations possibles de l'exploitation des architectures NUMA, à l'aide d'un modèle de programmation à base de tâches.
Cela a été découpé en trois axes de travail.


\subsection*{Analyse du comportement d'applications sur machine NUMA}

Avant de pouvoir penser aux améliorations, il faut commencer par analyser les différents points améliorables, tant du côté logiciel que matériel.
Si l'on souhaitait cibler les modèles de programmation à base de tâches, il fallait néanmoins choisir l'un des modèles existants pour l'étude concrète, et ce choix s'est porté sur OpenMP.

Les tâches avec dépendances ont été ajoutées peu de temps avant le début de cette thèse dans OpenMP, avec la version~4.0. Face à l'absence de suite de benchmarks ciblant spécifiquement cette construction, nous avons commencé par développer une suite de benchmarks, les KASTORS~\cite{Virouleau2014}.
Les applications présentes dans cette suite ont été adaptées depuis des applications existantes, afin d'utiliser les constructions dont nous avions besoin, et qui sont aujourd'hui utilisées par la communauté.


Dans le but de pouvoir étudier plus précisément le comportement de ces applications sur les architectures NUMA, nous avons écrit un outil, \outil.
Les applications reposent sur des tâches de calculs, dont les bonnes performances sont nécessaires pour la performance globale de l'application.
Étudier ces tâches au cours de l'exécution de l'application peut être contraignant~: instrumenter le support exécutif, regarder des traces d'exécution~; ce sont des options qui permettent de mettre en évidence un problème.
Mais les raisons derrière une différence de performance entre deux tâches similaires ne sont pas forcément évidentes, et rejouer une tâche pour tenter d'isoler le ou les paramètres à la source de cette variation est encore plus dur.

L'objectif derrière cet outil est le suivant~: à partir d'un code d'une tâche de calcul à étudier fournit par l'utilisateur et de ses paramètres en entrée, \outil permet de rejouer ce code en changeant certains paramètres.
Concrètement cela peut être par exemple le placement des données de la tâche, son jeu de données en entrée, ou encore le cœur sur lequel elle s'exécute.
L'utilisateur peut alors analyser finement quels sont les impacts de chaque paramètre et comment l'architecture sous-jacente réagit.



\subsection*{Quelles améliorations pour l'utilisateur et le support exécutif ?}

À partir des conclusions tirées du point précédent, le second objectif était de trouver, proposer, et évaluer une amélioration possible, tant pour l'utilisateur que pour le support exécutif.

Ces réflexions ont donné lieu à une contributions majeures, séparée en deux volets : le premier est axé sur la réponse au besoin de l'utilisateur, en proposant une clause |affinity| pour les tâches OpenMP~\cite{Virouleau2016b}.
Cette clause a pour but de permettre à l'utilisateur d'indiquer explicitement un lien fort entre une tâche et une ressource de la machine, que ce soit un cœur, un nœud, ou une donnée.
Le second est axé sur l'extension du support exécutif~\cite{Virouleau2016a}, d'une part pour faciliter la distribution des données sur la machine, et d'autre part pour exploiter les informations disponibles sur les données manipulées par les tâches, dans le but d'améliorer la localité des données au cours de l'exécution.


\subsection*{Place des travaux dans l'évolution du matériel et du logiciel}

Au cours de cette thèse les architectures NUMA ont évolué, on peut alors se demander dans quelle mesure l'évolution du matériel impacte les travaux de cette thèse.
Le dernier objectif est donc d'analyser les travaux effectués - voir les compléter - afin de proposer des approches indépendantes du matériel, dans le but de faciliter le travail du programmeur et du développeur de support exécutif à l'avenir.

Parmi nos travaux, \outil est indépendant de l'architecture. L'approche générale consiste à caractériser les sections de calculs vitales, à l'aide de scénarios d'expérience qui eux sont spécifiques à l'architecture, pour au final obtenir des données qui aideront à déterminer le comportement global de l'application.
Pour estimer ce comportement global à partir des données fournies par \outil, nous avons réalisé des travaux préliminaires sur un simulateur de support exécutif fonctionnant par vol de travail.
Le coût - en temps - de changer l'implémentation utilisée par un support exécutif réel est assez important, ces travaux préliminaires ont pour objectif de donner un premier aperçu de l'impact que pourrait avoir une modification de l'ordonnancement par le support exécutif.
Cela permet ainsi, lors du "portage" d'une application ou d'un support exécutif sur une nouvelle architecture, d'estimer son comportement, et évaluer si des changements dans l'un des deux sont nécessaires.
