\section{Objectifs}\label{sec:intro:objectives}

L'objectif principal de cette thèse était d'étudier les améliorations possibles de l'exploitation des architectures NUMA, à l'aide d'un modèle de programmation à base de tâches.
Cela a été découpé en trois axes de travail.


\subsection{Analyse du comportement d'applications sur machine NUMA}

TODO : parler du fait qu'on prend une application par partie, qu'on regarde les variables côté applications qui font changer les perf, et qu'est ce qui est limitant côté machine.

\subsection{Quelles améliorations pour l'utilisateur et le support exécutif ?}

TODO : à partir du constat précédent, comment on améliore l'existant ?

\subsection{Place des travaux dans l'évolution du matériel et du logiciel}

TODO : sur 3 ans on a vu les archis et les modèle de programmation évoluer, comment notre travail s'inscrit dans cette évolution, et vers quoi se tourner ?

