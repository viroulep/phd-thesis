\section{Organisation du contenu du manuscrit}\label{sec:intro:outline}

Le manuscrit est découpé en trois grandes parties.
La première partie traite des problématiques abordées par cette thèse, ainsi que des approches existantes sur les points techniques abordés.
Dans cette partie, le chapitre~\ref{chap:contexte} introduit les éléments de base nécessaires au déroulement de cette thèse : les architectures à mémoire partagée, les moyens existants de les programmer, et une description détaillée de certains outils et concepts techniques fondamentaux.
Le chapitre~\ref{chap:rw} revient sur l'état de l'art des techniques utilisées ou étendues par nos travaux.

La seconde partie regroupe nos travaux sur l'étude des machines NUMA, et l'amélioration de leur utilisation à travers OpenMP.
Le chapitre~\ref{chap:contrib:characterization} décrit nos efforts pour étudier le comportement des applications et de l'architecture sous jacente, et décrit l'orientation des travaux à la suite de nos observations.
Les extensions du langage et du support exécutif sont motivées, décrites et évaluées dans le chapitre~\ref{chap:contrib:openmp}.

Enfin la dernière partie se concentre sur les perspectives : le chapitre~\ref{chap:perspectives} revient sur l'évolution du matériel et du logiciel pendant la thèse, et discute des pistes de recherche envisageables pour pousser plus loin nos idées. Le chapitre~\ref{chap:conclusion} conclut notre travail et ce manuscrit.
