\section{Plan de thèse}\label{sec:intro:plan}
Ce manuscrit est découpé en cinq grandes parties :
\begin{enumerate}
\item \textbf{L'état de l'art}. Le chapitre~\ref{chap:rw:supervision} analyse les solutions actuelles capables de faire de l'observation de systèmes génériques. Ensuite, nous détaillons les travaux du domaine plus spécifique de la gestion des flux de données dans le chapitre~\ref{chap:rw:sgfd}.
\item \textbf{Modèle algébrique}. Le chapitre~\ref{chap:contrib:astral} présente les définitions d'\textit{Astral}, notre algèbre de gestion de données. Le chapitre~\ref{chap:validation:expressivite} analyse en détail l'expressivité offerte par \textit{Astral}, en la comparant à l'état de l'art, puis en démontrant des propriétés non triviales d'équivalence de requêtes. 
\item \textbf{Mise en œuvre et couplage relationnel}. Le chapitre~\ref{chap:contrib:astronef} décrit \textit{Astronef}, un intergiciel capable d'interpréter et d'évaluer une requête \textit{Astral}. Nous détaillons dans le chapitre~\ref{chap:contrib:asteroid} l'extension \textit{Asteroid} capable de manipuler un support persistant relationnel, complétant ainsi notre gestion de données pour l'observation.
\item \textbf{Expérimentations}. La mise en œuvre de notre solution d'observation dans le cadre du réseau local domestique est décrite dans le chapitre~\ref{chap:valid:domvision}. Puis, nous présentons dans le chapitre~\ref{chap:valid:perfs} des éléments d'évaluation de performances afin de démontrer que notre optimiseur permet d'évaluer de manière efficace des requêtes continues et des requêtes couplées avec un SGBD.
\item \textbf{Gestion des préférences}. Le chapitre~\ref{chap:prefs} présente une extension à notre contribution introduisant des opérateurs permettant de personnaliser les résultats en fonction de l'utilisateur. Nous présentons leur formalisation dans \textit{Astral} ainsi que deux implémentations que nous intégrons à \textit{Asteroid}. Des expérimentations permettent de comparer leurs performances.
\end{enumerate}

Finalement, le chapitre~\ref{chap:conclusion} conclut ce travail et présente quelques perspectives de recherche.
