\begin{savequote}[12cm]
<< Elle sert à quoi ta thèse ? >>

\end{savequote}
\chapter{Introduction}
\chaptertoc

L'évolution du calcul haute performance est aujourd'hui dirigée par les besoins croissant en calcul des applications de simulation numérique.
Ces applications sont omniprésentes dans l'industrie, et concernent parfois même directement le grand public.

Par exemple des secteurs comme l'aéronautique, les applications militaires, ou encore le nucléaire ont besoin de simuler des phénomènes à grande échelle, se traduisant souvent par la résolution de systèmes linéaires à plusieurs millions d'inconnues.
Les prévisions météorologiques à destination du grand public sont faites à l'aide d'applications simulant les interactions entre les différents éléments de l'atmosphère.
Il en va de même pour tout ce qui se rapporte à l'étude de la propagation des ondes sismiques dans le sol, ou de la prévision de l'impact d'un séisme sur un bassin de population, où le problème se traduit également par la résolution de systèmes linéaires.

Toutes ces simulations sont au final exécutées sur des supercalculateurs, et il n'y a pas de limite, a priori, au nombre de ressources qu'elles peuvent utiliser : que ce soit pour améliorer la précision de la simulation, ou augmenter la taille de l'ensemble simulé, elles pourront toujours bénéficier d'un plus grand nombre de ressources.
Avoir plus de ressources peut également permettre d'exécuter un plus grand nombre de simulations simultanément, voire de les coupler entre elles.

Si les supercalculateurs peuvent proposer plusieurs milliers de cœurs, ils sont en fait composés d'un grand nombre de nœuds de calcul avec un nombre de cœurs beaucoup plus faible.
Ces machines peuvent proposer, en plus de processeurs traditionnels, des accélérateurs plus ou moins spécifiques comme des GPUs ou des FPGA, formant une architecture dite \emph{hétérogène}.
La très grande majorité des nœuds de calcul intègrent plusieurs processeurs qui accèdent à une mémoire commune.

Contrairement aux processeurs du siècle dernier pour lesquels un changement de génération s'accompagnait d'une augmentation de leur fréquence de fonctionnement, l'évolution des processeurs contemporains se traduit aujourd'hui par la multiplication du nombres de cœurs de calcul qu'ils embarquent.
Pour illustrer ce phénomène il suffit de regarder par exemple la gamme de produits proposés par Intel : la première génération de Pentium 4 - Willamette - lancée par Intel en 2000 était constituée d'un unique cœur cadencé à 1.5GHz. 6 ans plus tard, la dernière génération de Pentium 4 - Cedar Mill - était également consituée d'un seul cœur, mais cette fois cadencé à 3.6 GHz. 10 ans plus tard en 2016, les processeurs de la génération Skylake d'Intel i7 ne dépassent pas les 3.4GHz de fréquence, mais tous ont 4 cœurs physiques au lieu d'un seul.

Avec ce changement de design, les modalités d'accès à la mémoire ont été repensées pour éviter les goulots d'étranglement se formant lors des accès concurrents de plusieurs cœurs au même bus mémoire.

Pour éviter trop de contention sur le bus mémoire, la mémoire est divisée en plusieurs bancs physiques différents, avec chacun leur contrôleur.
Sur chacun de ces bancs, les processeurs disposent d'un cache partagé commun en plus des caches privés à chaque processeur.
La conséquence directe de ce changement est que le temps d'accès à la mémoire est devenu non uniforme : il dépend directement de quel processeur essaye d'accéder à quelle partie de la mémoire.
On appelle ces architectures NUMA (pour \emph{Non Uniform Memory Access}) et elles sont aujourd'hui la brique de base pour créer des supercalculateurs.

Plusieurs modèles de programmation permettent de cibler ce genre d'architectures.
Les boucles parallèles et les tâches avec dépendances sont deux types de constructions très utilisées, et elles sont présentes dans la majorité des modèles de programmation.

Les boucles parallèles sont particulièrement adaptées aux applications régulières où les temps d'exécution de chaque itération est facilement prévisible.
Les tâches avec dépendances permettent d'exprimer un parallélisme à grain fin, et sont particulièrement adaptées dans le cas d'applications dont certains calculs peuvent être imprévisibles et entrainer un déséquilibre de charge.

Cela peut être expliqué par la manière donc sont ordonnancées ces deux types d'applications par le support exécutif~: dans le cas de boucle parallèle, les itérations à exécuter sont généralement découpées équitablement entre les cœurs de calcul.
Dans le cas des tâches avec dépendances, l'une des techniques les plus communes est l'ordonnancement par vol de travail~: l'application est exprimée comme un graphe de flot de données, chaque sous-partie - tâche - consommant et produisant des données. 
À chaque fois qu'un processeur devient inactif, il va récupérer --- \emph{voler} --- une tâche disponible pour l'exécuter.
Pour minimiser les périodes d'inactivités des processeurs, il faudrait donc qu'il y ait toujours des tâches disponibles à être exécuter.
Pour que le vol de travail soit efficace, il faut donc pouvoir exprimer un maximum de parallélisme : plus il y a de parallélisme, plus il y a de tâches, mieux on est capable d'équilibrer la charge au cours de l'exécution.

Même dans le cas d'une application où le travail peut sembler régulier et où une parallélisation par boucles parallèles semblerait convenir, le côté NUMA des architectures ajoute une variabilité dans le temps des accès mémoires, ce qui rend le temps d'exécution difficile à prévoir et peut entrainer du même coup un déséquilibre de charge.
L'utilisation des tâches avec dépendances semblent donc tout à fait appropriée lorsque l'on cible des architectures NUMA.

OpenMP est le standard \emph{de-facto} pour les architectures à mémoire partagée, qui a récemment évolué pour supporter les tâches avec dépendance en plus des boucles parallèles.
C'est le modèle que nous avons utilisé comme support pour nos idées, sachant qu'elles pourraient être transposées dans les autres modèles de programmation puisqu'ils utilisent des concepts similaires.
Les modèles de programmation existant présentent un manque lorsqu'il s'agit d'exploiter efficacement les machines NUMA.
Le programmeur doit faire de gros efforts pour effectuer des optimisations spécifiques peu portables, par exemple via des bibliothèques externes pour contrôler précisément le placement des données.
Les outils standards et non intrusifs permettent simplement de distribuer les pages de la mémoire sur les différentes parties physiques~: ce n'est pas suffisant, cela permet de diviser et répartir les données et donc de diminuer la contention sur les bus mémoire, mais cela ne permet malheureusement pas aux données dépassant la taille d'une page de rester groupées proches les une des autres, ce qui est un point clé pour avoir une bonne localité des données.
La localité des données est essentielle pour les performances, mais lorsqu'il faut équilibrer la charge et donc déplacer des tâches, elle est difficile (voire impossible) à conserver.
La bonne exploitation des machines NUMA repose donc sur deux objectifs incompatibles~: l'équilibrage de charge et la localité des données.

Cette thèse est axée sur l'amélioration des standards et techniques pour l'exploitation des machines NUMA, et cela passe par plusieurs étapes : tout d'abord fournir au programmeur les moyens de comprendre et analyser le comportement des parties critiques de son application.
Ensuite lui permettre de fournir plus d'information au support exécutif, principalement en lui permettant d'exprimer une \emph{affinité} entre ses tâches et les ressources de la machine.
Et enfin proposer des techniques d'ordonnancement prenant en compte ces informations, dans le but d'améliorer efficacement les performances globales de l'application.





\section{Objectifs}\label{sec:intro:objectives}

L'objectif principal de cette thèse était d'étudier les améliorations possibles de l'exploitation des architectures NUMA, à l'aide d'un modèle de programmation à base de tâches.
Cela a été découpé en trois axes de travail.


\subsection*{Analyse du comportement d'applications sur machine NUMA}

Avant de pouvoir penser aux améliorations, il faut commencer par analyser les différents points améliorables, tant du côté logiciel que matériel.
Si l'on souhaitait cibler les modèles de programmation à base de tâches, il fallait néanmoins choisir l'un des modèles existants pour l'étude concrète, et ce choix s'est porté sur OpenMP.

Face à l'absence de suite de benchmarks ciblant certaines fonctionnalités d'OpenMP que nous souhaitions utiliser, nous avons commencé par publier une suite de benchmarks, les KASTORS~\cite{Virouleau2014}.
Les applications présentes dans cette suite ont été adaptées depuis des applications existantes, afin d'utiliser les constructions dont nous avions besoin, et qui sont aujourd'hui utilisées par la communauté.

Nous avons également écrit un outil plus générique, NOMDEL4OUTIL, afin de pouvoir évaluer plus précisément certaines parties ciblées des applications.
L'objectif derrière cet outil est de se libérer de certaines contraintes se présentant lors de l'observation de ces parties de code au cours de l'exécution complète du programme.
En extrayant les parties critiques on peut faire varier à loisir, et précisément, des paramètres tels que le placement des données ou le jeu de données en entrée, afin d'analyser finement quels impacts ils ont et comment l'architecture sous-jacente réagit.


\subsection*{Quelles améliorations pour l'utilisateur et le support exécutif ?}

À partir des conclusions tirées du point précédent, le second objectif était de trouver, proposer, et évaluer des améliorations possibles, tant pour l'utilisateur que pour le support exécutif.

Ces réflexions ont donné lieu à deux contributions : la première est axée sur la réponse au besoin de l'utilisateur, en proposant une clause |affinity| pour les tâches OpenMP~\cite{Virouleau2016b}.
Cette clause a pour but de permettre à l'utilisateur d'indiquer explicitement un lien fort entre une tâche et une ressource de la machine, que ce soit un coeur, un noeud, ou une donnée.
La seconde est axée sur l'extension du support exécutif~\cite{Virouleau2016a}, d'une part pour faciliter la distribution des données sur la machine, et d'autre part pour exploiter les informations disponibles sur les données manipulées par les tâches, dans le but d'améliorer la localité des données au cours de l'exécution.


\subsection*{Place des travaux dans l'évolution du matériel et du logiciel}

Au cours de cette thèse les architectures NUMA ont évolué, on peut alors se demander dans quelle mesure l'évolution du matériel impacte les travaux de cette thèse.
Le dernier objectif est donc d'analyser les travaux effectués - voir les compléter - afin de proposer des approches indépendantes du matériel, dans le but de faciliter le travail du programmeur et du développeur de support exécutif à l'avenir.

Parmi nos travaux, NOMDEL4OUTIL propose une approche transposable à tout type d'architecture, et nous avons fait des travaux préliminaires sur un simulateur, qui à partir des données de cet outil peut donner un aperçu des performances de l'application complète.
Le coût - en temps - de changer l'implémentation utilisée par un support exécutif est assez important, ces travaux préliminaires ont pour objectif de donner un premier aperçu de l'impact que pourrait avoir une modification de l'ordonnancement par le support exécutif.
Cela permet ainsi, lors du "portage" d'une application ou d'un support exécutif sur une nouvelle architecture, d'estimer son comportement, et évaluer si des changements dans l'un des deux sont nécessaires.

\section{Organisation du contenu du manuscrit}\label{sec:intro:outline}

Le manuscrit est découpé en trois grandes parties.
La première partie traite des problématiques abordées par cette thèse, ainsi que des approches existantes sur les points techniques abordés.
Dans cette partie, le chapitre~\ref{chap:contexte} introduit les éléments de base nécessaires au déroulement de cette thèse : les architectures à mémoire partagée, les moyens existants de les programmer, et une description détaillée de certains outils et concepts techniques fondamentaux.
Le chapitre~\ref{chap:rw} revient sur l'état de l'art des techniques utilisées ou étendues par nos travaux.

La seconde partie regroupe nos travaux sur l'étude des machines NUMA, et l'amélioration de leur utilisation à travers OpenMP.
Le chapitre~\ref{chap:contrib:characterization} décrit nos efforts pour étudier le comportement des applications et de l'architecture sous jacente, et décrit l'orientation des travaux à la suite de nos observations.
Les extensions du langage et du support exécutif sont motivées, décrites et évaluées dans le chapitre~\ref{chap:contrib:openmp}.

Enfin la dernière partie se concentre sur les perspectives : le chapitre~\ref{chap:conclusion} revient sur l'évolution du matériel et du logiciel pendant la thèse, et discute des pistes de recherche envisageables pour pousser plus loin nos idées, avant de conclure notre travail et ce manuscrit.

