\begin{savequote}[6cm]
<< What do you work on?  >>
\qauthor{someone}
\end{savequote}
\chapter{Introduction}
\chaptertoc

L'évolution du calcul haute performance est aujourd'hui dirigée par les besoins des applications de simulation numérique.
Ces applications sont omniprésentes dans l'industrie, et concerne parfois même directement le grand public.

Par exemple des secteurs comme l'aéronautique, les applications militaires, ou encore le nucléaire ont besoin de simuler des phénomènes à grande échelle, se traduisant parfois par la résolution de systèmes de systèmes linéaires à plusieurs millions d'inconnues.
Les prévisions météorologiques à destination du grand public sont faites à l'aide d'applications simulant les intéractions entre les différents éléments de l'atmosphère.
Il en va de même pour tout ce qui se rapporte à l'étude de la propagation des ondes sismiques dans le sol, ou de la prévision de l'impact d'un séisme sur un bassin de population.

Toutes ces simulations sont au final exécutées sur des supercalculateurs, et il n'y a pas de limite aux nombres de ressources qu'elle peuvent utiliser : que ce soit pour améliorer la précision de la simulation, ou augmenter la taille de l'ensemble simulé, elles pourront toujours bénéficier d'un plus grand nombres de ressources.
De plus un grand nombre de ressources peut également permettre d'exécuter un plus grand nombre de simulations simultanément, voire de les coupler entre elles.

Si les supercalculateurs peuvent proposer plusieurs milliers de coeurs, ils sont en fait composés d'un grand nombre de noeuds de calcul avec un nombre de coeurs beaucoup plus faible.
Ces machines peuvent proposer, en plus de processeurs traditionnels, accélérateurs plus ou moins spécifiques comme des GPUs ou des FPGA, formant une architecture dite \emph{hétérogène}.
La très grande majorité des noeuds de calcul intègrent plusieurs processeurs qui accèdent à une mémoire commune.

Contrairement aux processeurs du siècle dernier pour lesquels un changement de génération s'accompagnait d'une augmentation de leur fréquence de fonctionnement, l'évolution des processeurs contemporains se traduit aujourd'hui par la multiplication du nombres de coeurs de calcul qu'ils embarquent.
Pour illustrer ce phénomène il suffit de regarder par exemple la gamme de produits proposés par Intel : la première génération de Pentium 4 - Willamette - lancée par Intel en 2000 était constituée d'un unique coeur cadencé à 1.5GHz. 6 ans plus tard, la dernière génération de Pentium 4 - Cedar Mill - était également consituée d'un seul coeur, mais cette fois cadencé à 3.6 GHz. 10 ans plus tard en 2016, les processeurs de la génération Skylake d'Intel i7 ne dépassent pas les 3.4GHz de fréquence, mais tous ont 4 coeurs physiques au lieu d'un seul.

Avec ce changement de design, les modalités d'accès à la mémoire ont été repensées pour éviter les goulots d'étranglement se formant lors des accès concurrents de plusieurs coeurs au même bus mémoire. Le moyen qui a été trouvé pour éviter trop de contention sur le bus mémoire fut de diviser la mémoire en plusieurs bloc physique différents, avec chacun leur contrôleur.
La conséquence directe de ce changement est que le temps d'accès à la mémoire est devenu non uniforme : il dépend directement de quel processeur essaye d'accéder à quelle partie de la mémoire.
On appelle ce genre d'architectures NUMA (pour \emph{Non Uniform Memory Access}) et elles sont aujourd'hui la brique de base pour créer des supercalculateurs.

L'amélioration des performances générales d'un supercalculateur passe donc directement par l'optimisation de l'exploitation des machines NUMA.
Plusieurs techniques de programmation permette de cibler ce genre d'architectures.
On peut notamment citer des techniques statiques à base de boucles parallèles.
En pratique cela marche particulièrement bien pour les applications où toutes les itérations des boucles sont régulières et où leur temps d'exécution est prévisible.
Néanmoins il existe beaucoup de cas où ce n'est pas le cas.
L'exemple typique est celui des applications de parcours de graphes, qui sont généralement irrégulières.
Mais même dans le cas d'une application où le travail peut sembler régulier, comme par exemple une multiplication de matrices, l'ajout d'une variabilité dans le temps des accès mémoires rend le temps d'exécution difficile à prévoir, et peut entrainer un déséquilibrage de charge.


Les techniques d'ordonnancement dynamiques peuvent garantir une utilisation plus efficace des ressources.
L'une des plus communes est l'ordonnancement par vol de travail : l'application est exprimée comme un graphe de flot de données, chaque sous partie - tâche - consommant et produisant des données.
Un programme dédié - le support exécutif - a la charge d'exécuter ce graphe : à chaque fois qu'un processeur devient inactif, il va récupérer une tâche disponible pour l'exécuter.
Pour que cela soit efficace, il faut pouvoir exprimer un maximum de parallélisme : plus il y a de parallélisme, mieux on est capable d'équilibrer la charge au cours de l'exécution.

Les modèles de programmation standard on su s'adapter : par exemple OpenMP, qui est le standard \emph{de-facto} pour ce genre d'architectures, ne proposait au départ que du parallélisme de boucles. Les dernières versions proposent de la programmation par tâche avec dépendances de données.
En revanche ces modèles de programmation présentent toujours un manque lorsqu'il s'agit d'exploiter efficacement les machines NUMA.
Le programmeur doit donc faire de gros efforts pour effectuer des optimisations spécifiques peu portables, par exemple via des bibliothèques externes pour contrôler précisément le placement des données.
Les outils standard et non intrusifs permettent simplement de distribuer les pages de la mémoire sur les différentes parties physiques : cela a pour avantage d'améliorer l'uniformité des temps d'accès à la mémoire, mais pas d'améliorer le temps d'accès moyen.
Rendant donc les processeurs uniformément mauvais vis à vis de l'accès à la mémoire.

Cette thèse est axée sur l'amélioration des standards et techniques pour l'exploitation des machines NUMA, et cela passe par plusieurs étapes : tout d'abord fournir au programmeur les moyens de comprendre et analyser le comportement des parties critiques de son application.
Ensuite lui permettre de fournir plus d'information au support exécutif, principalement en lui permettant d'exprimer une \emph{affinité} entre ses tâches et les ressources de la machine.
En enfin en proposant des techniques d'ordonnancement prenant en compte ces informations, dans le but d'améliorer efficacement les performances globales de l'application.





\section{Objectifs}\label{sec:intro:objectives}

L'objectif principal de cette thèse était d'étudier les améliorations possibles de l'exploitation des architectures NUMA, à l'aide d'un modèle de programmation à base de tâches.
Cela a été découpé en trois axes de travail.


\subsection*{Analyse du comportement d'applications sur machine NUMA}

Avant de pouvoir penser aux améliorations, il faut commencer par analyser les différents points améliorables, tant du côté logiciel que matériel.
Si l'on souhaitait cibler les modèles de programmation à base de tâches, il fallait néanmoins choisir l'un des modèles existants pour l'étude concrète, et ce choix s'est porté sur OpenMP.

Face à l'absence de suite de benchmarks ciblant certaines fonctionnalités d'OpenMP que nous souhaitions utiliser, nous avons commencé par publier une suite de benchmarks, les KASTORS~\cite{Virouleau2014}.
Les applications présentes dans cette suite ont été adaptées depuis des applications existantes, afin d'utiliser les constructions dont nous avions besoin, et qui sont aujourd'hui utilisées par la communauté.

Nous avons également écrit un outil plus générique, \outil, afin de pouvoir évaluer plus précisément certaines parties ciblées des applications.
L'objectif derrière cet outil est de se libérer de certaines contraintes se présentant lors de l'observation de ces parties de code au cours de l'exécution complète du programme.
En extrayant les parties critiques on peut faire varier à loisir, et précisément, des paramètres tels que le placement des données ou le jeu de données en entrée, afin d'analyser finement quels impacts ils ont et comment l'architecture sous-jacente réagit.


\subsection*{Quelles améliorations pour l'utilisateur et le support exécutif ?}

À partir des conclusions tirées du point précédent, le second objectif était de trouver, proposer, et évaluer des améliorations possibles, tant pour l'utilisateur que pour le support exécutif.

Ces réflexions ont donné lieu à deux contributions : la première est axée sur la réponse au besoin de l'utilisateur, en proposant une clause |affinity| pour les tâches OpenMP~\cite{Virouleau2016b}.
Cette clause a pour but de permettre à l'utilisateur d'indiquer explicitement un lien fort entre une tâche et une ressource de la machine, que ce soit un coeur, un noeud, ou une donnée.
La seconde est axée sur l'extension du support exécutif~\cite{Virouleau2016a}, d'une part pour faciliter la distribution des données sur la machine, et d'autre part pour exploiter les informations disponibles sur les données manipulées par les tâches, dans le but d'améliorer la localité des données au cours de l'exécution.


\subsection*{Place des travaux dans l'évolution du matériel et du logiciel}

Au cours de cette thèse les architectures NUMA ont évolué, on peut alors se demander dans quelle mesure l'évolution du matériel impacte les travaux de cette thèse.
Le dernier objectif est donc d'analyser les travaux effectués - voir les compléter - afin de proposer des approches indépendantes du matériel, dans le but de faciliter le travail du programmeur et du développeur de support exécutif à l'avenir.

Parmi nos travaux, \outil propose une approche transposable à tout type d'architecture, et nous avons fait des travaux préliminaires sur un simulateur, qui à partir des données de cet outil peut donner un aperçu des performances de l'application complète.
Le coût - en temps - de changer l'implémentation utilisée par un support exécutif est assez important, ces travaux préliminaires ont pour objectif de donner un premier aperçu de l'impact que pourrait avoir une modification de l'ordonnancement par le support exécutif.
Cela permet ainsi, lors du "portage" d'une application ou d'un support exécutif sur une nouvelle architecture, d'estimer son comportement, et évaluer si des changements dans l'un des deux sont nécessaires.

\section{Plan}\label{sec:intro:outline}

TODO

