\begin{savequote}[6cm]
<< What do you work on?  >>
\qauthor{someone}
\end{savequote}
\chapter{Introduction}
\chaptertoc

Le monde de l'informatique évolue rapidement, qui plus est quand il s'agit de la pièce maitresse de la machine : le processeur !

Jusqu'à il y a environ 10 ans, les fabricants de processeurs se livraient à une course à la plus haute fréquence.
Jusqu'au moment où ce n'était plus rentable au point de vue thermique ou énergétique. À partir de ce moment là, les fabricants ont basculé leur intérêt vers la multiplication des coeurs au sein d'un processeur, et ont stabilisé la fréquence individuelle de chaque coeur.

Pour illustrer ce phénomène il suffit de regarder par exemple la gamme de produits proposés par Intel : la première série de Pentium 4 - Willamette - lancée par Intel en 2000 avait un unique coeur cadencé à 1.5GHz. 6 ans plus tard, la dernière génération de Pentium 4 - Cedar Mill - avait également un seul coeur, mais cette fois cadencé à 3.6 GHz.

10 ans plus tard en 2016, les processeurs de la génération Skylake d'Intel i7 ne dépassent pas les 3.4GHz de fréquence, mais tous ont 4 coeurs physiques au lieu d'un seul.

Pendant ce laps de temps, certains objectifs sont restés les même :
\begin{itemize}
  \item on souhaite toujours exploiter le matériel au plus fort de leur potentiel, que ce soit du point de vue énergétique, du point de vue de la performance, ou du compromis des deux.
  \item on souhaite toujours créer des machines de plus en plus puissantes, afin que les programmes s'exécutent plus vite.
\end{itemize}

En revanche un certains nombres de problématiques sont apparues : l'exploitation efficace de plusieurs processeurs est complètement différentes de l'exploitation d'un seul processeurs.

Lorsqu'on traite avec un seul coeur, on s'attend à ce que le compilateur optimise le code séquentiel en fonction de l'architecture.
Pour exploiter plusieurs coeurs il faut pouvoir exprimer du \emph{parallélisme} au sein de l'application, c'est à dire décrire les parties séquentielles du programme qui veut s'exécuter en même temps, tout en gardant un comportement correct du programme.
Le compilateur ne peut pas spontanément trouver ça, il faut donc une modification explicite de la part du programmeur.

On peut avoir plusieurs coeurs au sein d'un même processeur, plusieurs processeurs au sein d'une machine, mais on peut aussi avoir plusieurs machines inter connectées, qui forme une seule "grosse" machine que l'on appelle \emph{supercalculateur}.

La manière d'exprimer du parallélisme est différente en fonction de quel type de machine on cible :
\begin{itemize}
  \item Lorsque l'on cible une seule machine avec un ou plusieurs processeurs, généralement tout les coeurs des processeurs peuvent accéder de manière transparente à la mémoire : on dit que ce type d'architecture est à \emph{mémoire partagée}.
  \item Lorsque l'on cible un supercalculateur, chaque machine aura une \emph{mémoire partagée}, mais ne verra que sa propre mémoire, et pas celle des autres machines. Cela ne les empêche évidement pas de se transférer des données entre elles. On dit que ce type d'architecture est à \emph{mémoire distribuée}.
\end{itemize}

L'augmentation du nombre de coeurs par processeur et par machine a donné naissance à plusieurs problématiques et thèmes de recherche :

\paragraph{La mémoire}
L'augmentation du nombres de coeurs accédant à la même mémoire partagée augmente la contention sur le bus mémoire pour y accéder.
Pour pouvoir gérer un plus grand nombre de coeurs tout en limitant la contention, la majorité des architectures à mémoire partagée récente divisent physiquement la mémoire en sous parties appelées \emph{noeud}.
Chacun de ces noeuds possède son propre controleur mémoire, ainsi qu'un sous ensemble des coeurs du système.
Ces noeuds sont liés ensemble via un système d'interconnexion, afin que cette division soit transparent pour le système d'exploitation, qui ne voit toujours qu'une unique machine.
Pour un coeur donné, l'accès à la mémoire située sur son propre noeud est plus rapide que sur l'un des autres noeuds. Cela veut donc dire que le temps d'accès à la mémoire n'est pas uniforme en fonction de quelle partie est ciblée.
On qualifie ce type d'architectures de "NUMA", pour "Non-Uniform Memory Access".

\paragraph{L'énergie}

 %Note : ~5000 kWh par an par foyer (soit ~570 watts), les rangs 1 et 2 peuvent consommer ~15kW
Que l'on utilise une petite machine à mémoire partagée de 16 coeurs, ou un gros supercalculateur avec 1 millions de coeurs, faire attention à l'énergie consommée est critique.
L'augmentation continue du nombre de coeurs ne pourra pas se faire sans porter une attention particulière à l'énergie : aujourd'hui il faut l'équivalent de la puissance d'environ 25 familles française pour faire tourner certains des premiers supercalculateur du top500.

\paragraph{Les outils}

Les langages de programmation "classique" tels que C/C++/Fortran sont principalement axés sur l'exécution séquentielle du code.
Les programmeurs ont besoin d'exprimer du parallélisme, donc des extensions aux langages existant ou des nouveaux langages doivent être créés et testés. Si possible de manière portable.

Ces langages ou extensions reposent très souvent sur un programme spécifique pour exécuter une application : le support exécutif.
Le but d'un programme de ce type est d'optimiser l'utilisation des ressources disponibles, tout en conservant la sémantique et la "correctness" (TODO : trouver une traduction) du programme.
Augmenter le nombre de ressources signifient plus de synchronisations, et plus de choix d'ordonnancement à effectuer.

\paragraph{Le matériel dédié}

Une partie de l'optimisation des machines passe par l'amélioration du matériel existant.

Il est commun d'utiliser plusieurs types de matériel pour pouvoir répondre efficacement à des besoins spécifiques, c'est pour cela que les supercalculateurs actuels embarquent généralement un ensemble d'accélérateurs tels que des GPUs et des FPGAs, en plus des CPUs traditionnels.

Les systèmes d'interconnexion sont aussi de bon candidats pour des améliorations et optimisations.
Améliorer la latence et la bande passante entre plusieurs machines ou au sein d'une même machines permet toujours d'améliorer les performances, et les technologies impliquées sont très différentes.



Comme on peut le voir il y a un grand nombre d'opportunités pour faire de la recherche dans le calcul haute performance.
La section suivante sera dédiée à préciser les différents thèmes de recherche qui ont été abordés au cours de la thèse, préciser le contexte dans laquelle elle s'est déroulée, et préciser les objectifs auxquels elles devaient répondre.



\section{Objectifs}\label{sec:intro:objectives}

L'objectif principal de cette thèse était d'étudier les améliorations possibles de l'exploitation des architectures NUMA, à l'aide d'un modèle de programmation à base de tâches.
Cela a été découpé en trois axes de travail.


\subsection*{Analyse du comportement d'applications sur machine NUMA}

Avant de pouvoir penser aux améliorations, il faut commencer par analyser les différents points améliorables, tant du côté logiciel que matériel.
Si l'on souhaitait cibler les modèles de programmation à base de tâches, il fallait néanmoins choisir l'un des modèles existants pour l'étude concrète, et ce choix s'est porté sur OpenMP.

Face à l'absence de suite de benchmarks ciblant certaines fonctionnalités d'OpenMP que nous souhaitions utiliser, nous avons commencé par publier une suite de benchmarks, les KASTORS~\cite{Virouleau2014}.
Les applications présentes dans cette suite ont été adaptées depuis des applications existantes, afin d'utiliser les constructions dont nous avions besoin, et qui sont aujourd'hui utilisées par la communauté.

Nous avons également écrit un outil plus générique, NOMDEL4OUTIL, afin de pouvoir évaluer plus précisément certaines parties ciblées des applications.
L'objectif derrière cet outil est de se libérer de certaines contraintes se présentant lors de l'observation de ces parties de code au cours de l'exécution complète du programme.
En extrayant les parties critiques on peut faire varier à loisir, et précisément, des paramètres tels que le placement des données ou le jeu de données en entrée, afin d'analyser finement quels impacts ils ont et comment l'architecture sous-jacente réagit.


\subsection*{Quelles améliorations pour l'utilisateur et le support exécutif ?}

À partir des conclusions tirées du point précédent, le second objectif était de trouver, proposer, et évaluer des améliorations possibles, tant pour l'utilisateur que pour le support exécutif.

Ces réflexions ont donné lieu à deux contributions : la première est axée sur la réponse au besoin de l'utilisateur, en proposant une clause |affinity| pour les tâches OpenMP~\cite{Virouleau2016b}.
Cette clause a pour but de permettre à l'utilisateur d'indiquer explicitement un lien fort entre une tâche et une ressource de la machine, que ce soit un coeur, un noeud, ou une donnée.
La seconde est axée sur l'extension du support exécutif~\cite{Virouleau2016a}, d'une part pour faciliter la distribution des données sur la machine, et d'autre part pour exploiter les informations disponibles sur les données manipulées par les tâches, dans le but d'améliorer la localité des données au cours de l'exécution.


\subsection*{Place des travaux dans l'évolution du matériel et du logiciel}

Au cours de cette thèse les architectures NUMA ont évolué, on peut alors se demander dans quelle mesure l'évolution du matériel impacte les travaux de cette thèse.
Le dernier objectif est donc d'analyser les travaux effectués - voir les compléter - afin de proposer des approches indépendantes du matériel, dans le but de faciliter le travail du programmeur et du développeur de support exécutif à l'avenir.

Parmi nos travaux, NOMDEL4OUTIL propose une approche transposable à tout type d'architecture, et nous avons fait des travaux préliminaires sur un simulateur, qui à partir des données de cet outil peut donner un aperçu des performances de l'application complète.
Le coût - en temps - de changer l'implémentation utilisée par un support exécutif est assez important, ces travaux préliminaires ont pour objectif de donner un premier aperçu de l'impact que pourrait avoir une modification de l'ordonnancement par le support exécutif.
Cela permet ainsi, lors du "portage" d'une application ou d'un support exécutif sur une nouvelle architecture, d'estimer son comportement, et évaluer si des changements dans l'un des deux sont nécessaires.

\section{Organisation du contenu du manuscrit}\label{sec:intro:outline}

Le manuscrit est découpé en trois grandes parties.
La première partie traite des problématiques abordées par cette thèse, ainsi que des approches existantes sur les points techniques abordés.
Dans cette partie, le chapitre~\ref{chap:contexte} introduit les éléments de base nécessaires au déroulement de cette thèse : les architectures à mémoire partagée, les moyens existants de les programmer, et une description détaillée de certains outils et concepts techniques fondamentaux.
Le chapitre~\ref{chap:rw} revient sur l'état de l'art des techniques utilisées ou étendues par nos travaux.

La seconde partie regroupe nos travaux sur l'étude des machines NUMA, et l'amélioration de leur utilisation à travers OpenMP.
Le chapitre~\ref{chap:contrib:characterization} décrit nos efforts pour étudier le comportement des applications et de l'architecture sous jacente, et décrit l'orientation des travaux à la suite de nos observations.
Les extensions du langage et du support exécutif sont motivées, décrites et évaluées dans le chapitre~\ref{chap:contrib:openmp}.

Enfin la dernière partie se concentre sur les perspectives : le chapitre~\ref{chap:conclusion} revient sur l'évolution du matériel et du logiciel pendant la thèse, et discute des pistes de recherche envisageables pour pousser plus loin nos idées, avant de conclure notre travail et ce manuscrit.

