\begin{savequote}[6cm]
<< What do you work on?  >>
\qauthor{someone}
\end{savequote}
\chapter{Introduction}
\chaptertoc

Computer science evolves fast, especially when it comes to the most basic and necessary element: the microprocessor.
Until around 10 years ago, microprocessors manufacturers were able to increase the frequencies to increase the overall performances of the microprocessor.
At some point it became less efficient, and if we look at how architectures changed since then, we can notice the focus of microprocessors manufacturers has moved from increasing the chip frequency to increasing the number of chips.

Some goals haven't changed:
 - we still want to exploit hardwares to the best of their potential, either energy-wise, performance-wise, or both.
 - we still need more powerful machines to make programs run faster and cheaper.

However a lot of related issues have changed: we can't deal with multiple processors just the same way as we deal with a single processor, and we can't deal with thousands of processors just the same way as we deal with 10 processors.

TODO: où et comment formuler la distinction entre mémoire partagée et mémoire distribuée ?

This increase in the number of processing units has lead to several other changes :

 - Memory: increasing the number of CPUs accessing the same RAM increases the contention on the memory bus. To accommodate a higher number of CPUs while still avoiding too much contention, most of recent shared memory architectures split their memory into several parts, usually named "nodes". Each of these node has their own memory controller, as well as a group of CPUs. These nodes are linked together through an interconnect, so that the operating system only see one single big machine.
 It is faster for a CPU to access the memory on its own node than the one on a remote node, which is why we refer to this kind of architecture as "Non-Uniform Memory Access" (NUMA).
 %Note : ~5000 kWh par an par foyer (soit ~570 watts), les rangs 1 et 2 peuvent consommer ~15kW
 - Energy: regardless of if we are talking about a small 16 cores shared memory computer or a big 1 million cores supercomputer, energy is a critical factor. Constantly increasing the number of processing units won't continue to happen without taking care of the energy issue: we need the power of ~25 families in France to power some of the first supercomputers of the top500.
 - Language: regular languages such as C/C++/Fortran are designed for sequential execution. When targeting such large architectures, programmers need to express parallelism, so new languages or extensions to existing languages have to be designed and experimented. Preferably in a portable way.
 - Parallel programming languages rely on a specific software - the runtime - to take care of the load balancing of the program, and make sure every resource is used at its best. Increasing the number of working cores can lead to more complex synchronizations and scheduling decision.
 - Other hardware changes: supercomputers use a mix of CPUs and GPUs, FPGA. Faster interconnect for supercomputer. Faster cache coherent interconnect for shared memory.



We can see that there are a lot of opportunities for research about the general "High Performance Computing" topic, and the next section will describe what was the objectives of this thesis.


\section{Objectifs}\label{sec:intro:objectives}

L'objectif principal de cette thèse était d'étudier les améliorations possibles de l'exploitation des architectures NUMA, à l'aide d'un modèle de programmation à base de tâches.
Cela a été découpé en trois axes de travail.


\subsection*{Analyse du comportement d'applications sur machine NUMA}

Avant de pouvoir penser aux améliorations, il faut commencer par analyser les différents points améliorables, tant du côté logiciel que matériel.
Si l'on souhaitait cibler les modèles de programmation à base de tâches, il fallait néanmoins choisir l'un des modèles existants pour l'étude concrète, et ce choix s'est porté sur OpenMP.

Face à l'absence de suite de benchmarks ciblant certaines fonctionnalités d'OpenMP que nous souhaitions utiliser, nous avons commencé par publier une suite de benchmarks, les KASTORS~\cite{Virouleau2014}.
Les applications présentes dans cette suite ont été adaptées depuis des applications existantes, afin d'utiliser les constructions dont nous avions besoin, et qui sont aujourd'hui utilisées par la communauté.

Nous avons également écrit un outil plus générique, NOMDEL4OUTIL, afin de pouvoir évaluer plus précisément certaines parties ciblées des applications.
L'objectif derrière cet outil est de se libérer de certaines contraintes se présentant lors de l'observation de ces parties de code au cours de l'exécution complète du programme.
En extrayant les parties critiques on peut faire varier à loisir, et précisément, des paramètres tels que le placement des données ou le jeu de données en entrée, afin d'analyser finement quels impacts ils ont et comment l'architecture sous-jacente réagit.


\subsection*{Quelles améliorations pour l'utilisateur et le support exécutif ?}

À partir des conclusions tirées du point précédent, le second objectif était de trouver, proposer, et évaluer des améliorations possibles, tant pour l'utilisateur que pour le support exécutif.

Ces réflexions ont donné lieu à deux contributions : la première est axée sur la réponse au besoin de l'utilisateur, en proposant une clause |affinity| pour les tâches OpenMP~\cite{Virouleau2016b}.
Cette clause a pour but de permettre à l'utilisateur d'indiquer explicitement un lien fort entre une tâche et une ressource de la machine, que ce soit un coeur, un noeud, ou une donnée.
La seconde est axée sur l'extension du support exécutif~\cite{Virouleau2016a}, d'une part pour faciliter la distribution des données sur la machine, et d'autre part pour exploiter les informations disponibles sur les données manipulées par les tâches, dans le but d'améliorer la localité des données au cours de l'exécution.


\subsection*{Place des travaux dans l'évolution du matériel et du logiciel}

Au cours de cette thèse les architectures NUMA ont évolué, on peut alors se demander dans quelle mesure l'évolution du matériel impacte les travaux de cette thèse.
Le dernier objectif est donc d'analyser les travaux effectués - voir les compléter - afin de proposer des approches indépendantes du matériel, dans le but de faciliter le travail du programmeur et du développeur de support exécutif à l'avenir.

Parmi nos travaux, NOMDEL4OUTIL propose une approche transposable à tout type d'architecture, et nous avons fait des travaux préliminaires sur un simulateur, qui à partir des données de cet outil peut donner un aperçu des performances de l'application complète.
Le coût - en temps - de changer l'implémentation utilisée par un support exécutif est assez important, ces travaux préliminaires ont pour objectif de donner un premier aperçu de l'impact que pourrait avoir une modification de l'ordonnancement par le support exécutif.
Cela permet ainsi, lors du "portage" d'une application ou d'un support exécutif sur une nouvelle architecture, d'estimer son comportement, et évaluer si des changements dans l'un des deux sont nécessaires.

\section{Organisation du contenu du manuscrit}\label{sec:intro:outline}

Le manuscrit est découpé en trois grandes parties.
La première partie traite des problématiques abordées par cette thèse, ainsi que des approches existantes sur les points techniques abordés.
Dans cette partie, le chapitre~\ref{chap:contexte} introduit les éléments de base nécessaires au déroulement de cette thèse : les architectures à mémoire partagée, les moyens existants de les programmer, et une description détaillée de certains outils et concepts techniques fondamentaux.
Le chapitre~\ref{chap:rw} revient sur l'état de l'art des techniques utilisées ou étendues par nos travaux.

La seconde partie regroupe nos travaux sur l'étude des machines NUMA, et l'amélioration de leur utilisation à travers OpenMP.
Le chapitre~\ref{chap:contrib:characterization} décrit nos efforts pour étudier le comportement des applications et de l'architecture sous jacente, et décrit l'orientation des travaux à la suite de nos observations.
Les extensions du langage et du support exécutif sont motivées, décrites et évaluées dans le chapitre~\ref{chap:contrib:openmp}.

Enfin la dernière partie se concentre sur les perspectives : le chapitre~\ref{chap:conclusion} revient sur l'évolution du matériel et du logiciel pendant la thèse, et discute des pistes de recherche envisageables pour pousser plus loin nos idées, avant de conclure notre travail et ce manuscrit.

