\part{Problématiques impliquées et approches existantes}

%\begin{savequote}[6cm]
%<< Some great quote  >>
%\qauthor{François Broquedis}
%\end{savequote}
\chapter{Contexte}\label{chap:contexte}
\chaptertoc
\vfill

\pagebreak

L'objectif de ce chapitre est de donner au lecteur les connaissances de base nécessaires pour l'appréciation du reste de la thèse.
Elles peuvent se classer en trois catégories~: la première concerne les architectures cibles pour nos travaux. Il y a une grande variété de matériels disponibles pour le calcul haute performance : la section~\ref{sec:context:numa} décrit en détails les architectures à mémoire partagée actuelles (NUMA), et la section~\ref{sec:context:os} décrit leur gestion par le système d'exploitation.
La seconde concerne les modèles de programmation à base de tâches. La section~\ref{sec:context:others} décrit les techniques modernes utilisées pour cibler les architectures NUMA, et décrit en détails les concepts de base ainsi que les points clés pertinents à nos travaux.
La section~\ref{sec:context:runtimes} regroupe des informations générales concernant les applications parallèles à base de tâches et le rôle des supports exécutifs dans leur exécution.
Enfin la section~\ref{sec:context:openmp} est dédiée à OpenMP, modèle de programmation qui a été utilisé pour l'application des travaux de la thèse.

\section{Shared memory architectures}\label{sec:context:numa}



\subsection{Single node design}

Talk about cache hierarchy, maybe give some number and a drawing of how it looks like?

\subsection{Interconnecting nodes}

Talk about how nodes are interconnected with each other, give some numbers (?) and some drawing.

\section{Exploitation des architectures NUMA par le système d'exploitation}\label{sec:context:os}

Le support des machines NUMA dans Linux est arrivé dès 2003~\cite{Dobson2003}.
Détailler les dispositifs spécifiques à l'exploitation des machines NUMA au niveau système ne peut pas se faire sans avoir connaissance de certains mécanismes déjà existant.
La section suivante détaille donc la manière dont Linux gère la mémoire d'une machine <<standard>>, puis nous détaillerons les spécificités liées au caractère NUMA de la machine, ainsi que certaines bibliothèques externes utiles pour les programmeurs.

\subsection{Gestion de la mémoire}

La compréhension de la gestion de l'allocation de la mémoire par le système d'exploitation est un point critique lorsque l'on traite les machines NUMA.
Le système d'exploitation gère la mémoire par petit fragments, appelés \emph{pages}.
Lors de l'allocation d'un tableau par exemple, si sa taille dépasse celle d'une \emph{page}, l'espace mémoire alloué consistera donc de plusieurs pages.


Par défaut, lorsque l'allocateur mémoire demande au système la réservation d'un espace mémoire constitué de plusieurs \emph{pages} (lors d'un appel à |malloc| par exemple), le système retarde l'allocation effective de ces pages jusqu'au moment du premier accès - \emph{first-touch}.
C'est ce premier accès qui va déterminer l'allocation physique de la page : elle sera alloué sur le nœud local du processeur effectuant le \emph{first-touch}.

L'allocateur par défaut de la libc fait de la réutilisation de pages. Cela veut dire qu'après plusieurs allocations/libérations successives d'espace mémoire, vous n'obtenez pas forcément des nouvelles pages.
L'objectif derrière cela est de réduire le coût général d'une allocation, puisque l'allocation et la libération physique de nouvelles pages est couteux, et que réutiliser des pages supprime ce coût.
En revanche le corollaire de ce constat est que le contrôle de l'allocation physique par \emph{first-touch} n'est possible que lors de la première allocation des pages.
Ce problème est limité à la durée de vie du programme, s'il est susceptible d'apparaître, il est alors possible d'utiliser des librairies externes pour l'allocation de données, comme décrit dans la section suivante.

Enfin, un dernier détail mérite de l'attention, il s'agit de la taille des pages manipulées par le système.
En effet celle ci est configurable, et deux options sont généralement possibles : soit des pages de 4 Kilo octets, soit des - \emph{huge} - pages de 2 Mega octets.
L'impact de cette taille peut être important dans le cas d'un bloc de données s'étendant sur plusieurs pages : plusieurs requêtes de pages de petites tailles seront plus sensibles à la \emph{latence} de la mémoire, alors qu'une requête d'une page de grande taille équivalente sera plus sensible à la \emph{bande passante}.

\subsection{Prise en compte des architectures NUMA et bibliothèques externes}

Avec l'arrivée du support des machines NUMA dans le système d'exploitation, plusieurs fonctionnalités fondamentales sont apparues et parmi elles on retrouve : la capacité d'obtenir des informations à propos de la machine, telles que le numéro de nœud d'un cœur donné, ou le numéro de nœud d'une adresse mémoire donnée.
On trouve aussi la possibilité de contrôler le placement d'un thread sur un ensemble de cœurs physiques à l'aide d'un masque d'affinité, ce qui permet par exemple de garantir la proximité physique de deux threads vis à vis d'un nœud, ou tout simplement de garantir qu'un thread sera exécuté par un cœur précis.

Plusieurs outils se sont basés sur ces briques de base pour proposer des fonctionnalités plus avancées, parmi eux \emph{numactl} et \emph{hwloc} :

\begin{description}
  \item [numactl~\cite{numactl} :] cet outil s'utilise avant l'utilisation d'un exécutable sur la ligne de commande. Il permet : de contrôler l'ensemble des cœurs (ou nœuds) sur lesquels les threads peuvent être placés au cours de l'exécution du programme ;
    de contrôler l'ensemble des nœuds NUMA sur lesquels peuvent être allouées des données ;
    et encore de modifier la politique d'allocation des pages du système, en permettant par exemple de distribuer les pages successives allouées, de manière cyclique sur un ensemble de nœuds.
  \item [hwloc~\cite{Broquedis2010} :] cet outil permet, entre autres, d'obtenir des informations précises sur le matériel et la topologie de la machine. Il expose les facteurs NUMA théoriques fournies par le matériel, et permet de visualiser les différents nœuds et groupes de nœuds.
  Il propose également des fonctions d'allocations plus fiable que |malloc|, garantissant l'allocation de nouvelles pages, et proposant différentes politiques d'allocations (incluant le \emph{first-touch}).
\end{description}


\bigskip

Dans la section précédente nous avons vu les détails concernant le matériel utilisé au cours de cette thèse, et dans cette section nous avons vu comment l'exploiter au niveau du système d'exploitation.

Le développeur d'application scientifique n'est généralement pas un spécialiste des fonctionnalités de base du système d'exploitation : il va plutôt être un expert de son application et des parties critiques qui la composent.
Il faut donc une couche par dessus le système d'exploitation pour lui permettre d'exprimer de manière plus abstraite son application. Cette couche est le \emph{modèle de programmation}.
Pour cibler les architectures à mémoire partagée, les modèles de programmation à base de tâches offrent des propriétés très intéressantes : ils permettent d'exprimer du parallélisme à grain fin, et l'équilibrage de charge est effectué de manière dynamique par le système exécutif.
Cela permet potentiellement de maximiser l'utilisation du nombre important de ressources disponibles sur une machine NUMA.
L'expression d'un programme à base de tâches permet également d'isoler les parties critiques du programme, ce qui peut faciliter leur analyse.

\section{Modèles de programmation à base de tâches}\label{sec:context:others}

Il existe de nombreux modèles de programmation à base de tâches, certaines fonctionnalités diffèrent, mais les concepts de base restent les même, et seront détaillés ci dessous.

\begin{todo}

  -> parler protocole de cohérence de cache (dans interconnexion peut être)
  -> remonter les sections archi, le cache mérite son propre truc
  -> Parler briévement des modèles de programmation exotiques, en disant qu'ils seront abordé en détail en 3
  -> motiver pourquoi cilk, tbb, et openmp (anotation sur fonction, code utilisateur, simple pragma)
\end{todo}

\subsection{L'unité de base : la tâche}

Une tâche peut être vue comme la plus petite quantité de travail séquentiel exécutable sur un processeur.
En pratique c'est une section de code bien définie du programme, et cela peut être une simple instruction, un bloc de code délimité, ou encore une fonction très complexe.
La quantité de calcul idéale dans une tâche - la \emph{granularité} - peut varier fortement en fonction de l'application et du support exécutif, ce point est abordé en détail dans la section~\ref{sec:context:others:granularity}

Une tâche est nécessairement accompagnée de données qu'elle manipule. De la même manière qu'une fonction utilise des paramètres, le bloc de code composant une tâche utilise des variables qui peuvent être soit locales (on parlera alors de données privées), soit partagées par d'autres parties du code (on parlera de données partagées).


\subsection{Traitement d'une tâche : de la création à l'exécution}\label{sec:context:others:costs}

Si la notion de tâche peut paraître simple, elle s'accompagne d'un certain nombre de traitement plus ou moins automatique (en fonction du modèle de programmation), mais qui dans tous les cas a un coût.
Les trois paragraphes suivants décrivent quelques points clés accompagnant l'utilisation des tâches, qui sont parfois cachés au programmeur.

\subsubsection{Création}

En fonction du modèle de programmation, la création d'une tâche peut être plus ou moins pénible pour le programmeur.

Prenons deux exemples représentatifs pour illustrer.

En StarPU natif, voici par exemple comment on pourrait définir et créer une tâche :

\begin{lstlisting}[language=c++,caption=Exemple simple en StarPU,label=lst:context:simple-starpu,basicstyle=\small]
void work() {
  // calcul
}

int main() {
  // ...
  struct starpu_codelet dummy_big_cl =
  {
    // ...
    .cpu_funcs = { work },
  };

  task = starpu_task_create();
  task->cl = &dummy_big_cl;
  starpu_task_submit(task);
}
\end{lstlisting}

On voit ici que l'utilisateur doit d'une part avoir isolé la partie calcul de son code, et d'autre part interagir avec le support exécutif de manière significative pour créer une tâche.

Si on regarde maintenant un exemple simple en OpenMP :

\begin{lstlisting}
void foo()
{
  // ...
  #pragma omp task
  {
    // calcul
  }
  // ...
}
\end{lstlisting}

Cela parait certes moins intrusif et pénible pour le programmeur, mais pour autant cela ne veut pas dire qu'il y a pas moins d'actions effectuées en pratique !
Que se passe-t-il vraiment ?

Le compilateur ne va pas laisser ces éléments tels quels dans l'objet binaire généré~: il va les transformer en un ensemble de fonctions élémentaires imposées par le support exécutif. Cet ensemble de fonctions élémentaires s'appelle l'\emph{ABI} (pour \emph{Abstract Binary Interface}).

Un exemple de ce type en OpenMP va donc entrainer deux transformations importantes par le compilateur :
\begin{itemize}
  \item l'\emph{outlining} de la fonction, qui consiste à externaliser le code de la tâche et son contexte dans une fonction séparée.
  \item la substitution du pragma par un appel au support exécutif.
\end{itemize}

Cela donnera au final un code binaire généré correspondant à un programme de ce type :

\begin{lstlisting}
void outlined(struct Context c)
{
  // unpacking du contexte, recréation des variables locales
  // calcul
}

void foo()
{
  // ...
  struct Context c;
  // capture des variables partagées
  runtime_specific_create_task(outlined, c);
  // ...
}
\end{lstlisting}

\begin{todo}
  Montrer le packing/unpacking ainsi que les appels au runtime (eg: libomp)
\end{todo}

Au final les actions effectuées pour la création d'une tâche sont équivalentes, bien que parfois cachées.


\subsubsection{Gestion}

Une fois que le programmeur a définie sa tâche et l'a soumise au support exécutif, celui ci doit créer et maintenir une structure de données représentant cette tâche.
Celle ci peut être plus ou moins grande en fonction des informations associées à la tâche.

Le support exécutif va utiliser ces informations au cours de l'exécution du programme pour déterminer quelles tâches sont prêtes pour l'exécution.
En pratique cela signifie que ces structures de données vont être placées dans des conteneurs tels que des files ou des piles, et qu'un certain nombre d'opérations seront effectuées dessus (ajout, suppression, parcours).

En conséquence le coût du maintien des informations à propos d'une tâche entre sa création et son exécution dépend du support exécutif et des structures de données qu'il utilise.


\subsubsection{Exécution}

Cette étape est l'un des points où la gestion par tâche dispose d'un gros avantage par rapport à la création individuelle de thread par le programmeur.
Au début de l'exécution de l'application, un certain nombre de cœurs physiques sont utilisables par le support exécutif (cela peut être déduit implicitement par le support exécutif, ou spécifié explicitement par le programmeur).
Lors de son initialisation, le support exécutif va \textbf{créer et attacher} un thread logique par cœur physique, virtualisant ainsi la gestion des cœurs physiques.

Ces threads vont se voir attribuer différents attributs, comme par exemple une structure de données contenant des tâches.
Ils seront des <<travailleurs>> permanents pour le support exécutif, qui leur donnera des tâches à exécuter au fur et à mesure.

Les threads sont donc les même tout au long de l'exécution de l'application, ce qui évite les coûts liés à la création ou à la destruction de threads. 


\subsection{Moyens de synchronisation}

Lorsqu'on parle de programmation parallèle, il faut bien évidemment parler de synchronisation.
Les différentes tâches définies par l'utilisateur vont être exécutées en parallèle sur la machine, mais dans beaucoup de cas certaines tâches doivent attendre la complétion d'une ou plusieurs tâches avant de pouvoir commencer à être exécutée.

Il y a deux grand types de synchronisations pour la programmation à base de tâche~: la synchronisation explicite, et les dépendances de données.

\subsubsection{Synchronisation explicite}

Le programmeur peut ajouter un point de synchronisation explicite dans le code.
Lorsque le thread exécutant la tâche atteint ce point de synchronisation, il se bloque et attend que l'ensemble des tâches qu'il a créé ait été exécuté avant de reprendre son exécution. Dans l'exemple du listing~\ref{lst:context:task-wait}, exprimé en OpenMP, la tâche C sera garantie d'être exécutée \textbf{après} les tâches A et B.

\begin{lstlisting}[caption=Synchronisation dans le thread courant (OpenMP),label=lst:context:task-wait]
void foo() {
  #pragma omp task
  A();
  #pragma omp task
  B();
  #pragma omp taskwait
  #pragma omp task
  C();
}
\end{lstlisting}


\subsubsection{Dépendances de données}

Le programmeur spécifie des dépendances de données, avec des modes, pour chacune des tâches.
Dans l'exemple du listing~\ref{lst:context:task-dep}, la tâche |write_A| possède une dépendance en \emph{écriture} sur la variable |a|, et la tâche |read_A| possède une dépendance en \emph{lecture}.

Étant donné qu'il y a une lecture et une écriture, le principe de cohérence séquentielle impose que les opérations soit ordonnées dans l'ordre où elles ont été créées (ici la tâche en lecture devrait avoir lieu \textbf{après} la tâche en écriture).

Si on regarde le reste du programme, la tâche |B| dispose d'une dépendance en écriture sur |b| et la tâche |C| souhaite lire |a| et |b|.

Étant donné que du point de vue des dépendances les tâches |write_A| et |B| sont indépendantes, elles pourraient très bien être exécutées en même temps, et |B| pourrait terminer son exécution avant même que |read_A| commence la sienne.

En revanche |C| devra forcément être exécutée après |read_A| et |B| puisqu'elle a une dépendance en lecture sur des données écrites par ces deux tâches, et qu'elle a été créée après dans l'ordre séquentiel.


\begin{lstlisting}[caption=Synchronisation via des dépendances (OpenMP),label=lst:context:task-dep]
void foo() {
  int a;
  int b;
  #pragma omp task depend(out: a)
  write_A(&a);
  #pragma omp task depend(in: a)
  read_A(&a);
  #pragma omp task depend(out: b)
  B(&b);
  #pragma omp task depend(in: a, b)
  C(a, b);
}
\end{lstlisting}

\begin{todo}
GRAPHE : 2.2.2 mini schéma dataflow du graphe correspondant au programme écrit.
\end{todo}

Les deux ont des avantages et des inconvénients : la synchronisation dans le thread courant représente très peu d'overhead lors de l'exécution, mais si le travail est légèrement déséquilibré, certains threads pourraient rester inactifs alors que des tâches pourraient être exécutées.
Les dépendances induisent un coût de calcul des tâches prêtes lors de l'exécution, mais maximise l'utilisation des ressources.


\subsection{Quelques exemples de modèles de programmation}

Plusieurs modèles de programmation populaires proposent d'exprimer du parallélisme à base de tâches, parmi lesquels Cilk, TBB, et OpenMP.

Les listings~\ref{lst:context:cilk},~\ref{lst:context:tbb}, et ~\ref{lst:context:openmp} donnent une comparaison de l'expression du même exemple simple, le calcul du nième nombre de la suite de Fibonacci.

\subsubsection{Cilk}

Cilk~\cite{cilk5} est un modèle de programmation basé sur C.
Il introduit principalement deux nouveaux mots clés : |cilk_spawn| et |cilk_sync|, pour, respectivement, exposer du parallélisme et introduire un point de synchronisation.
Le mot clé |cilk_spawn| vient précéder un appel de fonction pour indiquer que la fonction peut s'exécuter en parallèle. Cela en fait donc un modèle de programmation à base de tâches.
Cilk propose également une extension de la notation de tableau, ayant pour but de faciliter la vectorisation automatique par le compilateur.


\begin{lstlisting}[language=c++,caption=Fibonacci exprimé en Cilk,label=lst:context:cilk,basicstyle=\small]
int fib(int n) {
  if (n < 2)
    return n;
  int x = cilk_spawn fib(n-1);
  int y = fib(n-2);
  cilk_sync;
  return x + y;
}
\end{lstlisting}


\subsubsection{Threading Building Block}

Threading Building Block (TBB)~\cite{Reinders2007} est un modèle de programmation développé par Intel comme une bibliothèque C++.

Elle propose différentes fonctions pour que le programmeur puisse exprimer du parallélisme, dont notamment :
\begin{itemize}
  \item La fonction template |parallel_for|, s'appliquant sur une boucle et prenant en paramètre une fonction utilisateur.
    La bibliothèque découpe automatiquement l'espace d'itération en groupes d'itérations et envoie un itérateur C++ à la fonction utilisateur pour son traitement.
  \item Un ensemble de fonctions pour accéder à l'ordonnanceur de tâches de la bibliothèque.
\end{itemize}

La bibliothèque propose également un ensemble de structures de données à accès concurrent (listes, tables de hachage), ainsi que des allocateurs mémoires.

\begin{lstlisting}[language=c++,caption=Fibonacci exprimé en TBB,label=lst:context:tbb,basicstyle=\small]
#include "tbb/task_group.h"
using namespace tbb;

int Fib(int n) {
  if( n<2 ) {
    return n;
  } else {
    int x, y;
    task_group g;
    g.run([&]{x=Fib(n-1);}); // création d'une tâche
    g.run([&]{y=Fib(n-2);}); // création d'une autre tâche
    g.wait();                // synchronisation
    return x+y;
  }
}
\end{lstlisting}

\subsubsection{OpenMP}

OpenMP~\cite{openmp45} est un modèle de programmation supportant le C/C++ et Fortran.
Il s'utilise à travers des directives de compilation ainsi qu'une API, et propose lui aussi les constructions classiques : boucles, tâche, et autres éléments facilitant la programmation parallèle.

Originellement OpenMP ne proposait que du parallélisme de boucle, le concept de tâches n'a été introduit que plus tard avec OpenMP~3.0, et le concept de dépendances de données entre tâches a été introduit encore plus tard avec la version~4.0.

Le standard d'application de nos idées pour cette thèse étant OpenMP, une description détaillée des fonctionnalités et de ses spécificités est faite dans la section~\ref{sec:context:openmp}.

\begin{lstlisting}[language=c++,caption=Fibonacci exprimé en OpenMP,label=lst:context:openmp,basicstyle=\small]
int fib(int n) {
  if (n < 2)
    return n;
#pragma omp task
  int x = fib(n-1);
  int y = fib(n-2);
#pragma omp taskwait
  return x + y;
}
\end{lstlisting}

\subsection{Quantité de travail et granularité}\label{sec:context:others:granularity}


Dans ce type de modèles de programmation, la clé pour maximiser l'utilisation des ressources est de réduire l'overhead du support exécutif par rapport au calcul en trouvant le bon \emph{grain} de tâche.

Il faut donc jouer sur le degré de parallélisme pour atteindre les meilleures performances : les tâches doivent être suffisamment petites pour proposer le maximum de parallélisme, mais pas trop pour ne pas surcharger le support exécutif, vis à vis des coûts décrits dans la section~\ref{sec:context:others:costs}.

Ce grain optimal dépend de plusieurs facteurs : les structures de données utilisées par le support exécutif, le coût de création des tâches, et la quantité de travail mis à disposition par cœur via ce grain.

Cela peut être illustré via une application telle que la factorisation de Cholesky par bloc : à taille de matrice fixée le nombre de tâches créées dépend directement de la taille de bloc choisie.
Plus la taille de bloc est petite, plus le nombre de blocs créés (et donc le nombre de tâches, et le parallélisme potentiel) est important.

\begin{todo}
%\begin{figure}[ht]
	%\centering
	%\includegraphics[width=0.66\textwidth]{todo}
	%\caption{todo}\label{fig:context:granularity}
%\end{figure}
  courbe grain
\end{todo}

La figure~\ref{fib:context:granularity} illustre l'évolution des performances d'une factorisation de Cholesky d'une matrice de taille 8192, sur un nombre de cœurs fixé (64), en fonction de la taille de bloc et du support exécutif.
Comme on peut le voir, les parties extrèmes de la courbe se comportent de manière similaires quelque soit le support exécutif : une taille de bloc trop faible génère beaucoup trop de tâches et les support exécutifs sont complètement surchargés. Une taille de bloc trop importante limite complètement le parallélisme et donc les performances.

Le grain adapté n'est pas nécessairement unique : en fonction du support exécutif on peut avoir un choix plus ou moins important (TODO : mettre les vrais chiffres d'ompss et de libkomp/gcc).

La courbe tracée avec libkomp inclue nos travaux sur l'affinité, et cela permet d'illustrer que même si cela permet d'influer sur les performances maximales, cela a un impact minimal sur le grain.



Le choix du grain pour une tâche dépend entièrement de l'application, et reste à l'appréciation du programmeur.

\subsection*{Conclusion}

Bien que tout ces modèles de programmation aient leur spécificités, ils permettent tous de décrire l'application sous forme de graphe de tâches direct et acyclique (DAG).

L'étape suivante consiste à exécuter ce graphe sur la machine, et pour cela le support exécutif peut se reposer sur un ensemble important de techniques d'ordonnancement. 



\section{Supports exécutifs}\label{sec:rw:other-runtimes}

Il existe un certain nombres d'autres supports exécutifs, pour OpenMP comme d'autres modèle de programmation.
Les sections ci après introduisent ceux ayant des thématiques très proches de cette thèse.

\subsection{XKaapi}

XKaapi~\cite{Gautier2007} est un support exécutif, à base de tâche avec dépendances, ciblant les architectures multicœurs et hétérogènes.
Il repose sur hwloc pour découvrir la topologie de la machine, et utilise ces informations à de multiples endroits.

XKaapi dispose d'un nombre important de fonctionnalités spécifiques à l'ordonnancement de tâches.
Une attention particulière a été portée au coût de création des tâches au sein du support exécutif, qui a été diminué au maximum (TODO : ref needed).
Le moteur d'ordonnancement de XKaapi fonctionne par vol de travail, et implémente les étapes critiques de \emph{sélection} et de \emph{placement} décrites dans la section~\ref{sec:context:runtimes:ws}. Il est facile d'ajouter des heuristiques additionnelles pour ces deux étapes, ce qui nous a permis d'implémenter dans ce support exécutifs les extensions décrites dans la section~\ref{sec:contrib:ws:heuristics}.

Le nombre de files de tâches repose sur les informations fournies par hwloc~: XKaapi implémente une file de tâches par niveau de la hiérarchie (i.e.~: une file par cœur, une file par nœud NUMA, etc...), qui sont éventuellement utilisées par les heuristiques.

Pour la gestion de ces files, XKaapi implémente le protocole THE~\cite{cilk5} proposé par Cilk. Ce protocole permet de faire de manière non bloquante des accès concurrent à la même file de tâche. Le principe est le suivant~: le \emph{voleur} - distant - va venir prendre des tâches en tête de file, et la \emph{victime} (ou le thread local) va venir ajouter ou retirer des tâches en queue de file.
Le seul conflit se produit lorsque la file n'a qu'un seul élément, et il peut être résolu par un simple \emph{compare-and-swap}, se traduisant par un échec de la requête pour l'un, et un succès pour l'autre.

En plus de l'utilisation de ce protocole, XKaapi peut effectuer de l'agrégation de requêtes de vols~: lorsque plusieurs voleur font effectuer des requêtes sur la même victime, seul le premier voleur arrivé va effectuer la requête de vol, et récupérer suffisamment de tâche pour l'ensemble des voleurs.
Les gains théoriques liés à ce mécanismes ont été étudiés par Tchiboukdjian et al.~\cite{Tchiboukdjian2010a}.

Pour observer le comportement des applications exécutées, il dispose d'un outil de génération de traces.
Cela permet une analyse pointue du comportement de l'application, à travers les compteurs de performances matériels et une analyse par type de tâche.
Cet outil nous a permis de faire des observations préliminaires déjà très poussées, sur une étude de cas abordée dans la section~\ref{sec:contribs:apps:cholesky:observations}.

XKaapi est principalement utilisé comme prototype de recherche, et a été utilisé pour l'implémentation de certains travaux proches des thématiques de cette thèse, en particulier celle de la localité des données~\cite{Durand2013, Bleuse2014, Lima2015}.

Enfin il dispose d'une couche de compatibilité pour OpenMP, nommée libKOMP~\cite{Broquedis2012}.
Cette couche implémente à la fois les ABIs de libGOMP et libOMP, ce qui permet de l'utiliser pour exécuter des programmes OpenMP~4.5 directement en compilant via GCC ou Clang, et en changeant le support exécutif chargé à l'exécution.



\subsection{libGOMP}

libGOMP~\cite{Novillo2006} est le support exécutif OpenMP fourni avec le compilateur GCC.

Au niveau des fonctionnalités, il implémente la totalité du standard OpenMP~4.5.
Comme la majorité des supports exécutifs, libGOMP réutilise les threads qui sont créés entre différentes région parallèles successives, pour éviter d'avoir à payer le coût de destruction/création d'un thread inutilement.
Les gestion des constructions à base de boucles et de tâches sont complètement séparées dans le support exécutif.
Vis à vis de la hiérarchie de l'architecture cible, il n'y a aucune disposition particulière pour essayer de la prendre en compte.

Pour la gestion des tâches, il a des différences majeures dans la manière de fonctionner par rapport à XKaapi~: il fonctionne bien par vol de travail, mais en revanche il n'y a qu'une seule file de tâches par \emph{team}, et donc une seule file pour l'ensemble des threads !
Si fonctionnellemment cette caractéristiques n'est pas un problème, cela peut avoir un impact sur les performances compte tenu du fait que tous les threads devront se synchroniser pour accéder à la même struture de données.
Cela se voit d'ailleurs sur la figure~\ref{fig:context:granularity} illustrant l'impact de la granularité des tâches~: pour des petites tailles de bloc (et donc un grand nombre de tâches), libGOMP est loin derrière à cause du surcout entrainé par la gestion de la liste de tâches.
(cf figure 2.4 sur la granularité)

Néanmoins, en tant que support exécutif grand public et largement utilisé, il constitue une référence intéressante.

\subsection{libOMP}

libOMP est le support exécutif OpenMP fourni avec le compilateur Clang, directement basé sur le support exécutif d'Intel fourni avec ICC.
Ils partagent donc exactement les même caractéristiques.

Compte tenu du fait qu'il a été développé à la base par des développeurs d'Intel, une partie de ses fonctionnalités ont été motivées par l'exploitation du matériel produit par Intel comme le Xeon Phi.

De manière similaire à libGOMP, la gestion des boucles et des tâches est séparée, et les threads (et même les \emph{teams} et leurs structures de données associées) sont réutilisés par les régions parallèles successives.

En revanche libOMP se distingue de libGOMP de part ses structures de données~: chaque thread d'une \emph{team} possède une file de tâche propre.
Il conserve donc un fonctionnement très proche de XKaapi, dans le sens où il passe par des fonctions de \emph{sélection} et \emph{placement} lors du vol de travail.
Les heuristiques de base pour ces fonctions sont les suivantes~: la sélection a lieu aléatoirement parmi les file de tâches disponibles~; lors de vols successifs, le voleur essaye en priorité la dernière file dans laquelle il a réussi à voler une tâche. Le placement a lieu dans la file du thread courant.

Bien que ce mécanisme n'ait pas été initialement conçu pour permettre d'interchanger des stratégies, cela proposait une base suffisamment solide pour accueillir les extensions que nous proposons dans le chapitre~\ref{chap:contrib:openmp}.
Les modifications que nous avons apporté à ce support exécutif sont détaillées dans la section~\ref{sec:contribs:perf_eval:libkomp}.


\subsection{OmpSs}\label{subsec:rw:ompss}

OmpSs~\cite{OMPSs} est un modèle de programmation visant à étendre OpenMP, en particulier le support du parallélisme asynchrone (à base de tâches avec dépendances par exemple), et de l'hétérogénéité.
La syntaxe et les détails dans l'utilisation peuvent être légèrement différents, mais les constructions et concepts restent les même.
OmpSs est composé d'un compilateur, \emph{Mercurium}, et d'un support exécutif \emph{Nanos++}.

Du point de vue de la gestion des tâches, Nanos fonctionne également par vol de travail.
Par défaut l'ordonnanceur fonctionne à l'aide d'une unique file de tâches à priorité, néanmoins l'interface de base d'un ordonnanceur doit fournir les fonctions |getReadyTask| et |addReadyTask|, qui sont équivalente aux fonctions de sélection et placement déjà évoquées.
Il ne dispose pas d'ordonnanceur prenant en compte la localité des données, mais certains d'entre eux disposent de file de tâches associées à certains éléments de la hiérarchie (cœur ou nœud NUMA).


\subsection{OpenStream}

OpenStream~\cite{Pop2013} est un modèle de programmation par flots de données dérivant directement d'OpenMP~3.0.
Le programmeur défini des flots de données ainsi que des tâches opérant en lecture et/ou écriture sur une certaine quantité de données d'un flot (appelée \emph{window}).
Concrètement les flots de données peuvent être vus comme des tableaux, et les tâches opèrent sur un certain nombres d'éléments contiguës de celui ci.
Le support exécutif étudie ensuite l'ordre d'écriture dans les différentes parties d'un flot pour construire un graphe de dépendances des tâches, qui sera ensuite ordonnancé sur la machine.

Ce modèle se rapproche donc très fortement des tâches avec dépendances qui sont apparues dans la version suivante d'OpenMP.
OpenStream utilise un support exécutif avec des extensions pour les architectures NUMA, nous revenons dessus en détail dans la section~\ref{sec:rw:numa:thread-data}.

\subsection{StarPU}

StarPU~\cite{StarPU} est une librairie de programmation parallèle à base de tâche avec dépendances.
Son support exécutif hétérogène permet de cibler aussi bien des processeurs standards que des accélérateurs, à partir du moment où le programmeur a fourni différentes versions des tâches pour les différentes architectures cibles.

StarPU utilise des techniques avancées d'ordonnancement sur ressources hétérogènes, et propose différentes techniques d'ordonnancement en fonction du but recherché.
Point de vue performances, les ordonnancements de tâches disponibles peuvent être soit purement \emph{online} (tel que le vol de travail - \emph{ws}), ou dériver de techniques initialement \emph{offline} comme leurs ordonnanceurs \emph{dm}, où un ordonnancement initial similaire à HEFT est effectué.

\subsection{QUARK}

QUARK~\cite{Kurzak2013} (QUeing And Runtime for Kernels) est le support exécutif privilégié pour la bibliothèque d'algèbre linéaire PLASMA, dont certaines de nos applications sont adaptées.

Il fonctionne lui aussi à base de tâches, qui sont exclusivement des fonctions de l'utilisateur.
La création de tâches se fait à l'aide d'appels au support exécutif, et en plus d'un pointeur sur la fonction tâche le programmeur indique les variables manipulées et le type d'accès effectué.

Cela permet donc à QUARK de déterminer un ordre d'exécution sur les tâches pour son ordonnancement.
L'avantage principal de QUARK par rapport aux autres modèle de programmation similaires est qu'il propose des extensions spécifiques à certains algorithmes d'algèbre linéaire présent dans PLASMA.



\begin{savequote}[6cm]
<< truc
\qauthor{Test}
\end{savequote}

\chapter{Extending OpenMP}\label{chap:contrib:TODO}
\chaptertoc


%\input{tex/OpenMP on numa}
% Big section to describe how we use/extend OpenMP to exploit these architectures

% - Description of OpenMP language (tasking and such)
% - Description of OpenMP runtime

% - Motivating examples for our works using current OpenMP

% - Missing to improve things
%    - What we implemented
\cite{Virouleau2016}, Description, implementation and evaluation of an affinity clause for task directives

\cite{Virouleau2016b}, Using data dependencies to improve task-based scheduling strategies on numa architectures
%    - What remains


\section{OpenMP vs NUMA architectures}

\subsection{Current construct and clauses}
\subsection{Additional tools and techniques}
\subsection{Numbers, and why we need more}
\subsection{Conclusion: what matters?}

\section{Matching what is missing}
\subsection{Programmers needs}
\subsection{Expressing task-to-resource affinity}

\section{topo description des trucs précédents + qu'est ce qu'on peut utiliser dans le runtime}
\subsection{The programmer can give more information}
\subsection{Can the runtime exploit it?}

\section{Enhancing runtimes techniques}
\subsection{Common schedulers and their specificities}
\subsection{Data-based Heuristics}
\subsection{Concrete implementation}

\section{On top of Intel's runtime}
\subsection{TODO?}

\section{What is missing, can we do more ?}
More flexibility for initialization

\begin{savequote}[6cm]
<< truc
\qauthor{Test}
\end{savequote}


\chapter{Évaluation de performances}\label{chap:exp:env}
\chaptertoc

\section{Compiler and runtimes}

\subsection{Gcc/GOMP}
\subsection{Clang/Intel}
\subsection{LibKOMP}
\subsection{OMPSs}
\subsection{KSTAR/StarPU?}
\subsection{Openstream?}

\section{Additional software}
\subsection{Basic benchs (mem bandwidth, epcc)}
\subsection{Various BLAS libraries}


%\input{tex/exp}
% This chapter should include:
% - Hardware description
% - Software description
%   - In-depth analysis of linar algebra kernels?
% - Experimentation results with all the above

\cite{Duran2009}, Barcelona OpenMP Tasks Suite: A set of benchmarks targeting the exploitation of task parallelism in OpenMP

\cite{Virouleau2014}, Evaluation of OpenMP dependent tasks with the KASTORS benchmark suite
% - Perf/energy


\bigskip

Cette section a décrit l'ensemble des connaissances et concepts de base nécessaires à la pleine compréhension de ce manuscrit.
Cela concerne à la fois : le matériel, les architectures des processeurs et leur association pour former une machine NUMA ; le support du matériel au niveau du système d'exploitation ; les concepts exprimés dans les modèles de programmation, tels qu'OpenMP ; et enfin les détails d'implémentation des supports exécutifs.

Le chapitre suivant rentre beaucoup plus en détails sur les travaux de l'état de l'art en relation avec nos contributions, qui se basent sur les connaissances de ce chapitre.
