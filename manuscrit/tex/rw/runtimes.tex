\section{Compilateurs populaires et leurs supports exécutif}\label{sec:rw:compilers}


Avant de rentrer dans le vif du sujet, il nous paraît important d'insister sur la distinction entre compilateur et support exécutif.

Tous les modèles de programmation ne sont pas implémenté par tous les compilateurs, mais par contre le fonctionnement général reste le même.
Le code source va contenir des directives (|#pragma|) ou des appels de fonctions décrits par le modèle de programmation.

Le compilateur ne va pas laisser ces éléments tels quels dans l'objet binaire générer : il va les transformer en un ensemble de fonctions élémentaires imposées par le support exécutif. Cet ensemble de fonctions élementaires s'appelle l'\emph{ABI} (pour \emph{Abstract Binary Interface}).

Pour donner un exemple concret, un code de ce type en OpenMP :
\begin{lstlisting}
void foo()
{
  // ...
  #pragma omp task
  {
    // calcul
  }
  // ...
}
\end{lstlisting}

Entrainera deux transformations importantes :
\begin{itemize}
  \item l'\emph{outlining} de la fonction, qui consiste à externaliser le code de la tâche et son contexte dans une fonction séparée.
  \item la substitution du pragma par un appel au support exécutif.
\end{itemize}

Cela donnera au final un code binaire généré correspondant à un programme de ce type :

\begin{lstlisting}
void outlined(struct Context c)
{
  // unpacking du contexte
  // calcul
}

void foo()
{
  // ...
  struct Context c;
  // capture des variables locales et partagées
  runtime_specific_create_task(outlined, c);
  // ...
}
\end{lstlisting}

La conséquence directe de cette transformation est qu'il existe un \textbf{couplage fort} entre le compilateur et le support exécutif.

Dans le cas spécifique d'OpenMP il y a principalement 3 compilateurs qui implémentent complètement (et suivent rapidement) le standard : GCC, ICC, et Clang.


\subsubsection{GCC}

Le compilateur GCC est livré avec son support exécutif par défaut : libGOMP.
Les développeurs du compilateur sont généralement très réactifs aux évolutions du standard, et GCC supporte OpenMP~4.0 depuis la version 4.9.

libGOMP ne dispose pas de stratégie d'ordonnancement particulièrement dédiée aux architectures NUMA.
L'ordonnancement des tâches est gérée par la technique du vol de travail, avec une queue de tâche unique pour l'ensemble des threads.



\subsubsection{ICC}


ICC est livré avec son support exécutif - open source~\footnote{https://www.openmprtl.org/} - libIOMP.
Il supporte la version~4.0 d'OpenMP depuis la version 15.0.

ICC dispose d'un support privilégié pour accélérateur (construction |target|) ciblant les Xeon Phi.


\subsubsection{Clang/LLVM}

Le support d'OpenMP dans Clang a été ajouté d'un coup, contrairement à GCC où il a été progressivement enrichi.
Intel a tout d'abord ajouté le support dans un clone de Clang : \emph{clang-omp}~\footnote{https://clang-omp.github.io/}.
Il y a ensuite eu un effort d'ingénierie pour l'inclure dans Clang, et la license du support exécutif d'Intel a été changé pour être compatible avec l'infrastructure LLVM, ce qui a permis de l'embarquer directement dans le code source, en le renommant libOMP.

OpenMP~4.0 est supporté depuis la version 3.8 de Clang (sauf pour la partie |target|).

\subsubsection{Compatibilité}

Dans un scenario idéal, on pourrait à loisir interchanger les compilateurs et supports exécutifs utilisés : par exemple compiler du code avec GCC, et l'exécuter en utilisant le support exécutif d'Intel.

Clang et ICC compilant pour le même support exécutif, eux deux sont effectivement interchangeables.

En revanche libGOMP et libIOMP/libOMP utilisent des ABIs différentes, les codes générés par GCC et Clang/ICC sont donc a priori incompatibles.
Heureusement les développeurs de libOMP ont implémentés une couche d'interconnexion entre les deux ABIs, ce qui permet effectivement d'interchanger à loisir ces 3 compilateurs populaires et leurs support exécutifs.

\section{Autres supports exécutifs}\label{sec:rw:other-runtimes}

Il existe un certain nombres d'autres supports exécutifs, pour OpenMP comme d'autres modèle de programmation.
Les sections ci après décrivent ceux ayant des thématiques très proches de cette thèse.

\subsection{Kaapi - libKOMP}

Kaapi~\cite{Gautier2007} est un support exécutif à base de tâche avec dépendances.

Il dispose d'une couche - libKOMP~\cite{Broquedis2012} - implémentant à la fois les ABIs de libGOMP et libOMP, ce qui permet de l'utiliser directement en compilant via GCC ou Clang, et en changeant le support exécutif chargé à l'exécution.

L'ordonnancement des tâches est fait par vol de travail, et le support exécutif propose une vision hiérarchique de la machine, ce qui offre une base pour implémenter des stratégies d'ordonnancement plus complexe (TODO: reformuler).


\subsection{OmpSs}\label{subsec:rw:ompss}

OmpSs~\cite{OMPSs} est un modèle de programmation visant à étendre OpenMP, visant à étendre le support du parallélisme asynchrone (à base de tâches avec dépendances par exemple), et de l'hétérogénéité.
La syntaxe est les détails dans l'utilisation peuvent être légèrement différents, mais les constructions et concepts restent les même.

OmpSs est composé d'un compilateur, \emph{Mercurium}, et d'un support exécutif \emph{Nanos++}.

\subsection{StarPU}

Support exécutif à base de tâches, permettant de cibler les architectures à mémoire partagée ainsi que des accélérateurs : support exécutif hétérogène.

\cite{StarPU}, StarPU: A Unified Platform for Task Scheduling on Heterogeneous Multicore Architectures

Accessible pour OpenMP via le compilateur Kstar.

Others

\subsection{Quark}

\cite{Kurzak2013}, Multithreading in the PLASMA Library
TODO : se documenter

\subsection{OpenStream}

Cite Andi

Modèle de programmation par flôt de données.
Fonctionnalités NUMA.


Blabla
