\section{Caractérisation et benchmarking}\label{sec:rw:benchmarking}

\begin{todo}
  En fait les benchmarks important que je comptais aborder ici (principalement BOTS et PLASMA, sur lesquels les kastors sont basés) sont inclus dans la motivation pour 5.1...
\end{todo}

%Nous décrivons ci-après quelques benchmarks populaires classés selon leur cible : soit l'évaluation de parties très spécifiques du langage OpenMP, soit l'évaluation plus générale de \emph{classes} d'application, comme l'agèbre linéaire ou les stencils.

%\subsection{Évaluation des contructions d'OpenMP}

%epcc, nas, specomp

%%Avant cela d'autres benchmarks tels que PARSEC~\cite{Bienia2011}, ou , ou bien encore Rodinia~\cite{Rodinia2010} ciblaient certaines constructions d'OpenMP, mais aucun n'a été adapté pour tirer profit du parallélisme de tâches.


%\subsection{Évaluation générale d'application}\label{sec:rw:benchmarking}
%\subsection{BOTS}
%SPECOMP~\cite{Muller2012}
%La Barcelona OpenMP Tasks Suite~\cite{Duran2009} est un ensemble de benchmarks ayant pour but d'évaluer le parallélisme de tâche dans OpenMP.


%\subsection{Plasma}

