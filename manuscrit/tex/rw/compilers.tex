\section{Compilateurs et interopérabilité}\label{sec:rw:compilers}


Avant de rentrer dans le vif du sujet, il nous paraît important d'insister sur la distinction entre compilateur et support exécutif.

Tous les modèles de programmation ne sont pas implémentés par tous les compilateurs, mais par contre le fonctionnement général reste le même.
Le code source va contenir des directives (|#pragma|) ou des appels de fonctions décrits par le modèle de programmation. Le code source va ensuite être plus ou moins transformé par le compilateur.

La section~\ref{sec:context:others:costs} décrit en détail la transformation de code s'opérant dans le cas d'OpenMP.
La conséquence directe de cette transformation est qu'il existe un \textbf{couplage fort} entre le compilateur et le support exécutif.

Dans le cas spécifique d'OpenMP il y a principalement 3 compilateurs qui implémentent (et suivent rapidement) le standard : GCC, ICC, et Clang.

\subsection{Un point sur l'état des compilateurs}

TODO : décrire le compilo comme un point d'entrée sur une ABI, derrière on branche un runtime

\paragraph{GCC}

Le compilateur GCC est livré avec son support exécutif par défaut : libGOMP.
Les développeurs du compilateur sont généralement très réactifs aux évolutions du standard, et GCC supporte OpenMP~4.0 depuis la version 4.9.

libGOMP ne dispose pas de stratégie d'ordonnancement particulièrement dédiée aux architectures NUMA.
L'ordonnancement des tâches est gérée par la technique du vol de travail, avec une queue de tâche unique pour l'ensemble des threads.



\paragraph{ICC}


ICC est livré avec son support exécutif - open source~\footnote{https://www.openmprtl.org/} - libIOMP.
Il supporte la version~4.0 d'OpenMP depuis la version 15.0.

ICC dispose d'un support privilégié pour accélérateur (construction |target|) ciblant les Xeon Phi.


\paragraph{Clang/LLVM}

Le support d'OpenMP dans Clang a été ajouté d'un coup, contrairement à GCC où il a été progressivement enrichi.
Intel a tout d'abord ajouté le support dans un clone de Clang : \emph{clang-omp}~\footnote{https://clang-omp.github.io/}.
Il y a ensuite eu un effort d'ingénierie pour l'inclure dans Clang, et la license du support exécutif d'Intel a été changé pour être compatible avec l'infrastructure LLVM, ce qui a permis de l'embarquer directement dans le code source, en le renommant libOMP.

OpenMP~4.0 est supporté depuis la version 3.8 de Clang (sauf pour la partie |target|).

\subsection{Compatibilité}

Dans un scenario idéal, on pourrait à loisir interchanger les compilateurs et supports exécutifs utilisés : par exemple compiler du code avec GCC, et l'exécuter en utilisant le support exécutif d'Intel.

Clang et ICC compilant pour le même support exécutif, eux deux sont effectivement interchangeables.

En revanche libGOMP et libIOMP/libOMP utilisent des ABIs différentes, les codes générés par GCC et Clang/ICC sont donc a priori incompatibles.
Heureusement les développeurs de libOMP ont implémentés une couche d'interconnexion entre les deux ABIs, ce qui permet effectivement d'interchanger à loisir ces 3 compilateurs populaires et leurs support exécutifs.
