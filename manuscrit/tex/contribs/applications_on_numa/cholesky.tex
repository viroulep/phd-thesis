\section{Une étude de cas : Cholesky}


\subsection{Étude de cas : Cholesky}
\label{chap:contribs:apps:cholesky}

TODO : décrire en détail l'algorithme de Cholesky, et faire un point sur les différentes performances que l'on obtient en fonction de comment on fait varier l'initialisation de données.

GRAPHE : 4.2.4 schéma du DAG de Cholesky, un joli graph en couleur avec une couleur par noyau. point bonus pour un schéma explicatif du fait que dpotrf est la clé pour générer le parallélisme.

GRAPHE : 4.2.4 performances détaillées Cholesky : impact de la taille de la matrice, de la taille de bloc, de la machine. tailles = 8k, 16k, 32k, bs = 128/256/512. (*sans* affinité, juste un topo)

