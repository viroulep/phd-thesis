\section{Exécution précise de noyaux}\label{sec:contribs:outil}

\subsection{Motivation}

Les résultats de la section précédente montrent que l'on peut globalement évaluer le comportement d'une application.
La suite naturelle est l'observation des différents facteurs pour essayer de déterminer s'il y a des améliorations possibles pour ce comportement, et comment faire pour l'améliorer.
Au final une application basée sur des tâches avec dépendances se ramène à ordonnancer un graphe de tâches sur un ensemble de ressources.
Sachant que dans le contexte d'une machine NUMA le temps d'exécution d'une tâche dépend, entre autre, de la ressource sur laquelle elle est placée, on a envie de pouvoir modéliser le comportement individuel de chaque type de tâche en fonction de son placement et du placement des données.
Une fois cela fait, cela permettra d'identifier des potentielles variations de comportement, et ajuster les heuristiques d'ordonnancement pour prendre en compte ces variations.

C'est à ce besoin que répond NOMDEL4OUTIL : permettre à l'utilisateur de définir ce qu'il exécute et où, et garantir cette exécution, avec un certain nombre de variables observables.

\subsection{Description d'un scenario}

Un scenario est défini de la manière suivante :
 - Un ensemble de données et variables
 - Une liste d'actions à effectuer
 - Un ensemble de caractéristiques à observer

\subsubsection{Données et variables}
\subsubsection{Actions}
\subsubsection{Observeurs}


\subsection{Design de l'outil}

Afin de garantir un minimum de "bruit" lors des expériences, il fallait que l'archicture de l'outil soit simple, avec peu de logique quand au contrôle de l'exécution des tâches.

Le flot d'exécution est le suivant :
 - L'utilisateur fourni un scenario
 - L'ensemble des coeurs utilisés est déterminé à partir du scénario
 - Pour chaque coeur, un thread est créé et attaché à ce coeur, et une queue (FIFO) de tâches est créée
 - L'ensemble des tâches définies dans le scenario est poussée dans les queues dans l'ordre défini
 - Le scenario est déroulé
 - Un résumé des caractéristiques observées est affiché

\subsection{Exemples de noyaux}


\subsection{Quelles caractéristiques observer ?}
Perf ctr, perfs tout court

\subsection{Application aux noyaux de Cholesky}
\subsection{Impact de la bibliothèque BLAS}

matter for top perf, not for overall behavior

\subsection{Conclusion : modélisation de base possible}

