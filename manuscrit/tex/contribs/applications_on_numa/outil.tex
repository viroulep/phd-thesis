\section{Exécution précise de noyaux}\label{sec:contribs:outil}

\subsection{Motivation}

Les résultats de la section précédente montrent que l'on peut globalement évaluer le comportement d'une application.
La suite naturelle est l'observation des différents facteurs pour essayer de déterminer s'il y a des améliorations possibles pour ce comportement, et comment faire pour l'améliorer.
Au final une application basée sur des tâches avec dépendances se ramène à ordonnancer un graphe de tâches sur un ensemble de ressources.
Sachant que dans le contexte d'une machine NUMA le temps d'exécution d'une tâche dépend, entre autre, de la ressource sur laquelle elle est placée, on a envie de pouvoir modéliser le comportement individuel de chaque type de tâche en fonction de son placement et du placement des données.
Une fois cela fait, cela permettra d'identifier des potentielles variations de comportement, et ajuster les heuristiques d'ordonnancement pour prendre en compte ces variations.

C'est à ce besoin que répond NOMDEL4OUTIL : permettre à l'utilisateur de définir ce qu'il exécute et où, et garantir cette exécution, avec un certain nombre de variables observables.

\subsection{Description d'un scenario}

Un scenario est défini de la manière suivante :
\begin{itemize}
 \item Un ensemble de données et variables
 \item Une liste d'actions à effectuer
 \item Un ensemble de caractéristiques à observer
\end{itemize}

Cette définition très générique permet beaucoup de flexibilité, et les sections suivantes précisent les différentes caractéristiques de ces points, ainsi que des exemples concrêt d'utilisation.

Il est important que le format de description d'un scenario soit humainement lisible, et ne conduise pas à une recompilation systématique du programme. C'est donc une description en YAML qui a été choisie.

\subsubsection{Données et variables}

Elles sont indispensables car c'est la dessus que vont se baser les actions du scenario.

L'utilisateur doit fournir les noms et type des variables utilisées en paramètres des différents noyaux, elles peuvent être réutilisées par différents noyaux, mais dans tous les cas l'utilisateur est le seul responsable de leur gestion.

En pratique ces variables peuvent soit être des constantes, ou bien être initialisées par une action dans le scénario.

Si on imagine l'exemple d'un scenario simple ou l'on souhaite exécuter une multiplication de matrices carrées - |dgemm| - sur un coeur donné, nous avons besoin de trois matrices |a|, |b|, et |c|, ainsi que d'une taille de bloc, |block_size|.

Voici concrêtement à quoi ressemblerait la déclaration de ces données :

\begin{lstlisting}[language=yaml]
- a:
  - type: "double *"
- b:
  - type: "double *"
- c:
  - type: "double *"
- block_size:
  - type: "int"
  - value: 256
\end{lstlisting}

\subsubsection{Actions}

C'est là où on décrit effectivement les noyaux exécutés au cours du scénario.
L'utilisateur indique une série d'actions à exécuter, et avec quels paramètres.

Chaque action peut avoir les caractéristiques suivantes :
\begin{itemize}
  \item core: nombre entier indiquant le coeur sur lequel exécuter l'action.
  \item kernel: chaine de caractères avec le nom de l'action, correspondant à un noyau connu du programme.
  \item params: liste de variables à passer à l'action, leur nom doit correspondre à des données déclarées dans la section précédente.
  \item repeat: nombre entier indiquant le nombre de fois que cette action doit être répétée.
  \item sync: booléen indiquant s'il faut synchroniser le démarrage de cette action avec celui des autres actions présentes dans les file d'attente des autres coeurs.
\end{itemize}

Si on continue dans l'exemple simple d'une multiplication de matrices carrées, il faut que l'on définisse les actions suivantes : l'initialisation de chaque matrice, le lancement du dgemm une fois que ces matrices sont initialisées.
Afin d'avoir une mesure plus précise du comportement du noyau, on peut indiquer une répétition du noyau, ici on choisi 50 pour l'exemple.

Voici en pratique à quoi ressemblerait un tel scenario :
\begin{lstlisting}[language=yaml]
  actions: 
  - kernel: init_blas_bloc
    sync: false
    params: 
    - a
    - block_size
    core: 0
    # Pour l'initialisation, un seul appel suffit
    repeat: 1
  - kernel: init_blas_bloc
    sync: false
    params: 
    - b
    - block_size
    core: 0
    repeat: 1
  - kernel: init_blas_bloc
    sync: false
    params: 
    - c
    - block_size
    core: 0
    repeat: 1
  - kernel: dgemm
    sync: false
    params: 
    - a
    - b
    - c
    - block_size
    core: 0
    # Ici c'est le noyau de calcul, on veut donc le repeter
    repeat: 50
\end{lstlisting}

Simplement exécuter ces actions ne nous donnera pas grand chose, il faut donc définir un ensemble de paramètres à observer.

\subsubsection{Observateurs}

L'outil propose un certain nombre de caractéristiques observables :

\begin{itemize}
  \item le temps passé dans l'action (en millisecondes).
  \item pour les noyaux d'algèbre linéaire, la performance de l'action en Gflops.
  \item des compteurs de performances à travers PAPI
\end{itemize}

En continuant sur notre exemple, voici à quoi ressemblerait la section du scénario si nous souhaitions observer la performance de la multiplication de matrice en Gflops, le nombre de cycles passé dans l'action, ainsi que le nombre de cache miss au niveau 3 généré par l'action.

\begin{lstlisting}[language=yaml]
flops_dgemm:
# le nombre de flops depend de la taille de bloc, il faut donc la donner en parametre
- block_size
papi:
- PAPI_TOT_CYC
- PAPI_L3_TCM
\end{lstlisting}

\subsection{Design de l'outil}

Afin de garantir un minimum de "bruit" lors des expériences, il fallait que l'archicture de l'outil soit simple, avec peu de logique relative au contrôle de l'exécution des tâches.

Le flot d'exécution est le suivant :
\begin{itemize}
 \item L'utilisateur fourni un scenario
 \item L'ensemble des coeurs utilisés est déterminé à partir du scénario
 \item Pour chaque coeur, un thread est créé et attaché à ce coeur, et une queue (FIFO) de tâches est créée
 \item L'ensemble des tâches définies dans le scenario est poussée dans les queues dans l'ordre défini
 \item Le scenario est déroulé
 \item Un résumé des caractéristiques observées est affiché
\end{itemize}

\subsection{Exemples de noyaux}

Noyaux de cholesky, dgemm à la main, mem bandwidth

\subsection{Quelles caractéristiques observer ?}
Perf ctr, perfs tout court

\subsection{Application aux noyaux de Cholesky}
\subsection{Impact de la bibliothèque BLAS}

matter for top perf, not for overall behavior

\subsection{Conclusion : modélisation de base possible}

