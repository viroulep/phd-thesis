\section{Exécution précise de noyaux}\label{sec:contribs:outil}

\subsection{Besoins pour un outil spécifique}

Les résultats de la section précédente montrent que l'on peut tout à fait étudier le comportement global d'une application sur une machine. On peut évidemment observer aussi les variations de ce comportement global lorsqu'on change certains détails, via l'utilisation par exemple de \emph{numactl}.

En revanche, si l'on connait bien son application, on a envie de pouvoir étudier le comportement précis de certaines parties critiques de l'application afin de pouvoir identifier ce qui cause le comportement global.

La suite naturelle de cette identification est de déterminer s'il y a des améliorations possibles pour ce comportement local, et comment l'améliorer en pratique.

Dans le cas d'une application à base de flots de données, chaque partie de l'application est bien identifiée, et correspond à un noeud dans le graphe de tâches.
Toutes les données manipulées par une partie de l'application sont facilement identifiées également, puisqu'il s'agit des connexions entre les noeuds du graphe de tâches.

Dans le contexte d'une machine NUMA, le temps d'exécution d'une tâche dépend à la fois du placement de son exécution, ainsi que du placement de ces données. On a donc envie de pouvoir modéliser le comportement individuel de chaque type de tâche en fonction de son placement et du placement des données.

Une fois cela fait, cela permettra d'identifier des potentielles variations de comportement, et ajuster les heuristiques d'ordonnancement pour prendre en compte ces variations.

C'est à ce besoin que répond NOMDEL4OUTIL : permettre à l'utilisateur de définir ce qu'il exécute et où, et garantir cette exécution, avec un certain nombre de variables observables.

Dans la suite de ce chapitre, on appellera l'ensemble des paramètres décrivant cette expérience un \emph{scenario}.

\subsection{Description d'un scenario}

Ce qu'on appelle ici un \emph{scenario} n'est ni plus ni moins que la description d'une expérience.
Par exemple on pourrait vouloir "observer les performances en gigaflops d'une multiplication de matrices carrées sur le coeur 0 d'une machine".
C'est un scenario simple, et l'exemple que l'on prendra pour illustrer les points un peu plus formel qui vont suivre.

En pratique un scenario est défini de la manière suivante :
\begin{itemize}
 \item Un ensemble de données et variables
 \item Une liste d'actions à effectuer
 \item Un ensemble de caractéristiques à observer
\end{itemize}

Cette définition très générique permet beaucoup de flexibilité, et les sections suivantes précisent les différentes caractéristiques de ces points, ainsi que des exemples concrêt d'utilisation.

Il est important que le format de description d'un scenario soit humainement lisible, et ne conduise pas à une recompilation systématique du programme. C'est donc une description en YAML qui a été choisie.

\subsubsection{Données et variables}

Elles sont indispensables car c'est la dessus que vont se baser les actions du scenario.

L'utilisateur doit fournir les noms et type des variables utilisées en paramètres des différents noyaux, elles peuvent être réutilisées par différents noyaux, mais dans tous les cas l'utilisateur est le seul responsable de leur gestion.

En pratique ces variables peuvent soit être des constantes, ou bien être initialisées par une action dans le scénario.

Pour revenir à l'exemple du scenario simple ou l'on souhaite exécuter une multiplication de matrices carrées - |dgemm| - sur un coeur donné, nous avons besoin de trois matrices |a|, |b|, et |c|, ainsi que d'une taille de bloc, |block_size|.

Voici concrêtement à quoi ressemblerait la déclaration de ces données :

\begin{lstlisting}[language=yaml,caption=Exemple de déclaration de variables,label=lst:tool:data-example]
data:
  - a:
    - type: "double *"
  - b:
    - type: "double *"
  - c:
    - type: "double *"
  - block_size:
    - type: "int"
    - value: 256
\end{lstlisting}

\subsubsection{Actions}

C'est là où on décrit effectivement les noyaux exécutés au cours du scénario.
L'utilisateur indique une série d'actions à exécuter, et avec quels paramètres.

Chaque action peut avoir les caractéristiques suivantes :
\begin{itemize}
  \item core: nombre entier indiquant le coeur sur lequel exécuter l'action.
  \item kernel: chaine de caractères avec le nom de l'action, correspondant à un noyau connu du programme.
  \item params: liste de variables à passer à l'action, leur nom doit correspondre à des données déclarées dans la section précédente.
  \item repeat: nombre entier indiquant le nombre de fois que cette action doit être répétée.
  \item sync: booléen indiquant s'il faut synchroniser le démarrage de cette action avec celui des autres actions présentes dans les file d'attente des autres coeurs.
\end{itemize}

Si on continue dans l'exemple simple d'une multiplication de matrices carrées, il faut que l'on définisse les actions suivantes : l'initialisation de chaque matrice, le lancement du dgemm une fois que ces matrices sont initialisées.
Afin d'avoir une mesure plus précise du comportement du noyau, on peut indiquer une répétition du noyau, ici on choisi 50 pour l'exemple.

Voici en pratique à quoi ressemblerait un tel scenario :
\begin{lstlisting}[language=yaml,caption=Exemple de déclaration d'actions,label=lst:tool:actions-example]
actions:
  - kernel: init_blas_bloc
    sync: false
    params: 
    - a
    - block_size
    core: 0
    # Pour l'initialisation, un seul appel suffit
    repeat: 1
  - kernel: init_blas_bloc
    sync: false
    params: 
    - b
    - block_size
    core: 0
    repeat: 1
  - kernel: init_blas_bloc
    sync: false
    params: 
    - c
    - block_size
    core: 0
    repeat: 1
  - kernel: dgemm
    # Les actions sont executees dans l'ordre de declaration, pour un seul dgemm c'est inutile de synchroniser
    sync: false
    params: 
    - a
    - b
    - c
    - block_size
    core: 0
    # Ici c'est le noyau de calcul, on veut donc le repeter
    repeat: 50
\end{lstlisting}

Simplement exécuter ces actions ne nous donnera pas grand chose, il faut donc définir un ensemble de paramètres à observer.

\subsubsection{Observateurs}

L'outil propose un certain nombre de caractéristiques observables :

\begin{itemize}
  \item le temps passé dans l'action (en millisecondes).
  \item pour les noyaux d'algèbre linéaire, la performance de l'action en Gflops.
  \item des compteurs de performances à travers PAPI
\end{itemize}

En continuant sur notre exemple, voici à quoi ressemblerait la section du scénario si nous souhaitions observer la performance de la multiplication de matrice en Gflops, le nombre de cycles passé dans l'action, ainsi que le nombre de cache miss au niveau 3 généré par l'action.

\begin{lstlisting}[language=yaml,caption=Exemple de déclaration d'observateurs,label=lst:tool:watchers-example]
watchers:
  flops_dgemm:
    # Le nombre de flops depend de la taille de bloc,
    # il faut donc la donner en parametre
    - block_size
  papi:
    - PAPI_TOT_CYC
    - PAPI_L3_TCM
\end{lstlisting}

\subsection{Design de l'outil}

Afin de garantir un minimum de "bruit" lors des expériences, il fallait que l'archicture de l'outil soit simple, avec peu de logique relative au contrôle de l'exécution des tâches.

Le flot d'exécution est le suivant :

\paragraph{Lecture et analyse du scenario}
L'outil charge le scenario fourni par l'utilisateur, créer les différentes données, et analyse les actions pour déterminer l'ensemble des coeurs physiques qui seront utilisés au cours des actions.

\paragraph{Déroulement des actions}
Pour chacun des coeurs utilisés, un thread est créé et attaché à ce coeur. De plus, une queue d'actions (FIFO) est créée pour ce thread.
L'ensemble des actions sont poussées dans les tâches correspondantes dans l'ordre du fichier.

Dès que les threads sont créés ils commencent à exécuter les actions prêtes ; avant et après chaque actions l'ensemble des observateurs pour cette actions sont appelés afin de les collecter pour cette action.

Lorsque la dernière action du scenario est terminée, l'ensemble des données des observateurs est affiché à l'utilisateur sur la sortie standard.

\subsection{Exemples de noyaux}

\subsubsection{Test de bande passante}

Noyau naïf pour estimer la bande passante noeud à noeud

\subsubsection{dgemm naif}

cf les perf plus haut, pour vérifier que le comportement des caches miss était pas lié à l'implem.


\subsection{Application aux noyaux de Cholesky}

On a identifié les noyaux BLAS important dans la section~\ref{chap:contribs:apps:cholesky}.

\subsubsection{Quelles caractéristiques observer ?}
Perf ctr, perfs tout court

\paragraph{Impact de la taille de bloc}

GRAPHE : 4.3.5, dgemm/dsyrk/dtrsm faire : graphe sur tailles de bloc critiques vis à vis L2/L3, accès locaux (obj: voir l'impact de la taille de bloc)
Meilleur cas : tout local, quelques tailles < L2, < L3, > L3

\paragraph{Impact des accès distants}

GRAPHE : 4.3.5, dgemm/dsyrk/dtrsm faire : Pour "bons" blocs, montrer impact de lecture distante, + écriture distante.

Pire cas : tout distant, quelques tailles < L2, < L3, > L3

\subsubsection{Impact de la bibliothèque BLAS}

GRAPHE : 4.3.5, dgemm/dsyrk/dtrsm faire : étude de comportement sur des paramètres représentatifs (obj : montrer l'impact des blas)
OpenBLAS, MKL, ATLAS
matter for top perf, not for overall behavior

\subsection{Conclusion : modélisation de base possible}

