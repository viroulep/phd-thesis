
\section{Réponse à un manque pour le programmeur}
\subsection{Description du besoin}


\subsection{Ajout d'une clause affinité}

\begin{todo}
    le reste de cette section c'est récupérer d'un papier, à mettre en page et améliorer
\end{todo}

Cette partie détaille l'introduction du mot clé |affinity|, ainsi que l'implémentation côté support
exécutif pour exploiter cette fonctionnalité.

Les deux principaux composants des architectures NUMA que l'on considère pour
cette proposition sont les cœurs et les nœuds. Un point clé pour obtenir
de bonnes performances sur des architectures NUMA est de garantir qu'une tâche
s'exécute \emph{proche} de ses données.
On distingue donc trois types d'affinité que le programmeur pourrait avoir besoin
d'exprimer :

\begin{description}
    \item [affinité à un thread :]
      le support exécutif devrait essayer d'ordonnancer la tâche sur le thread donné.
    \item [affinité à un nœud NUMA :]
      le support exécutif devrait essayer d'ordonnancer la tâche sur n'importe
      quel thread du nœud NUMA donné.

    \item [affinité à une donnée :]
      quand une tâche devient prête pour l'exécution, le support exécutif devrait
      l'ordonnancer sur n'importe quel thread attaché au nœud NUMA sur lequel
      la donnée a été physiquement allouée.
\end{description}

De plus, le programmeur peut indiquer si cette affinité est \emph{stricte},
indiquant que la tâche \textbf{doit} s'exécuter sur la ressource indiquée.
Si le programmeur n'indique pas une affinité stricte, l'ordonnanceur peut décider
d'exécuter la tâche sur une ressource différente, pour assurer l'équilibrage de
charge par exemple.

Cette extension visant les constructions de type tâche, elle a été implémentée
comme une nouvelle clause pour la directive |task|. La syntaxe proposée est la
suivante :

\begin{lstlisting}
affinity([node | thread | data]: expr[, strict])
\end{lstlisting}

Si \emph{expr} désigne un id de thread, elle devrait désigner l'id de thread dans
les |OMP_PLACES| définies pour la \textit{team} courante. Exemple : si les places
sont |"{2},{5},{8}"|, et que \emph{expr} est évaluée comme valant 0, l'id du thread désigné est |"{2}"|.

Si \emph{expr} désigne un id de nœud NUMA, elle devrait désigner un id de nœud
dans l'ensemble des nœuds NUMA construit à partir de la liste des |OMP_PLACES|.

Si \emph{expr} désigne une donnée, elle devrait être une adresse mémoire.
Si le nœud NUMA associé à la donnée ne peut être déterminé, le nœud par défaut
est le premier dans la \textit{team} OpenMP.

Si \emph{expr} désigne une ressource hors limites, la valeur est prise modulo le
nombre de ressources.

\subsection{topo/discussion description des trucs précédents + qu'est ce qu'on peut utiliser dans le runtime}

