\section*{Discussions et conclusion}

Une analyse préliminaire des architectures et applications a permis de trouver et quantifier l'impact d'un paramètre majeur sur les performances des parties critiques d'application~: la localité des données.
Ce chapitre a montré qu'il était possible d'étendre un modèle de programmation et ses supports exécutifs dans but double~: enrichir le graphe de tâches d'une application avec des informations sur l'affinité entre une tâches et ses données~; et utiliser efficacement ces informations lors de l'ordonnancement de l'application.

Parmi les améliorations possibles, nous avons pu constater que l'impact de la localité des données était directement liée à la quantité de données manipulées.
On peut également supposer que le type d'opération effectuée sur ces données pourrait avoir un impact.
De plus dans certains cas le compilateur pourrait fournir ce type d'information au support exécutif.
Il y a donc une opportunité pour des futurs travaux sur l'amélioration de l'intéraction entre le compilateur et les supports exécutifs.

Se pose aussi la question de la performance du support exécutif~: compte tenu des performances de référence des noyaux, est-ce que la performance obtenue via un ordonnancement par vol de travail est proche du maximum atteignable ?
Existe-t-il un meilleur ordonnancement ?
Nous avons commencé à développer un simulateur, présenté dans le chapitre~\ref{chap:simulation}, dont l'un des objectifs serait de répondre à ces questions.
