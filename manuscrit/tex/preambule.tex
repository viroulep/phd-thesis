% Couverture
\MakeUGthesePDG

\clearpage
\hbox{}\newpage

\begin{center}\textbf{\large Étude et amélioration de l'exploitation des architectures NUMA à travers des supports exécutifs}

\quad

\textbf{Résumé}
\end{center}

L'évolution du calcul haute performance est aujourd'hui dirigée par les besoins des applications de simulation numérique.
Ces applications sont exécutées sur des supercalculateurs qui peuvent proposer plusieurs milliers de cœurs, et qui sont découpés en un très grand nombre de nœuds de calcul ayant eux un nombre de cœurs beaucoup plus faible.
Chacun de ces nœuds de calcul repose sur une architecture à mémoire partagée, dont la mémoire est découpée en plusieurs blocs physiques différents : cela implique un temps d'accès dépendant à la fois de la donnée accédée ainsi que du processeur y accédant.
On appelle ces architectures NUMA (pour \emph{Non Uniform Memory Access}).

La manière actuelle de les exploiter tend vers l'utilisation d'un modèle de programmation à base de tâches, qui permet de traiter des programmes irréguliers au dela du simple parallélisme de boucle.
L'exploitation efficace des machines NUMA est critique pour l'amélioration globale des performances des supercalculateurs.
Cette thèse a été axée sur l'amélioration des techniques usuelles pour leur exploitation : elle propose une réponse au compromis qu'il faut faire entre localité des données et équilibrage de charge, qui sont deux points critiques dans l'ordonnancement d'applications.
Les contributions de cette thèse peuvent se découper en deux parties : une partie dédiée à fournir au programmeur les moyens de comprendre, analyser, et mieux spécifier le comportement des parties critiques de son application, et une autre partie dédiée à différentes améliorations du support exécutif.
Cette seconde partie a été évaluée sur différentes applications, ce qui a permis de montrer des gains de performances significatifs.



\quad

\textbf{Mots-clés} : NUMA, support exécutif, OpenMP, tâches, caractérisation, simulation

\subsection*{Financement}

Ce travail a été réalisé dans le cadre du projet ELCI, un projet collaboratif Français financé par le FSN ("Fond pour la Société Numériqe"), qui associe des partenaires académiques et industriels pour concevoir et produire un
environnement logiciel pour le calcul intensif.



\vfill
\pagebreak

\begin{center}\textbf{\large Studying and improving the use of NUMA architectures through runtime systems }

\quad

\textbf{Abstract}
\end{center}


Nowadays the evolution of High Performance Computing follows the needs of numerical simulations.
These applications are executed on supercomputers which can offer several thousands of cores, split into a large number of computing nodes, which possess a relatively low number of cores.
Each of these nodes consists of a shared memory architecture in which the memory is physically split into several distinct blocks: this implies that the memory access time depends both on which data is accessed, and on which core tries to access it.
These architectures are named NUMA (for \emph{Non Uniform Memory Access}).

The current way to exploit them tends to be through a tasks-based programming model, which can handle irregular applications beyond a simple loop-based parallelism.
Efficient use of NUMA architectures is critical for the overall performance improvements of supercomputers.
This thesis focuses on improving common techniques for the exploitation of these architectures: it proposes an answer to the tradeoff that has to be made between data locality and load balancing, that are two critical aspects of applications scheduling.
Contributions of this thesis can be split into two parts: the first part is dedicated to providing the programmer with means to understand, analyze, and better characterize the behavior of their applications' critical parts, and the second part is dedicated to several improvements made to the runtime systems.
This last part has been evaluated on various applications and has shown some significant performance gains.


\quad

\textbf{Keywords}: NUMA, runtime systems, OpenMP, tasks, characterization, simulation

\begin{savequote}[9cm]
<< If we knew what it was we were doing, it would not be called research, would it? >>

\qauthor{Albert Einstein}

\end{savequote}

\chapter*{Remerciements}

Je souhaite dans un premier temps remercier les deux personnes qui m'accompagnent depuis que j'ai démarré ma vie scientifique il y a 5 ans~: Thierry Gautier et François Broquedis.
Merci de m'avoir fait apprécier la recherche et de m'avoir trouvé un sujet sur mesure, de m'avoir fait confiance, et d'avoir su m'encourager.
Malgré les changements d'équipes vous avez été assidus dans mon encadrement, aussi assidus que pour faire des enfants, puisqu'à vous deux vous en avez fait plus que le nombre d'années qu'il m'a fallu pour faire cette thèse~!
Je ne me serais pas engagé dans cette thèse si cela avait été avec d'autres encadrants.

Je souhaite également remercier Fabrice Rastello de m'avoir accueilli dans l'équipe CORSE, et d'avoir pris la direction de ma thèse à ce moment là.
Tu as indéniablement apporté plein d'idées à cette thèse, et plein de compétences que je n'aurais pas pu avoir dans un autre contexte.

J'adresse des chaleureux remerciements à Emmanuel Jeannot et Julien Langou  pour avoir accepté de rapporter mon manuscrit, ainsi qu'à Karine Heydemann pour avoir accepté de prendre part à mon jury en tant qu'examinatrice.

Merci à tous mes collègues des équipes MESCAL, MOAIS, NANO-D, et CORSE d'avoir partagé ces années de thèse avec moi.
En particulier à Annie et Imma d'avoir survécu aux démarches administratives~; à Mathias, Marie, et Pierrick mes co-bureaux de Montbonnot~: malgré vos efforts pour m'avertir des <<dangers>> de la thèse je ne vous ai pas écoutés ;)

%\greektext ευχαριστώ \latintext
Merci
George pour m'avoir donné des cours de Grec en plus de tolérer mes nombreux goodies, \textbf{Fa\textcolor{BrickRed}{bi}\textcolor{Goldenrod}{an}} \& François pour avoir hébergé de nombreuses soirées très agréables, et bien sûr Raphaël pour m'avoir permi de ne pas être à la rue pendant les 6 derniers mois de ma thèse~! Je n'oublie pas non plus tous les autres avec qui on a pu partager de nombreuses bières et pizzas à Grenoble.

Parce que 3 ans de thèse c'est aussi 52 compétitions de speedcubing, je voudrais remercier Zecho, Zippo, Lina, Rio, Po, l'homme le plus intelligent de France~\cite{TF1}, et tous les autres cubeurs et cubeuses en France ou à l'international que j'oublie~: vous avez activement participé au maintien de ma bonne santé mentale~!
Un énorme merci en particulier à celle et ceux qui ont relu mon manuscrit et ont permis de retirer l'aléatoire sur la présence des 's' (ou pas).

J'ai aussi de gros remerciements à faire aux personnes avec qui j'ai pu partager de nombreuses soirées dans la joie et la bonne humeur au Fam's (avec ou sans coinche)~: Mazette, Noémie, Archnouff, Arnaud, Do et Tout Rouge, entre autres.

Merci Fred et Mouchoum d'avoir été là (avant de m'abandonner lâchement :p) et de m'avoir fait découvrir toutes les joyeusetés qu'il est possible de faire à partir de vermicelles arc-en-ciel.

Un énorme merci à Testi (pour tout en fait, tout lister ça serait bien trop long), on ne l'aura pas gagné ce tournoi mais les deux bonnes nouvelles apportées cette dernière année compensent largement~!


Je tiens à remercier tout particulièrement ma famille~; d'une part pour avoir affronté les grèves de la SNCF pour venir me voir parler pendant 45 minutes d'un sujet incompréhensible, mais surtout pour m'avoir supporté dans mon parcours et avoir été présents quand il le fallait.


Enfin il y en a une qui a du passer ces quelques paragraphes à chercher son nom, mais je l'ai gardée pour la fin~: merci Manou~!
Merci d'avoir été là pendant ces 3 ans, merci de ne pas être (trop) désespérée quand je m'enfile un pot d'Häagen-Dazs devant la télé, quand je décide d'être productif entre 22h et 1h, ou encore quand je propose des menus bien trop gras pour être raisonnables.
Je suis déjà un peu fou de base, je suis bien content d'avoir trouvé un peu plus de folie ailleurs pour stabiliser tout ça.
J'ai vraiment bien fait d'arriver à l'arrache totale dans un pays dont je ne parle pas la langue et de squatter le taxi d'une inconnue sans dépenser un centime.
Vive les conférences au Brésil ! ;)

% TOC
\cleardoublepage
\dominitoc
\makeatletter
\renewcommand{\contentsname}{Sommaire}
\renewcommand{\tableofcontents}[1][\contentsname]{%
  \chapter*{#1}
  \@starttoc{toc}
}
\makeatother
\tableofcontents
