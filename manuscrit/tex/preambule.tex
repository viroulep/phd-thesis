% Couverture
\MakeUGthesePDG

\clearpage
\ifodd\value{page}\hbox{}\newpage\fi

\begin{center}\textbf{\large Étude et amélioration de l'exploitation des architectures NUMA à travers des supports exécutifs}

\quad

\textbf{Résumé}
\end{center}

Résumé

Les simulations numériques sont aujourd'hui omniprésentes dans l'industrie.
Que ce soit pour la météo, l'aéronautique, ou les applications militaires, ces simulations sont extrèmement gourmandes en ressources, et pourront toujours en utiliser plus si on leur en donne plus !
Ces applications sont exécutées sur des supercalculateurs avec plusieurs milliers de coeurs, qui sont en fait composés de multiples machines avec un plus faible nombre de coeurs.

Au cours de la dernière décennie, on a pu noter que l'intérêt des fabricants se portait sur la multiplication des coeurs au sein d'un processeur, plutôt que sur l'augmentation de la fréquence de ce dernier.
Cela a donné naissance a des machines multiprocesseurs dont l'accès à la mémoire centrale n'est pas uniforme en fonction du processeur qui l'accède (NUMA).
Ce type de machines est au coeur des supercalculateurs, et l'optimisation de leur utilisation est la clé pour l'amélioration globale des performances.

Dans cette thèse nous proposons d'une part une étudie approfondie du comportement de ce type d'architectures face à des éléments de base des simulations numériques ; et d'autre part des techniques d'optimisation des supports exécutifs ciblant ces architectures, avec comme standard d'application OpenMP~4.0.

Ces techniques ont été évaluées sur un spectre varié de machines du type NUMA.



\quad

\textbf{Mots-clés} : trucs

\begin{center}\textbf{\large TODO }

\quad

\textbf{Abstract}
\end{center}

TODO

\quad

\textbf{Keywords} : stuff

%\begin{savequote}[7cm]
%<< Blah
%
%\quad blabla
%
%\quad blablabla >>
%\qauthor{TG}
%\end{savequote}
%% Remerciements...
%\chapter*{Remerciements}
%
%\begin{todo}
%Test todo
%\end{todo}


% TOC
\cleardoublepage
\dominitoc
\makeatletter
\renewcommand{\tableofcontents}[1][\contentsname]{%
  \chapter*{#1}
  \@starttoc{toc}
}
\makeatother
\tableofcontents
