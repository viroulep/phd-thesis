% Couverture
\MakeUGthesePDG

\clearpage
\ifodd\value{page}\hbox{}\newpage\fi

\begin{center}\textbf{\large Étude et amélioration de l'exploitation des architectures NUMA à travers des supports exécutifs}

\quad

\textbf{Résumé}
\end{center}

L'évolution du calcul haute performance est aujourd'hui dirigée par les besoins des applications de simulation numérique.
Ces applications sont exécutées sur des supercalculateurs qui peuvent proposer plusieurs milliers de cœurs, et qui sont découpés en un très grand nombre de nœuds de calcul ayant eux un nombre de cœurs beaucoup plus faible.
Chacun de ces nœuds de calcul repose sur une architecture à mémoire partagée, dont la mémoire est découpée en plusieurs blocs physiques différents : cela implique un temps d'accès dépendant à la fois de la donnée accédée ainsi que du processeur y accédant.
On appelle ces architectures NUMA (pour \emph{Non Uniform Memory Access}).

La manière actuelle de les exploiter tend vers l'utilisation d'un modèle de programmation à base de tâches, qui permet de traiter des programmes irréguliers au dela du simple parallélisme de boucle.
L'exploitation efficace des machines NUMA est critique pour l'amélioration globale des performances des supercalculateurs.
Cette thèse a été axée sur l'amélioration des techniques usuelles pour leur exploitation : elle propose une réponse au compromis qu'il faut faire entre localité des données et équilibrage de charge, qui sont deux points critiques dans l'ordonnancement d'applications.
Les contributions de cette thèse peuvent se découper en deux parties : une partie dédiée à fournir au programmeur les moyens de comprendre, analyser, et mieux spécifier le comportement des parties critiques de son application, et une autre partie dédiée à différentes améliorations du support exécutif.
Cette seconde partie a été évaluée sur différentes applications, ce qui a permis de montrer des gains de performances significatifs.



\quad

\textbf{Mots-clés} : NUMA, support exécutif, OpenMP, tâches, caractérisation, simulation

\begin{center}\textbf{\large Studying and improving the use of NUMA architectures through runtime systems }

\quad

\textbf{Abstract}
\end{center}


Nowadays the evolution of High Performance Computing follows the needs of numerical simulations.
These applications are executed on supercomputers which can offer several thousands of cores, split into a large number of computing nodes, which possess a relatively low number of cores.
Each of these nodes consists of a shared memory architecture, in which the memory is physically split into several distinct blocks: this implies that the memory access time depends both on which data is accessed, and on which core tries to access it.
These architectures are named NUMA (for \emph{Non Uniform Memory Access}).

The current way to exploit them tends to be through a tasks-based programming model, which can handle irregular applications beyond a simple loop-based parallelism.
Efficient use of NUMA architectures is critical for the overall performance improvements of supercomputers.
This thesis has been targetted at improving common techniques for their exploitation: it proposes an answer to the tradeoff that has to be made between data locality and load balancing, that are two critical aspects of applications scheduling.
Contributions of this thesis can be split into two parts: the first part is dedicated to providing the programmer with means to understand, analyze, and better characterize the behavior of their applications' critical parts, and the second part is dedicated to several improvements made to the runtime systems.
This last part has been evaluated on various applications, and has shown some significant performance gains.


\quad

\textbf{Keywords}: NUMA, runtime systems, OpenMP, tasks, characterization, simulation

%\begin{savequote}[8cm]
%<< Une thèse sans théorème ce n'est pas une thèse >>
%\qauthor{Denis Trystram}
%\end{savequote}
%% Remerciements...
%\chapter*{Remerciements}

%\begin{todo}
%Test todo
%\end{todo}


% TOC
\cleardoublepage
\dominitoc
\makeatletter
\renewcommand{\contentsname}{Sommaire}
\renewcommand{\tableofcontents}[1][\contentsname]{%
  \chapter*{#1}
  \@starttoc{toc}
}
\makeatother
\tableofcontents
