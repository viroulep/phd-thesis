\section{Évolution d'un modèle de programmation : OpenMP}\label{sec:context:openmp}

OpenMP~\cite{openmp45} est le standard de-facto pour les machines à mémoire partagée.

Jusqu'à la version 2.0 incluse, OpenMP ne proposait qu'un nombre restreint de fonctionnalités, principalement tournées vers la parallélisation de boucles régulières.
L'évolution de son utilisation a amené le comité d'architecture d'OpenMP à ajouter le concept de \emph{tâche} à partir de la version 3.0, en s'inspirant de modèle de programmation déjà existant tel que Cilk.
La version 4.0 d'OpenMP étend les tâches pour ajouter les dépendances de données, et introduit le support pour du matériel de plus en plus utilisé : les accélérateurs.
On y voit aussi l'apparition de l'affinités de thread, des directives SIMD, ou encore des instructions atomiques.
La version actuelle d'OpenMP est la version 4.5, qui ajoute certaines fonctionnalités aux constructions existantes, et introduit le concept de \emph{taskloop} : des boucles dont les groupes d'itérations forment des tâches indépendantes.
La version en cours de préparation est OpenMP~5.0 ; parmi diverses améliorations, elle devrait notamment ajouter le support d'une interface utilisable par des outils souhaitant se greffer au plus proche d'OpenMP.

Les sections suivantes décrivent les différentes fonctionnalités d'OpenMP, dans l'ordre chronologique où elles sont apparues.

\subsection{Boucles}

OpenMP a été pensé dès le départ comme un modèle \emph{fork-join} (illustré sur la figure~\ref{fig:context:fork-join}), et l'utilisation de base ciblait la parallélisation de boucles régulières.

GRAPHE: \label{fig:context:fork-join}fork join

Le modèle fonctionne de la manière suivante : le programme est exécuté séquentiellement jusqu'à la rencontre d'une région parallèle (cela peut être par exemple une boucle).
À partir de là, plusieurs threads sont créés (ou réutilisés) pour exécuter la construction en parallèle. Cet ensemble de threads forme une \emph{team} de thread pour la région parallèle.
À la fin de la région parallèle, tous les threads se synchronisent et l'exécution séquentielle est reprise.
Le programmeur peut contrôler combien de threads sont utilisés dans la \emph{team} créée lors de l'entrée dans la région parallèle.

Lorsque les threads rencontre une construction de partage de travail - ou \emph{worksharing construct} - telle qu'une boucle par exemple, l'ensemble des itérations va être réparti parmi les différents threads de la team.
Le programmeur peut contrôler cette répartition à l'aide de la clause |schedule|.

Pendant l'exécution de l'application c'est un programme spécifique - le \emph{support exécutif} qui est en charge de l'équilibrage de charge, et de l'affectation du "travail" aux différent threads.
Le langage OpenMP est standard et bien défini, mais les performances dépendent directement du support exécutif.

Pour complémenter le langage, OpenMP propose également plusieurs fonctionnalités à travers une API. Il est possible par exemple de récupérer dynamiquement le nombre total de threads, l'indice du thread courant, ou encore d'ajuster l'ordonnanceur de boucles.
Tout cela existe depuis les toutes premières versions d'OpenMP.


\subsection{Tâches}

Les modèles de programmation à base de tâches permettent d'exprimer du parallélisme à grain fin. L'un des avantages majeurs de ce modèle est qu'il permet au support exécutif d'assigner dynamiquement les différentes tâches, s'adaptant ainsi très bien aux déséquilibrages de charge. Il permet également de composer très facilement des parties de code avec des caractéristiques différentes.

Une \emph{tâche} OpenMP peut être vue comme la plus petite \emph{quantité de travail} qu'un thread OpenMP peut exécuter.
Les tâches peuvent être créées par un thread OpenMP et exécutées par n'importe quel thread de la région parallèle.
Comme la création de tâches à l'exécution du programme est beaucoup plus économique que la création et la synchronisation de threads, le développeur peut pousser la parallélisation de son application encore plus loin : il peut considérer la parallélisation de portions de code qui avaient un grain trop fin pour être parallélisées avec des threads.

Dans la version 3.0 d'OpenMP la synchronisation des tâches est effectuée grâce au mot clé |taskwait|, qui indique au support exécutif d'attendre la complétion des tâches générées jusqu'à ce point, avant de reprendre l'exécution.
Le développeur de l'application est responsable de la création et de la synchronisation explicite des tâches, mais c'est le support exécutif qui est en charge de l'affectation des tâches aux threads pendant l'exécution du programme.

La version 4.0 d'OpenMP~\cite{openmp40} pousse le concept de tâche plus loin en ajoutant le mot clé |depend|, spécifiant les modes d'accès de chaque variable partagée utilisée par la tâche pendant son exécution.
Le mode d'accès peut être soit |in|, |out|, ou |inout| selon que la variable correspondante soit respectivement lue comme entrée, écrite en sortie, ou à la fois lue et écrite par la tâche en question.
Cette information peut ensuite être traitée par le support exécutif pour décider si une tâche est prête à être exécutée ou si il faut d'abord attendre la complétion d'une ou plusieurs autres tâches.


\subsection{Vectorisation}

Nous avons vu dans la section~\ref{sec:context:numa:simd} que les processeurs disposent d'instructions vectorielles - \emph{SIMD} - depuis le début des années 2000.
Avec la multiplications des architectures et de la taille des registres, beaucoup de nouvelles instructions spécifiques à certaines tailles de registre sont apparues.

Si la vectorisation automatique n'est pas possible, il est coûteux pour le programmeur de modifier son code en fonction des architectures.
OpenMP~4.0 tente de résoudre se problème par l'introduction d'une construction |simd|, qui permet au programmeur d'indiquer quelles parties de son code peuvent être vectorisées et avec quelles contraintes (longeur maximale, alignement, quelles sont les variables d'itérations, ...).


\subsection{Accélérateurs}

La majorité des supercalculateurs intègrent des accélérateurs (comme des GPUs).
L'exploitation de ces accélérateurs imposent parfois l'usage d'un language spécifique au constructeur (tel que Cuda).

Bien que des standards, comme OpenCL~\cite{Stone2010}, existent pour cibler différents types d'accélérateurs, cibler ce type d'architecture via OpenMP était impossible.

OpenMP~4.0 introduit la construction |target|, qui permet de demander au compilateur de créer une tâche à partir de la région de code sélectionnée, qui peut ensuite être exporté sur accélérateur.
Plusieurs clauses permettent de spécifier des dépendances, ainsi que l'ensemble des données à transférer vers l'accélérateur, et l'ensemble des données à rapatrier depuis l'accélérateur à l'issue des calculs.

Si la norme indique le support de plusieurs accélérateurs, ils doivent cependant être tous du même type, et la transformation du code C/C++/Fortran du bloc est laissé au compilateur implémentant la norme.
Dans le cas d'utilisation d'appels de fonctions, le programmeur doit indiquer qu'elles disposeront d'une définition pour l'\emph{hôte} et l'\emph{accélérateur} via une construction |declare target|.

Ces contraintes font qu'aujourd'hui la construction n'est pas clairement adoptée telle quelle par les programmeurs d'architectures hétérogènes.

\subsection{Affinité}

Une notion d'affinité de thread a été également ajoutée à la version 4.0 d'OpenMP.
Il s'agit d'un ensemble de fonctionnalités ayant pour but de donner du contrôle au programmeur sur le placement des threads OpenMP sur la topologie physique de la machine.

Le premier ajout est celui du concept de \emph{places} : il s'agit d'un moyen de représenter des emplacements physiques sur lesquels les threads OpenMP peuvent venir se placer.
Le contrôle de la sélection des \emph{places} se fait via la variable d'environnement |OMP_PLACES|.
Elle peut être une liste précise d'indice de coeurs physiques, ou prendre la valeur d'éléments plus génériques de la machine tels que \emph{sockets} ou \emph{cores}.

Il n'y a pas de dépendances entre le nombre de threads dans une région parallèle (spécifié par |OMP_NUM_THREADS|) et la liste des \emph{places} : il peut y avoir plus de threads que de \emph{places} ou inversement, sans que cela empêche l'exécution du programme.

Le second ajout est la possibilité de spécifier la manière dont sont affectés les threads aux \emph{places}, à l'aide de la variable d'environnement |OMP_PROC_BIND|.
Elle peut prendre les valeurs suivantes : \emph{close}, \emph{spread}, \emph{master}.

\begin{todo}
réécrire cette partie pour faire moins manuel et plus intéressant vis à vis de nos travaux
\end{todo}

Avec la valeur \emph{close}, les threads seront successivement affectés aux places les plus proches les unes des autres. Par exemple si les places sont \emph{"cores"} - l'ensemble des coeurs de la machine, et que la machine a deux sockets de 8 coeurs, les threads OpenMP seront affectés successivement aux coeurs 0 (sur le socket 0) jusqu'à 7 (toujours sur le socket 0), puis 8 (sur le socket 1) jusqu'à 15 (sur le socket 1).
Avec la valeur \emph{spread}, l'ordre d'affectation sera alors le coeur 0 (sur le socket 0), puis 8 (socket 1), puis 1 (socket 0), puis 9 (socket 1), etc...
Avec la valeur \emph{master}, les threads sont placés sur la même place que le thread master.
L'exemple typique pour cette valeur est la gestion de régions parallèles imbriquées : en affectant la valeur "sockets" à |OMP_PLACES|, on peut avoir deux régions parallèles imbriquées dont les threads sont correctement affectés aux deux sockets (les deux threads master est affectés à deux sockets distincts, mais pour chaque région tous les threads sont associées au même socket).

