\section{Évolution d'un modèle de programmation : OpenMP}\label{sec:context:openmp}

TODO: où et comment formuler le fait qu'il faut d'une part maximiser la performance séquentielle, mais aussi maximiser le parallélisme

OpenMP~\cite{openmp40} est le standard de-facto pour les machines à mémoire partagée, au fur et à mesure que le matériel a évolué, le standard a évolué avec lui.

\subsection{Boucles}

OpenMP a été pensé dès le départ comme un modèle \emph{fork-join} (TODO: voir graph), et l'utilisation de base ciblait la parallélisation de boucles régulières.

Le modèle fonctionne de la manière suivante : le programme est exécuté séquentiellement jusqu'à la rencontre d'une région parallèle (cela peut être par exemple une boucle).

À partir de là, plusieurs threads sont créés (ou réutilisés) pour exécuter la construction en parallèle. Cet ensemble de threads forme une \emph{team} de thread pour la région parallèle.

À la fin de la région parallèle, tous les threads se synchronisent et l'exécution séquentielle est reprise.

Le programmeur peut contrôler combien de threads sont utilisés dans la \emph{team}, ainsi que la manière dont sont attribuées les itérations de la boucle.

TODO: parler de static, dynamic, etc ?

Pendant l'exécution de l'application c'est un autre programme spécifique - le \emph{support exécutif} qui est en charge de l'équilibrage de charge, et de l'affectation du "travail" aux différent threads.

Le langage OpenMP est standard et bien défini, mais plusieurs support exécutifs existent et les performances dépendent directement de celui ci.

Pour complémenter le langage, OpenMP propose également plusieurs fonctionnalités à travers une API. Il est possible par exemple de récupérer dynamiquement le nombre total de threads, l'indice du thread courant, ou encore d'ajuster l'ordonnanceur de boucles.

Tout cela existe depuis les toutes premières versions d'OpenMP.
Avec l'évolution du matériel et des besoins des utilisateurs, le comité d'architecture d'OpenMP a ajouté le concept de \emph{tâche} à partir de la version 3.0.



\subsection{Tâches}

Les modèles de programmation à base de tâches permettent d'exprimer du parallélisme à grain fin. L'avantage majeur de ce modèle est qu'il permet au support exécutif d'assigner dynamiquement les différentes tâches, s'adaptant ainsi très bien aux divers temps d'exécution des tâches.

Une \emph{tâche} OpenMP peut être vue comme la plus petite \emph{quantité de travail} qu'un thread OpenMP peut exécuter.
Les tâches peuvent être créées par un thread OpenMP et exécutées par n'importe quel thread de la région parallèle.
Comme la gestion des tâches à l'exécution du programme est beaucoup plus économique que de créer et synchroniser des threads, le développeur peut pousser la parallélisation de son application encore plus loin : il peut considérer la parallélisation de portions de code qui avaient un grain trop fin pour être parallélisées avec des threads.

Dans la version 3.0 d'OpenMP la synchronisation des tâches est effectuée grâce au mot clé |taskwait|, qui indique au support exécutif d'attendre la complétion des tâches générées jusqu'à ce point, avant de reprendre l'exécution.

Le développeur de l'application est responsable de la création et de la synchronisation explicite des tâches, mais c'est le support exécutif qui est en charge de l'affectation des tâches aux threads pendant l'exécution du programme.

La version 4.0 d'OpenMP pousse le concept de tâche plus loin en ajoutant le mot clé |depend|, spécifiant les modes d'accès de chaque variable partagée utilisée par la tâche pendant son exécution.

Le mode d'accès peut être soit |in|, |out|, ou |inout| selon que la variable correspondante soit respectivement lue comme entrée, écrite en sortie, ou à la fois lue et écrite par la tâche en question.

Cette information peut ensuite être traitée par le support exécutif pour décider si une tâche est prête à être exécutée ou si il faut d'abord attendre la complétion d'une ou plusieurs autres tâches.


\subsection{Vectorisation}



Pour cette section : ok + de parallélisme, mais ça c'est spécifiquement pour améliorer les perfs séquentielles

TODO : faire plus de texte ?

OpenMP 4.0 also brought in SIMD - Single Instruction Multiple Data - constructs.
The aim was to provide a standard interface for vectorization instructions.


\subsection{Accélérateurs}

La majorité des supercalculateurs intègrent des accélérateurs (comme des GPUs), et pour faciliter leur exploitation la construction |target| a été ajoutée au standard dans la version 4.0.

TODO : faire plus de texte ?

\subsection{Affinité}

La notion d'affinité de thread a été également ajoutée à la version 4.0 d'OpenMP.
Il s'agit d'un ensemble de fonctionnalités ayant pour but de donner du contrôle au programmeur sur le placement des threads OpenMP sur la topologie physique de la machine.

Le premier ajout est celui du concept de \emph{places} : il s'agit d'un moyen de représenter des emplacements physiques sur lesquels les threads OpenMP peuvent venir se placer.
Le contrôle de la sélection des \emph{places} se fait via la variable d'environnement |OMP_PLACES|.
Elle peut être une liste précise d'indice de coeurs physiques, ou prendre la valeur d'éléments plus génériques de la machine tels que \emph{sockets} ou \emph{cores}.

Il n'y a pas de dépendances entre le nombre de threads dans une région parallèle (spécifié par |OMP_NUM_THREADS|) et la liste des \emph{places} : il peut y avoir plus de threads que de \emph{places} ou inversement, sans que cela empêche l'exécution du programme.

Le contrôle de l'affectation des threads OpenMP aux \emph{places} se fait grâce à la variable d'environnement |OMP_PROC_BIND|.
Elle peut prendre les valeurs suivantes : \emph{close}, \emph{spread}, \emph{master}.

Avec la valeur \emph{close}, les threads seront successivement affectés aux places les plus proches les unes des autres. Par exemple si les places sont \emph{"cores"} - l'ensemble des coeurs de la machine, et que la machine a deux sockets de 8 coeurs, les threads OpenMP seront affectés successivement aux coeurs 0 (sur le socket 0) jusqu'à 7 (toujours sur le socket 0), puis 8 (sur le socket 1) jusqu'à 15 (sur le socket 1).
Avec la valeur \emph{spread}, l'ordre d'affectation sera alors le coeur 0 (sur le socket 0), puis 8 (socket 1), puis 1 (socket 0), puis 9 (socket 1), etc...
Avec la valeur \emph{master}, les threads sont placés sur la même place que le thread master.
L'exemple typique pour cette valeur est la gestion de régions parallèles imbriquées : en affectant la valeur "sockets" à |OMP_PLACES|, on peut avoir deux régions parallèles imbriquées dont les threads sont correctement affectés aux deux sockets (les deux threads master est affectés à deux sockets distincts, mais pour chaque région tous les threads sont associées au même socket).
