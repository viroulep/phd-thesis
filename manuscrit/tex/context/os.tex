\section{Exploitation des architectures NUMA par le système d'exploitation}\label{sec:context:os}

Le support des machines NUMA dans Linux est arrivé dès 2003~\cite{Dobson2003}.
Détailler les dispositifs spécifiques à l'exploitation des machines NUMA au niveau système ne peut pas se faire sans avoir connaissance de certains mécanismes déjà existants.
La section suivante détaille donc la manière dont Linux gère la mémoire d'une machine <<standard>>, puis nous détaillerons les spécificités liées au caractère NUMA de la machine, ainsi que certaines bibliothèques externes utiles pour les programmeurs.

\subsection{Gestion de la mémoire}

La compréhension de la gestion de l'allocation de la mémoire par le système d'exploitation est un point critique lorsque l'on traite les machines NUMA.
Le système d'exploitation gère une mémoire \emph{paginée}.
Lors de l'allocation d'un tableau par exemple, si sa taille dépasse celle d'une \emph{page}, l'espace mémoire alloué consistera donc de plusieurs pages.
Lorsqu'un allocateur utilise une primitive d'allocation de pages telle que \emph{mmap}, elles sont allouées selon une politique d'allocation physique des pages.
La politique par défaut dans Linux est d'allouer physiquement la page lors du premier accès - \emph{first-touch}, sur le nœud NUMA du cœur effectuant ce premier accès.
Dans ces conditions, le contrôle de l'allocation physique n'est possible que lors de la première allocation des \emph{pages}.
Cette contrainte est limitée à la durée de vie du programme, mais il est aussi possible d'utiliser des librairies externes pour gérer plus finement l'allocation de données, comme décrit dans la section suivante.

Enfin, un dernier détail mérite de l'attention, il s'agit de la taille des \emph{pages} manipulées par le système.
En effet celle-ci est configurable, et deux options sont généralement possibles~: soit des \emph{pages} de 4 Kilo octets, soit des - \emph{huge} - \emph{pages} de 2 Mega octets.

\subsection{Prise en compte des architectures NUMA et bibliothèques externes}\label{sec:context:os:lib}

Avec l'arrivée du support des machines NUMA dans le système d'exploitation Linux, plusieurs fonctionnalités fondamentales sont apparues et parmi elles on retrouve~: la capacité d'obtenir des informations à propos de la machine, telles que le numéro de nœud d'un cœur donné, ou le numéro de nœud d'une adresse mémoire donnée.
On trouve aussi la possibilité de contrôler le placement d'un thread sur un ensemble de cœurs physiques à l'aide d'un masque d'affinité, ce qui permet par exemple de garantir la proximité physique de deux threads vis à vis d'un nœud, ou tout simplement de garantir qu'un thread sera exécuté par un cœur précis.

Plusieurs outils se sont basés sur ces briques de base pour proposer des fonctionnalités plus avancées, comme par exemple~:

\begin{description}
  \item [numactl~\cite{numactl} :] cet outil s'utilise avant l'utilisation d'un exécutable sur la ligne de commande. Il permet~: de contrôler l'ensemble des cœurs (ou nœuds) sur lesquels les threads peuvent être placés au cours de l'exécution du programme ;
    de contrôler l'ensemble des nœuds NUMA sur lesquels peuvent être allouées des données ;
    et encore de modifier la politique d'allocation des \emph{pages} du système, en permettant par exemple de distribuer les \emph{pages} successives allouées de manière cyclique sur un ensemble de nœuds.
  \item [hwloc~\cite{Broquedis2010} :] cet outil permet, entre autres, d'obtenir des informations précises sur le matériel et la topologie de la machine. Il expose les facteurs NUMA théoriques fournis par le matériel, et permet de visualiser les différents nœuds et groupes de nœuds.
  Il propose également des fonctions d'allocation plus fiables que |malloc|, garantissant l'allocation de nouvelles \emph{pages}, et proposant différentes politiques d'allocation (incluant le \emph{first-touch}).
\item [MaMi~\cite{Broquedis2009} :] cet allocateur est implémenté au sein de ForestGOMP, décrit en détails dans la section~\ref{sec:rw:numa:thread-data}. Il permet l'allocation et la migration de \emph{pages}, soit explicitement via des appels dédiés, soit implicitement en marquant des pages à migrer au \emph{next-touch}~: elles seront alors migrées sur le nœud du prochain cœur qui y accède.
\end{description}


\bigskip

Dans la section précédente nous avons vu les détails concernant le matériel utilisé au cours de cette thèse, et dans cette section nous avons vu comment l'utiliser au niveau du système d'exploitation.

Le développeur d'applications scientifiques n'est généralement pas un spécialiste des fonctionnalités de base du système d'exploitation~: il va plutôt être un expert de son application et des parties critiques qui la composent.
Il faut donc une couche par dessus le système d'exploitation pour lui permettre d'exprimer de manière plus abstraite son application. Cette couche est le \emph{modèle de programmation}.
Pour cibler les architectures à mémoire partagée, les modèles de programmation à base de tâches offrent des propriétés très intéressantes~: ils permettent d'exprimer du parallélisme à grain fin, et l'équilibrage de charge est effectué de manière dynamique par le système exécutif.
Cela permet potentiellement de maximiser l'utilisation du nombre important de ressources disponibles sur une machine NUMA.
L'expression d'un programme à base de tâches permet également d'isoler les parties critiques du programme, ce qui peut faciliter leur analyse.
