\section{NUMA}\label{sec:context:numa}

The memory topology is organized by pairs of NUMA nodes connected together through Intel QuickPath Interconnect.
These pairs can communicate together through a proprietary fabric called NUMALink6 with up to two hops.
%, represented by the edges of the graph.In other words, taking node $n_0$ as example :
%\begin{itemize}
%\item node $n_0$ communicates with node $n_1$ through Intel QPI ;
%\item node $n_0$ is one hop away from node $n_4$ \emph{(e.g. communications between node $n_0$ et node $n_4$ cross one NUMALink6 memory controller)};
%\item node $n_0$ is two hops away from node $n_{20}$ \emph{(e.g. communications between node $n_0$ et node $n_{20}$ cross two NUMALink6 memory controllers)};
%\end{itemize}

Table \ref{tab:idchire} shows the distances advertised by the hwloc library~\cite{Broquedis2010}
that represents the communication time for different distances normalized to the time of a local communication.
Distances named \textit{local} and \textit{peer} form a pair of NUMA nodes (through Intel QPI),
other nodes are either one hop away or two hops away (through NUMALink6).
%node $n_0$ to any other node of the Intel192 machine, whether the node is a peer (node $n_1$ in this case), is located one hop away (nodes $n_2$ to $n_9$, $n_{12}$, $n_{13}$, $n_{16}$ and $n_{17}$) or is located two hops away (nodes $n_{10}$, $n_{11}$, $n_{14}$, $n_{15}$ and $n_{18}$ to $n_{23}$). These factors, which have been correlated with experimental values on both bandwidth and latency by Pilla et al.~\cite{pilla:tel-00981136}, show how much impact the architecture topology may have on a parallel application performance.
\newcolumntype{C}[1]{>{\centering\let\newline\\\arraybackslash\hspace{0pt}}m{#1}}
\begin{table}
\caption{NUMA distances from node 0 advertised by the hwloc library on Intel192.}
\begin{center}
\begin{tabular}{C{3cm} C{2cm} C{2cm} C{2cm} C{2cm}}
    \toprule[0.15em]
		\emph{NUMA nodes location} & \emph{local} & \emph{peer} & \emph{one hop away} & \emph{two hops away} \\
    \midrule[0.1em]
	hwloc distances & 1.0 & 5.0 & 6.5 & 7.9 \\
    \bottomrule[0.15em]
\end{tabular}
\end{center}
\label{tab:idchire}
\end{table}
