\documentclass{Styles/llncs}
%\documentclass[12pt,letterpaper]{article}
\usepackage{times}

\usepackage[utf8]{inputenc}
\usepackage[T1]{fontenc}

\usepackage{url}
\usepackage{graphicx}
\usepackage{caption}
\usepackage{subcaption}
\usepackage{subfig}
\usepackage{wrapfig}
\usepackage{multirow}
\usepackage{boxedminipage}
\usepackage{xspace}
\usepackage{listings}
\usepackage{listingsutf8}
\usepackage{verbatim}
\usepackage{parcolumns}
\usepackage{color}
\usepackage[usenames,dvipsnames,svgnames,table]{xcolor}
%Prevents floating item to "jump" between sections
\usepackage[section]{placeins}
\usepackage{booktabs}
\newcommand{\arevoir}[1]{#1}

\newcommand{\kaapi}{\textsc{\mbox{kaapi}}\xspace}

\newcommand{\libXKOMP}{\textsc{libKOMP}\xspace}

\usepackage{xcolor}
\usepackage{todonotes}
\usepackage[color,leftbars]{changebar}

\newcommand{\cfsect}[1]{(\textit{cf.} section~\ref{#1})}
\newcommand{\cfsectpage}[1]{(\textit{cf.} section~\ref{#1}, page~\pageref{#1})}
\providecommand{\figureref}[1]{\figname~\ref{#1}}
\providecommand{\cftab}[1]{(\textit{cf.} tableau~\ref{#1})}
\newcommand{\cmd}[1]{{\upshape\texttt{\symbol{"5C}#1}}}

\newenvironment{remarque}
{\description \item[Remarque:] \ \slshape}
{\enddescription}

\makeatletter
\newbox\sf@box
\newenvironment{SubFloat}[2][]%
  {\def\sf@one{#1}%
   \def\sf@two{#2}%
   \setbox\sf@box\hbox
     \bgroup}%
  { \egroup
   \ifx\@empty\sf@two\@empty\relax
     \def\sf@two{\@empty}
   \fi
   \ifx\@empty\sf@one\@empty\relax
     \subfloat[\sf@two]{\box\sf@box}%
   \else
     \subfloat[\sf@one][\sf@two]{\box\sf@box}%
   \fi}
\makeatother
\renewcommand\floatpagefraction{.9}
\renewcommand\topfraction{.9}
\renewcommand\bottomfraction{.9}
\renewcommand\textfraction{.1}
\setcounter{totalnumber}{50}
\setcounter{topnumber}{50}
\setcounter{bottomnumber}{50}
\renewcommand{\ttdefault}{pcr}
\lstset{
	tabsize=4,
%	frame=single,
	breaklines=true,
	basicstyle=\ttfamily,
	frame=tb,
	framerule=0.2pt,
%	frameround={tttt},
	showstringspaces=false,
	language=c,
%	linewidth=0.95\textwidth,
	keywordstyle=\color{black}\bfseries,
%	keywordstyle=\color{blue},
	commentstyle=\color{OliveGreen},
	stringstyle=\color{red}\itshape,
	inputencoding=utf8/latin1,
	numbers=left,
	numberstyle=\tiny,
	numbersep=5pt,
% OMP define
emph={\#,pragma, taskwait, omp, task, depend}, emphstyle=\color{RoyalBlue}\bfseries,
emph={[2]in,inout,out,cw}, emphstyle={[2]\color{BrickRed}\bfseries},
emph={[3]tied,untied,shared}, emphstyle={[3]\color{Gray}\bfseries},
emph={[4]lu0,fwd,bdiv,bmod}, emphstyle={[4]\color{DarkGreen}\bfseries},
emph={[5]cw, affinity}, emphstyle={[5]\color{DarkViolet}\bfseries},
    %moredelim=**[is][\only<3>{\color{red}}]{@}{@},
}
\lstdefinestyle{smaller}{basicstyle=\scriptsize\ttfamily}
\lstMakeShortInline|

\newcommand{\benchs}{KASTORS }

\sloppy

\begin{document}

\title{TODO}
\author{
  Philippe Virouleau \and François Broquedis \and Thierry Gautier \and Fabrice Rastello
 \vspace*{-1ex}}
\institute{
   Inria,
   Univ. Grenoble Alpes,  CNRS, Grenoble Institute of Technology, LIG, Grenoble, France
   LIP, ENS de Lyon, France\\
   \email{firstname.lastname@inria.fr}\\
   \email{thierry.gautier@inrialpes.fr}\\
}
\date{}
\maketitle

\begin{abstract}
  \vspace*{-5ex} TODO

\smallskip
  \noindent\textbf{Keywords:}
  \emph{
    OpenMP, task dependencies, affinity, runtime systems, NUMA
  }
\end{abstract}


\section{Introduction}

TODO inclure l'absence de possibilité pour placer les données

\section{Motivation: examples where we need support}

TODO

- trouver une application qui nécessite cela, typiquement une application itérative type stencil
  -> done, jacobi, on peut comparer taskdep+affinity vs taskdep vs for, affini est meilleurs
  -> dpot, si le runtime peut être malin, c'est tout aussi simple que le programmeur indique explicitement quelle dépendance est "importante"
- une clause qui permet de contrôler l’affinité tâche / donnée ou tâche / ressources est importante pour les perf
- le comité discute de cela

\section{Contributions}

2/ contribution du papier
- implémentation d’une clause affinity dans clang et modification du runtime associé
- scheduler affinity aware (europar)
- définition de stratégie de scheduling de tâches dans région parallèle pour permettre de gérer l’affinité en utilisant la politique first touch de l’OS

\section{Implem and results}

Ici on peut refaire tourner jacobi, dpotrf, dgeqrf, avec toute la couche "affinity" de clang.

TODO~\cite{kaapi}
\section{Related work}

TODO


\section*{Acknowledgments}


This work is integrated and supported by the ELCI  project, a French FSN ("Fond pour la Société Numérique")
project that associates academic and industrial partners to design and provide software environment for very high performance
computing.
  \small \bibliographystyle{Styles/iplain}
%\nocite{*}
\bibliography{Bib/paper}

\end{document}
