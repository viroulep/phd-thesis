\documentclass{Styles/llncs}
%\documentclass[12pt,letterpaper]{article}
\usepackage{times}

\usepackage[utf8]{inputenc}
\usepackage[T1]{fontenc}

\usepackage{url}
\usepackage{amsmath}

\usepackage{graphicx}
\usepackage{caption}
\usepackage{subcaption}
\usepackage{subfig}
\usepackage{wrapfig}
\usepackage{multirow}
\usepackage{boxedminipage}
\usepackage{xspace}
\usepackage{listings}
\usepackage{listingsutf8}
\usepackage{verbatim}
\usepackage{parcolumns}
\usepackage{color}
\usepackage[usenames,dvipsnames,svgnames,table]{xcolor}
%Prevents floating item to "jump" between sections
\usepackage[section]{placeins}
\usepackage{booktabs}
\newcommand{\arevoir}[1]{#1}

\newcommand{\kaapi}{\textsc{\mbox{XKaapi}}\xspace}

\newcommand{\libXKOMP}{\textsc{libKOMP}\xspace}
\newcommand{\TG}[1]{{\color{red}\bfseries TG: #1}}

\usepackage{xcolor}
\usepackage{todonotes}
\usepackage[color,leftbars]{changebar}

\newcommand{\cfsect}[1]{(\textit{cf.} section~\ref{#1})}
\newcommand{\cfsectpage}[1]{(\textit{cf.} section~\ref{#1}, page~\pageref{#1})}
\providecommand{\figureref}[1]{\figname~\ref{#1}}
\providecommand{\cftab}[1]{(\textit{cf.} tableau~\ref{#1})}
\newcommand{\cmd}[1]{{\upshape\texttt{\symbol{"5C}#1}}}

\newenvironment{remarque}
{\description \item[Remarque:] \ \slshape}
{\enddescription}

\makeatletter
\newbox\sf@box
\newenvironment{SubFloat}[2][]%
  {\def\sf@one{#1}%
   \def\sf@two{#2}%
   \setbox\sf@box\hbox
     \bgroup}%
  { \egroup
   \ifx\@empty\sf@two\@empty\relax
     \def\sf@two{\@empty}
   \fi
   \ifx\@empty\sf@one\@empty\relax
     \subfloat[\sf@two]{\box\sf@box}%
   \else
     \subfloat[\sf@one][\sf@two]{\box\sf@box}%
   \fi}
\makeatother
\renewcommand\floatpagefraction{.9}
\renewcommand\topfraction{.9}
\renewcommand\bottomfraction{.9}
\renewcommand\textfraction{.1}
\setcounter{totalnumber}{50}
\setcounter{topnumber}{50}
\setcounter{bottomnumber}{50}
\renewcommand{\ttdefault}{pcr}
\lstset{
	tabsize=4,
%	frame=single,
	breaklines=true,
	basicstyle=\ttfamily,
	frame=tb,
	framerule=0.2pt,
%	frameround={tttt},
	showstringspaces=false,
	language=c,
%	linewidth=0.95\textwidth,
	keywordstyle=\color{black}\bfseries,
%	keywordstyle=\color{blue},
	commentstyle=\color{OliveGreen},
	stringstyle=\color{red}\itshape,
	inputencoding=utf8/latin1,
	numbers=left,
	numberstyle=\tiny,
	numbersep=5pt,
% OMP define
emph={\#,pragma, taskwait, omp, task, depend}, emphstyle=\color{RoyalBlue}\bfseries,
emph={[2]in,inout,out,cw,data,numa,core}, emphstyle={[2]\color{BrickRed}\bfseries},
emph={[3]tied,untied,shared, firstprivate, private}, emphstyle={[3]\color{Gray}\bfseries},
emph={[4]lu0,fwd,bdiv,bmod}, emphstyle={[4]\color{DarkGreen}\bfseries},
emph={[5]affinity, omp_get_numa_num, omp_get_core_num, omp_get_num_cores, omp_get_num_numas, omp_get_numa_from_data,omp_set_task_affinity}, emphstyle={[5]\color{DarkViolet}\bfseries},
    %moredelim=**[is][\only<3>{\color{red}}]{@}{@},
}
\lstdefinestyle{smaller}{basicstyle=\scriptsize\ttfamily}
\lstMakeShortInline|


%% For IFP algorithm
\usepackage[algoruled]{algorithm2e}
\usepackage{algorithmic}
\SetKwRepeat{Do}{do}{while}%

\usepackage{tikz}
\usetikzlibrary{arrows,patterns,plotmarks,shapes,snakes,er,3d,automata,backgrounds,topaths,trees,petri,mindmap}
\usetikzlibrary{patterns}
\usepgflibrary{patterns}
\usepackage{pgf}
\usepackage{pgfplots}



\newcommand{\benchs}{KASTORS }

\sloppy

\begin{document}

\title{TODO}
\author{
  Philippe Virouleau \and Adrien Roussel$^{*}$ \and François Broquedis \and Thierry Gautier \and Fabrice Rastello
 \vspace*{-1ex}}
\institute{
   Inria,
   Univ. Grenoble Alpes,  CNRS, Grenoble Institute of Technology, LIG, Grenoble, France
   LIP, ENS de Lyon, France\\
   $^{*}$ IFPEN, Rueil Malmaison\\
   \email{firstname.lastname@inria.fr}\\
   \email{thierry.gautier@inrialpes.fr}\\
}
\date{}
\maketitle

\begin{abstract}
  \vspace*{-5ex} TODO
\smallskip
  \noindent\textbf{Keywords:}
  \emph{
    OpenMP, task dependencies, affinity, runtime systems, NUMA
  }
\end{abstract}


\section{Introduction}

OpenMP has become a major standard to program parallel applications on a wide variety of parallel platforms ranging from desktop notebooks to high-end supercomputers. It provides keywords to express fine grain task-based parallelism that boost the applications performance and scalability on large scale shared memory machines. In particular, tasking in OpenMP helps the programmers parallelize applications with irregular workload, letting the runtime system in charge of performing load balancing through task scheduling in a dynamic way. However, very little support exists to express and to control the affinity between tasks and data on systems with a decentralized memory layout, like \emph{Non-Uniform Memory Architectures} (NUMA). On such systems, the memory is physically split into several banks, also called \emph{NUMA nodes}, which leads to different memory latencies and throughputs depending on the location of the memory bank a core is accessing data from. To get the most performance out of such architectures, OpenMP runtime systems thus need to be extended to make the task scheduler aware of both the underlying hardware and the relation that exists between a task and the data it accesses.

 We relate in this paper our experiences to reach high performance out of OpenMP numerical applications on 192-core NUMA machine. The recently-added \emph{places} concept in the OpenMP 4.0 specification provides ways of binding OpenMP parallel regions to user-defined partitions of the machine. This basically ends up binding the threads of the corresponding region to a set of cores. Thus, relying on the first-touch memory allocation policy as a portable solution to control memory binding, OpenMP places can help to control thread affinity with respect to the memory.
%Nevertheless, if this solution exists it remains cumbersome with inherent shortages: it does not ensure clarity in the program between computations and memory access; Design of library remains complex due to non-composable non functional properties; It does not provide robust solution when load is high unbalanced between threads of different parallel regions.
However, the concept behind OpenMP places needs to be extended to improve the performance of task-based applications, as tasks are most of the time scheduled over threads in a dynamic way according to a work-stealing execution model.  This is why the OpenMP \emph{Architecture Review Board} is currently discussing the introduction of a new \textit{affinity} feature to make the runtime system aware of the affinities between the tasks and the data they access.

In this paper, we present how we control task and data placement inside our OpenMP runtime system, implementing an \emph{affinity} clause which syntax is very close to the one currently discussed by the ARB. We also explain how we manage such information at runtime in order to improve the execution of task-based OpenMP programs on NUMA systems, with a particular focus on the scheduling data structure and the scheduling algorithm.

The contribution of this paper is threefold:
\begin{itemize}
\item We propose an OpenMP-friendly \emph{affinity} extension to the Clang-3.8 compiler able to express affinities between tasks and memory and pass this information along to the runtime system ;
\item We describe an extension to our task-based OpenMP runtime system to guide the scheduling of tasks according to such information to reach better performance on NUMA systems ;
\item We present some preliminary experimental results on running OpenMP benchmarks with tasks dependencies on a 192-core NUMA system, with and without using \emph{affinity}.
\end{itemize}

%Francois : je sais pas s'il faut le mettre dans l'intro ça.
%Some preliminary experimental results demonstrate the capacity to manage finer information such as handle by affinity of task and memory. By analyzing our experimental results, we note OpenMP depend tasks based programs, with assumption that all accessed memory regions are encoded in dependencies, can be automatically scheduled with the same level of performance without adding clause to specify affinity.

The remainder of this paper is organized as follows : blablabla.

\begin{figure}[tc]
%\begin{center}
\begin{minipage}[c]{0.4\linewidth}
\begin{lstlisting}[frame=none,style=smaller,showlines=true,mathescape=true]{bicgstab}
 $\textbf{Compute}$ $r_0 = b - Ax_{0}$
 $r^{*}_{0}$ arbitrary
 $p_0 = r_0$
 $j=0$
 do {
	$\alpha_j = \dfrac{(r_j, r^*_0)}{(Ap_j,r^*_0)}$
	$s_j = r_j - \alpha_j A p_j$
	$ \omega_j = (As_j, s_j) / (As_j, As_j)$
	$x_{j+1} = x_j + \alpha_j p_j + \omega_j s_j$
	$r_{j+1} = s_j - \omega_j A s_j$
	$\beta_j = \dfrac{(r_{j+1},r^*_0)}{(r_j,r^*_0)} \times \dfrac{\alpha_j}{\omega_j}$
	$p_{j+1} = r_{j+1} + \beta_j ( p_j - \omega_j A p_j )$
	$j = j + 1$
 } while ( !convergence)
\end{lstlisting}
\centerline{BiCGStab algorithm}
\end{minipage}\hfill
\begin{minipage}[c]{0.5\linewidth}
\begin{lstlisting}[frame=none,style=smaller,showlines=true,mathescape=true,firstnumber=1]{bicgstab}
$\textbf{Compute}$ $r_0 = b - Ax_{0}$

#pragma omp parallel
#pragma omp single
do {
	$t_j = Ap_j$
	$s_j = r_j - \alpha_j A p_j$
	$ \omega_j = (As_j, s_j) / (As_j, As_j)$
	$x_{j+1} = x_j + \alpha_j p_j + \omega_j s_j$
	$r_{j+1} = s_j - \omega_j A s_j$
	$\beta_j = \dfrac{(r_{j+1},r^*_0)}{(r_j,r^*_0)} \times \dfrac{\alpha_j}{\omega_j}$
	$p_{j+1} = r_{j+1} + \beta_j ( p_j - \omega_j A p_j )$
	$j = j + 1$
while ( !convergence)
\end{lstlisting}
\centerline{Parallelization}
\end{minipage}
%\end{center}
\caption{BiCGStab Algorithm. Left: Mathematical formulation of the algorithm. Right: Sketch of the parallel program using a high level API interface.} \label{lst:bicgstab}
\end{figure}
\section{Motivation: examples where we need support}

The high memory throughput  of NUMA architectures~\cite{} have been introduced at the price of non uniformity in memory latency. Local memory access has lower latency than accessing data on remote memory bank. To get the most performance, computational works should ideally only access to local memory. 

A lot of projects in HPC projects are dealing with sparse linear solver as fundamental building block.
Let us consider the classical BiCGStab~\cite{} algorithm, a classical method for solving sparse linear algebra system. Listing of figure~\ref{lst:bicgstab} sketches the algorithm (left part). $A$ is the sparse matrice, $p_j$, $s_j$, $r_j$ are vectors. The algorithm is structured around a the main loop  that iterates until convergence. At each iteration, the algorithm performs global access to data in (sparse) matrix vector products ($A p_j$ at line 6 and $A s_j$ at line 8) as well in several dot products.

Parallelization of BiCGStab has been presented in several related works~\cite{,}. The algorithm of figure~\ref{lst:bicgstab} is parallelize using one parallel region surrounding the main loop.
\\Open
- trouver une application qui nécessite cela, typiquement une application itérative type stencil
  -> done, jacobi, on peut comparer taskdep+affinity vs taskdep vs for, affini est meilleurs
  -> dpot, si le runtime peut être malin, c'est tout aussi simple que le programmeur indique explicitement quelle dépendance est "importante"
- une clause qui permet de contrôler l’affinité tâche / donnée ou tâche / ressources est importante pour les perf
- le comité discute de cela

\section{Affinity in OpenMP directives}

In this section, we describe the details of our propositions, starting with the
language extension description, followed by our runtime extensions we implemented
to take advantage of this information.

\subsection{Machine model}
\TG{Important ici}

Dire ici comment et à quoi correspond la numérotation des nœuds numa et des ids.
Je pense qu'il faut dire :
- OMP\_PLACES= => list de places sur les cores de la machine\\
- \texttt{#pragma omp parallel} utilise OMP\_PLACES / OMP\_PROC\_BIND pour attribuer un cpuset à chaque thread de la région

\TG{Donc a priori le core id == thread id au sens \texttt{omp\_get\_thread\_num()} dans une région parallèle => replacer l'utilisation de core par thread !}

Pour les bancs NUMA, il faut procéder de la même manière : 
OMP\_NUMA\_PLACES => ensemble de bancs NUMA à utiliser. A voir si ensemble d'ensemble ou non.
Dans le programme numaid i => ième banc numa dans OMP\_NUMA\_PLACES.
Si OMP\_NUMA\_PLACES pas définit => par défaut == ensemble des bancs NUMA.

\subsection{Extension of the OpenMP Task directive}

We propose an extension to precisely control the \emph{affinity} of a task with a specific part of the architecture hierarchy.

The two main components of NUMA architectures are cores and nodes, and one of
the key to get performances out of NUMA architectures is to ensure tasks are
executing close to their data.
Therefore we identified three different kinds of \emph{affinity} the programmer
may need to express, which are the following:
\begin{description}
    \item [A physical core.]
      The runtime should try to schedule the task on the specified physical core.

    \item [A NUMA node.]
      The runtime should try to schedule the task on any of the physical cores on
      the given NUMA node.

    \item [A data.]
      Once the task becomes ready for execution, the runtime should try to schedule it on any of the physical cores on
      the NUMA node on which the given data has been physically allocated.
\end{description}

Additionally, the programmer can specify if this affinity is \emph{strict} (the task must be executed on the given resource), or not.

Since this extension is aimed for the tasking construct, we implemented it as a new
clause for the \emph{task} directive. The proposed syntax for the clause is the following:
\begin{lstlisting}
affinity([numa,core,data]: expr[, strict])
\end{lstlisting}

\TG{Commentaire sur la expr si out of ressources. Commentaire dans le cas de la clause data. } 

\subsection{Extension of the OpenMP runtime API functions}
In order to dynamically get information about the current team hierarchy, we also propose
the following runtime API functions:
\TG{Ici je pense qu'il faudra garder supprimer \texttt{omp\_get\_num\_cores} et l'autre pour ne garder
que les classiques sur les threads. Pour cela renommer core => thread dans la clause affinity.
}
\begin{lstlisting}
//Get the number of NUMA nodes in the team
omp_get_num_numas(void);
//Get the NUMA node the task is currently executed on
omp_get_numa_num(void);
//Get the NUMA node the data has been allocated on
omp_get_numa_from_data(void *data);
//FIXME: les deux appels suivant sont en fait get_num_threads etc
omp_get_num_cores(void);
omp_get_core_num(void);
\end{lstlisting}

These functions allow to query informations about the hardware topology. On machine without numa support we consider that there is one numa node for all the cores.

We also add a runtime API function that mimic the \textit{affinity} clause. The function:
\begin{lstlisting}
//Set the affinity information to the next created tasks
omp_set_task_affinity( 
     omp_affinitykind_t, uintptr_t data, int strict );
\end{lstlisting}
The scope of the function call is the next created task in the current \textit{task region}.
This function takes an \texttt{omp\_affinitykind\_t} value (either \texttt{omp\_affinity\_core}, \texttt{omp\_affinity\_numa} or \texttt{omp\_affinity\_data}) to specify which kind of affinity control is applied. \texttt{value} is either an integer that represents either an identifier to the numa node or the identifier to a core; either it is the address in the process address space used to select the affinity numa node \textbf{when} the task becomes ready for execution.


We implemented these propositions in the Clang compiler, based on the 3.8 version\footnote{https://github.com/viroulep/clang}; and we also added the corresponding entry points in Clang's OpenMP runtime\footnote{https://github.com/viroulep/openmp}.


\subsection{Runtime extension}

We implemented extensions in the OpenMP runtime developed in our team, \libXKOMP~\cite{Durand2013,libkomp},
which is based on the \kaapi~\cite{Bleuse2014,parco2015} runtime system.
\kaapi is a task-based runtime system, using workstealing has a general scheduling strategy.
Here is a brief description of some of its key internal structures and machanisms.

\subsubsection{The way \kaapi models the architecture.}
\TG{Je pense ici qu'il ne faut pas parler de place != des places OpenMP. Avant (papier iwomp 2013) nous parlions de locality domain et je pense qu'il faut reprendre ce terme pour éviter toute confusion. }.
\kaapi sees the architecture topology as a hierarchy of \verb/places/.
A \verb/place/ is a list of tasks associated with a subset of the machine processing units.
\kaapi's places are very similar to the notion of \emph{shepherd} introduced in \cite{DBLP:journals/ijhpca/OlivierPWSP12}, or ForestGOMP's \emph{runqueues}~\cite{BroFurGogWacNam10IJPP}.
\kaapi most of the time only considers two levels of places : node-level places,
which are bound to the set of processors contained in a NUMA node, and processor-level places, which are bound to a single processor of the platform.
This way, at the processor level one \verb/place/ is associated to each of the physical cores, and
at the NUMA node level one \verb/place/ is associated to each of the NUMA nodes.


\subsubsection{The way \kaapi enables ready tasks and steals them.}

The scheduling framework in \kaapi~\cite{Bleuse2014,parco2015} relies on virtual functions
for \textit{selecting a victim} and \textit{selecting a place} to push a ready task.
When a processor becomes idle, the runtime system calls a function, called  \verb/WSselect/ for \emph{work-stealing select}, to browse the topology to find a place from which stealing a task from the place task queue.
%There are many ways to do so, implemented in different \emph{selection strategies} called \verb/WSselect/, for \emph{work-stealing select} in the remaining of the paper.
%Once a place is selected, the processor will take a ready task from its queue.

\subsubsection{Implementation of the affinity}
We extend the internal control variables in order to add affinity property. Runtime API functions can get and set this internal control variable \textit{affinity-var}. There is one copy per data environment. The variable is composed of two fields: a \texttt{omp\_affinitykind\_t} value and an integer large enough to encode a pointer.

The scope of this \textit{affinity} ICV value is until the next task generating construct in the OpenMP task region of the encountering thread~\cite{openmp40}\TG{Qui vérifie le vocabulaire "task region of the encountering thread" : tâche courante en cours d'exécution}. 
On a task generating construct, if it has an affinity clause, the runtime set the appropriate kind of affinity and the integer value in the ICVs.
During task creation, these parameters will be set in the internal task descriptor.

When a task becomes ready to be executed, the function responsible for the \textit{selection of the place} to push
the task will look at the affinity and select the appropriate place. The capacity to deferred the evaluation of the affinity until the task becomes ready allows the runtime to take better decision.
\kaapi relies on the \verb/get_mempolicy/ function to identify the NUMA node on which a data is allocated.

As described earlier, an affinity can be \emph{strict} or not; to implement this we used
a private queue per place. If the affinity is strict, the task is pushed to the place's private queue.
During the \textit{victim selection}, a thread may only steal from the place's
public queue (in case of a place attached to a  NUMA node, every thread on this node can steal from the private queue).

\section{Usage example and experimentation results}

\subsection{Implementation in benchmarks}

We looked into our jacobi application from the KASTORS benchmark suite~\cite{virouleau:hal-01081974}.
The application is a 2D stencil computational kernel that is repeatedly applied until
convergence is detected. We used a blocked version of this algorithm.
This blocked version uses tasks for data initialization, and either \emph{for} constructs or
\emph{dependent tasks} constructs.
Each operation on a point of the matrix depends on its neighbouring blocks,
therefore the blocks should be physically evenly distributed among the nodes,
and the computational tasks should be located close to these data.

Knowing the number of cores in the team, the matrix size, and the block size, we computed a mapping
between multiple neighbouring blocks and the different cores.

We used the affinity clause to achieve two goals:
\begin{itemize}
    \item Ensure the physical distribution of the data during initialization
    \item Ensure tasks stay close to their dependencies during computation, by putting it on its partition's core.
\end{itemize}

We implemented both a strict affinity and a non-strict affinity version.


\begin{figure}[htbp]
\begin{lstlisting}[caption=Usage example of the affinity clause for initialization,frame=tlrb,style=smaller,label=lst:init-aff]{lst:init-aff}
for (j = 0; j < ny; j+= block_size)
  for (i = 0; i < nx; i+= block_size) {
    #pragma omp task firstprivate(i,j) private(ii,jj)\
       affinity(core:GET_PARTITION(i, j, block_size, nx, ny, square_len), 1)
    {
      for (jj=j; jj<j+block_size; ++jj)
        for (ii=i; ii<i+block_size; ++ii) {
          if (ii == 0 || ii == nx - 1 || jj == 0 || jj == ny - 1)
            (*unew)[ii][jj] = (*f)[ii][jj];
          else
            (*unew)[ii][jj] = 0.0;
        }
    }
  }
\end{lstlisting}
\end{figure}

TODO virer un des deux

\begin{figure}[htbp]
\begin{lstlisting}[caption=Usage example of the affinity clause for computation,frame=tlrb,style=smaller,label=lst:compute-aff]{lst:compute-aff}
// Compute a new estimate.
for (int j = 0; j < ny; j += block_size) {
  for (int i = 0; i < nx; i += block_size) {
    int xdm1 = i == 0 ? 0 : block_size;
    int xdp1 = i == nx-block_size ? 0 : block_size;
    int ydp1 = j == ny-block_size ? 0 : block_size;
    int ydm1 = j == 0 ? 0 : block_size;
    #pragma omp task shared(u_, unew_) \
                     depend(out: unew[i: block_size][j: block_size]) \
                     depend(in: f[i: block_size][j: block_size], \
                                u[i: block_size][j: block_size], \
                                u[(i - xdm1): block_size][j: block_size], \
                                u[i: block_size][(j + ydp1): block_size], \
                                u[i: block_size][(j - ydm1): block_size], \
                                u[(i + xdp1): block_size][j: block_size]) \
                     affinity(core:GET_PARTITION(i, j, block_size, nx, ny,
                                                 square_len), 1)
    compute_estimate(i/block_size, j/block_size, u_, unew_, f_, dx, dy,
                     nx, ny, block_size);
  }
}
\end{lstlisting}
\end{figure}

\subsection{Experiments results}

We compared four version of the application with both Clang's OpenMP runtime
and \kaapi runtime:

\begin{itemize}
  \item Blocked with \emph{for} construct used for computation.
  \item Blocked with dependent \emph{tasks}
  \item Blocked with \emph{for}, using affinity during initialization
  \item Blocked with dependent \emph{tasks}, using affinity
\end{itemize}

TODO Décrire les différentes variantes
(affinity, initialisation etc)

décrire choix des tailles+tailles de bloc

graph

conclusion: type d'application (stencil) très dépendant de la localisation de ses voisins
-> le côté strict permet de garantir que les tâches ne bougent pas
-> ne pas mettre d'affinité/d'initialisation implique que les tâches auront de gros temps d'accès à leurs données
-> régler le problème du block-for avec initialisation, si itérations dans le "bon" ordre ça se passe bien.

\subsection{Parallel linear algebra algorithms}

\subsubsection{Sparse Matrix Vector product (SpMV operation)}

In this section, we present two major algorithms of linear algebra. We split data following matrix graph partitioning techniques \cite{Saad:2003:IMS:829576} while using automatic graph partitioner like Metis~\cite{metis} tools. In such a decomposition, a matrix A is split in several sub-domains of several rows. A sub-graph is assigned to each partition, where internal nodes refers to block diagonal entries and off diagonal elements are stored in external matrix data structure. To ensure an efficient data distribution on numa nodes, all the local data structures to a partition are allocated in parallel. Vectors are split following row permutations and splitting dictated by graph partitioning, and local parts of the vectors are distributed too. Sparse matrix are stored in CSR format.

Let us consider now the operation $A \times X = Y$, where A is a sparse matrix, x is the input vector and b the output result. Considering the above data partitioning, the multiplication operation can be written as in figure~\ref{lst:spmv}.

\begin{figure}[tc]
\begin{center}
\begin{minipage}[c]{0.9\linewidth}
\small{
}
\begin{lstlisting}[frame=none,style=smaller,showlines=true,mathescape=true,firstnumber=1]{spmv}
$\textbf{Compute} Y =  A\times X$
$A$ sparse matrix, see text for discussion
$X$ vectors, $X[i]$ sub vector associated to partition $i$
$Y$ same as $X$
/* omp parallel region outside the function */
for (i=0; i < $n_{partitions}$; ++i)
{
	#pragma omp task depend(in:X[i]) depend(out:Y[i]) \
		affinity(thread:i,0)
	/* store result in Y[i] */
	csr_mult( A.interior[i], X[i], Y[i]); 

	for (j=0; j < $A.neighbor\_of[i]$; ++j)
		#pragma omp task depend(in: X[j]) reduction(+: Y[i]) \
		     affinity( thread: i, 0)
		/* add result in Y[i]  */
		csr_mult_addin(A.interior[i][j], X[j], X[i]);
}
\end{lstlisting}
\end{minipage}\hfill
\end{center}
\caption{OpenMP SpMV Algorithm} \label{lst:spmv}
\end{figure}

Affinity are constrained by assigning the tasks to a thread following an owner compute rules: a task is mapped on the thread holding the output subvector $Y[i]$ (lines 8 and 14). 
%To avoid bus contention on the external copy for each sub-domain from their neighbours, we give a local copy of the vector $x$ to each numa node before SpMV computations for this test. TG: du fait des copies sans entrelacement par du calcul, il doit y avoir plus de contention
Each partition computations are encapsulated in an independent task, so the number of inserted tasks per SpMV operation is the number of partitions. 
In our experiment, $500$ iterates of SpMV operations are timed and the average times is reported in the figure~\ref{figs:spmv:3000} with an increasing number of threads $p$, where $p= {2,4,8,10,12,16,32,48,64,92,128,160,192}$. 
re the executions differs from the use of runtime systems. "XKaapi" stand for the use of XKaapi runtime system, and "Intel" the use of Intel runtime. For this experiment, we take matrices coming from the Finite Volume discretization of a 2D Laplace problem on several mesh sizes.  Reported graph is for mesh of size $2000 \times 2000$.

As we can see, for both executions the time spent to compute all the operations decreases. However, XKaapi runtime systems offers more powerful results because of the use of affinity clauses to place tasks on specific numa node where the output data is allocated to avoid massive data retrieving.
\TG{A reprendre: expliquer que intel runtime > increase because 2 numa nodes if p>8}
\TG{Pourquoi cela tend à diminuer ensuite ?}

\begin{figure}
  \centering
    \begin{tikzpicture}[scale=0.9]
      \begin{axis}[%ybar,
                   symbolic x coords = {2,4,8,10,12,14,16,32,48,64,92,128,160,192},
                   legend style = {area legend,legend pos = north east,},
                   xtick = data,
                   xticklabel style={/pgf/number format/1000 sep=},
                   width=1\textwidth,
                   xlabel=\textsc{cores},
                   ylabel=\textsc{time (seconds)}] 
        \addplot table[x=th,y=2000]{data/RuntimeSystems/iWomp/Spmv_omp.dat.csv};
        \addplot table[x=th,y=2000]{data/RuntimeSystems/iWomp/Spmv_kaapi.dat.csv};
        \legend{Intel,XKaapi};
      \end{axis}
    \end{tikzpicture}
    \caption{Mesh size: 2000x2000}
    \label{figs:spmv:2000}
\end{figure}



\section{Related work}

\section{Related Work}
\label{sec:related_work}

\TG{Related work de iwomp 2013. A reprendre ne garder que le gestion des queues etc.}\\

Many research projects have been carried out to improve execution of
OpenMP applications on NUMA machines.

The HPCTools group at the University of Houston has been working in
this area for a long time, proposing compile-time techniques that can
help improving memory affinity on hierarchical architectures like
distributed shared memory
platforms~\cite{Marowka:2004:OAD:1064428.1064431}. Huang et
al.~\cite{Huang-Chapman-locality-OpenMP} proposed OpenMP extensions to
deal with memory affinity on NUMA machines, like ways of explicitly
aligning tasks and data inside logical partitions of the architecture
called \textit{locations}. While the proposed extension is interesting
to deal with regular memory-bound applications, it does not tackle
the problems induced by irregular workloads.

Olivier et
al.~\cite{Olivier-pre-queue-Bronis,Olivier:2012:CMW:2388996.2389085}
introduced node-level queues of OpenMP tasks, called \textit{locality
  domains}, to ensure tasks and data locality on NUMA systems. The
runtime system does not maintain affinity information between tasks
and data during execution. Data placement is implicitly obtained
considering that the tasks access memory pages that were allocated
using the \textit{first-touch} allocation policy. The authors thus
ensure locality by always scheduling a task on the same locality
domain, preventing application programmers to experiment with other
memory bindings.

The INRIA Runtime group at the University of Bordeaux proposed the
ForestGOMP runtime system~\cite{BroAumGogThiWacNam10IPDPS} that comes
with an API to express affinities between OpenMP parallel regions and
dynamically allocated data. ForestGOMP implements load balancing of
nested OpenMP parallel regions by moving branches of the corresponding
tree of user-level threads on a hierarchical way. Memory affinity
information is gathered at runtime and can be taken into account when
performing load balancing. Our work extends this approach to deal with
parallel loops while ensuring load balancing in a different way.

Mahéo et al.~\cite{Maheo-MPC-OpenMP} used similar techniques to
speed up hybrid MPI/OpenMP synchronizations on hierarchical
architectures, including NUMA machines. Both our work and theirs build
upon the same concepts and can be stated as complementary.

Subramanian and Eager~\cite{Subramaniam:1994} introduce an affinity
loop scheduler for unbalanced workload. They study iterative
applications involving series of parallel loops in which \textit{"the
  execution times of any particular iteration do not vary widely from
  one execution of the loop to the
  next"}\cite{Subramaniam:1994}. Their proposition is based upon a
two-phase algorithm: first, the iterations are equally divided between
the processors; then the scheduler dynamically adapts the workload by
making idle processors steal a constant fraction ($1/P$) of the
remaining iterations from occupied ones. In~\cite{Yong97}, Yong
\textit{et al.}  extends the work of Subramanian and Eager by
providing new ways of adapting the workload considering an history of
previous executions of a particular parallel loop.

Taking advantage of the temporal coherency of the simulation is a very
interesting idea but cannot be used in all situations. For instance,
it would not be effective on applications involving several parallel
loops with varying workloads, like the PMA application we used to
evaluate our \textit{adaptive} loop scheduler. The first phase of our
scheduling strategy is similar to the one introduced by Subramanian
\textit{et al.}, as \textit{adaptive} equally pre-distributes the loop
iterations over the processors, which is a compromise between
balancing the workload of the loop and preserving the affinity across
several executions of the same loop. However, the second phase of our
algorithm is different from the one implemented by Subramanian's
adaptive loop scheduler. Indeed, \textit{adaptive} relies on a
work-stealing algorithm~\cite{Frigo:1998:ICM:277652.277725} in which
idle processors steal half of the remaining iterations from a
randomly-selected victim. With such an approach, our \textit{adaptive}
loop scheduler does not require to maintain a global view of the
workloads associated to each processor, unlike
proposition~\cite{Subramaniam:1994}.



\section*{Acknowledgments}


This work is integrated and supported by the ELCI  project, a French FSN ("Fond pour la Société Numérique")
project that associates academic and industrial partners to design and provide software environment for very high performance
computing.
  \small \bibliographystyle{Styles/iplain}
%\nocite{*}
\bibliography{Bib/paper}

\end{document}
