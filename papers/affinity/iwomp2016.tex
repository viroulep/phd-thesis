\documentclass{Styles/llncs}
%\documentclass[12pt,letterpaper]{article}
\usepackage{times}

\usepackage[utf8]{inputenc}
\usepackage[T1]{fontenc}

\usepackage{url}
\usepackage{amsmath}

\usepackage{graphicx}
\usepackage{caption}
\usepackage{subcaption}
\usepackage{subfig}
\usepackage{wrapfig}
\usepackage{multirow}
\usepackage{boxedminipage}
\usepackage{xspace}
\usepackage{listings}
\usepackage{listingsutf8}
\usepackage{verbatim}
\usepackage{parcolumns}
\usepackage{color}
\usepackage[usenames,dvipsnames,svgnames,table]{xcolor}
%Prevents floating item to "jump" between sections
\usepackage[section]{placeins}
\usepackage{booktabs}
\newcommand{\arevoir}[1]{#1}

\newcommand{\kaapi}{\textsc{\mbox{kaapi}}\xspace}

\newcommand{\libXKOMP}{\textsc{libKOMP}\xspace}

\usepackage{xcolor}
\usepackage{todonotes}
\usepackage[color,leftbars]{changebar}

\newcommand{\cfsect}[1]{(\textit{cf.} section~\ref{#1})}
\newcommand{\cfsectpage}[1]{(\textit{cf.} section~\ref{#1}, page~\pageref{#1})}
\providecommand{\figureref}[1]{\figname~\ref{#1}}
\providecommand{\cftab}[1]{(\textit{cf.} tableau~\ref{#1})}
\newcommand{\cmd}[1]{{\upshape\texttt{\symbol{"5C}#1}}}

\newenvironment{remarque}
{\description \item[Remarque:] \ \slshape}
{\enddescription}

\makeatletter
\newbox\sf@box
\newenvironment{SubFloat}[2][]%
  {\def\sf@one{#1}%
   \def\sf@two{#2}%
   \setbox\sf@box\hbox
     \bgroup}%
  { \egroup
   \ifx\@empty\sf@two\@empty\relax
     \def\sf@two{\@empty}
   \fi
   \ifx\@empty\sf@one\@empty\relax
     \subfloat[\sf@two]{\box\sf@box}%
   \else
     \subfloat[\sf@one][\sf@two]{\box\sf@box}%
   \fi}
\makeatother
\renewcommand\floatpagefraction{.9}
\renewcommand\topfraction{.9}
\renewcommand\bottomfraction{.9}
\renewcommand\textfraction{.1}
\setcounter{totalnumber}{50}
\setcounter{topnumber}{50}
\setcounter{bottomnumber}{50}
\renewcommand{\ttdefault}{pcr}
\lstset{
	tabsize=4,
%	frame=single,
	breaklines=true,
	basicstyle=\ttfamily,
	frame=tb,
	framerule=0.2pt,
%	frameround={tttt},
	showstringspaces=false,
	language=c,
%	linewidth=0.95\textwidth,
	keywordstyle=\color{black}\bfseries,
%	keywordstyle=\color{blue},
	commentstyle=\color{OliveGreen},
	stringstyle=\color{red}\itshape,
	inputencoding=utf8/latin1,
	numbers=left,
	numberstyle=\tiny,
	numbersep=5pt,
% OMP define
emph={\#,pragma, taskwait, omp, task, depend}, emphstyle=\color{RoyalBlue}\bfseries,
emph={[2]in,inout,out,cw}, emphstyle={[2]\color{BrickRed}\bfseries},
emph={[3]tied,untied,shared}, emphstyle={[3]\color{Gray}\bfseries},
emph={[4]lu0,fwd,bdiv,bmod}, emphstyle={[4]\color{DarkGreen}\bfseries},
emph={[5]cw, affinity}, emphstyle={[5]\color{DarkViolet}\bfseries},
    %moredelim=**[is][\only<3>{\color{red}}]{@}{@},
}
\lstdefinestyle{smaller}{basicstyle=\scriptsize\ttfamily}
\lstMakeShortInline|


%% For IFP algorithm
\usepackage[algoruled]{algorithm2e}
\usepackage{algorithmic}
\SetKwRepeat{Do}{do}{while}%

\usepackage{tikz}
\usetikzlibrary{arrows,patterns,plotmarks,shapes,snakes,er,3d,automata,backgrounds,topaths,trees,petri,mindmap}
\usetikzlibrary{patterns}
\usepgflibrary{patterns}
\usepackage{pgf}
\usepackage{pgfplots}



\newcommand{\benchs}{KASTORS }

\sloppy

\begin{document}

\title{TODO}
\author{
  Philippe Virouleau \and Adrien Roussel$^{*}$ \and François Broquedis \and Thierry Gautier \and Fabrice Rastello
 \vspace*{-1ex}}
\institute{
   Inria,
   Univ. Grenoble Alpes,  CNRS, Grenoble Institute of Technology, LIG, Grenoble, France
   LIP, ENS de Lyon, France\\
   $^{*}$ IFPEN, Rueil Malmaison\\
   \email{firstname.lastname@inria.fr}\\
   \email{thierry.gautier@inrialpes.fr}\\
}
\date{}
\maketitle

\begin{abstract}
  \vspace*{-5ex} TODO

\smallskip
  \noindent\textbf{Keywords:}
  \emph{
    OpenMP, task dependencies, affinity, runtime systems, NUMA
  }
\end{abstract}


\section{Introduction}

OpenMP has become a major standard to program parallel applications on a wide variety of parallel platforms ranging from desktop notebooks to high-end supercomputers. It provides keywords to express fine grain task-based parallelism that boost the applications performance and scalability on large scale shared memory machines. In particular, tasking in OpenMP helps the programmers parallelize applications with irregular workload, letting the runtime system in charge of performing load balancing through task scheduling in a dynamic way. However, very little support exists to express and to control the affinity between tasks and data on systems with a decentralized memory layout, like \emph{Non-Uniform Memory Architectures} (NUMA). On such systems, the memory is physically split into several banks, also called \emph{NUMA nodes}, which leads to different memory latencies and throughputs depending on the location of the memory bank a core is accessing data from. To get the most performance out of such architectures, OpenMP runtime systems thus need to be extended to make the task scheduler aware of both the underlying hardware and the relation that exists between a task and the data it accesses.
 
 We relate in this paper our experiences to reach high performance out of OpenMP numerical applications on 192-core NUMA machine. The recently-added \emph{places} concept in the OpenMP 4.0 specification provides ways of binding OpenMP parallel regions to user-defined partitions of the machine. This basically ends up binding the threads of the corresponding region to a set of cores. Thus, relying on the first-touch memory allocation policy as a portable solution to control memory binding, OpenMP places can help to control thread affinity with respect to the memory.
%Nevertheless, if this solution exists it remains cumbersome with inherent shortages: it does not ensure clarity in the program between computations and memory access; Design of library remains complex due to non-composable non functional properties; It does not provide robust solution when load is high unbalanced between threads of different parallel regions.
However, the concept behind OpenMP places needs to be extended to improve the performance of task-based applications, as tasks are most of the time scheduled over threads in a dynamic way according to a work-stealing execution model.  This is why the OpenMP \emph{Architecture Review Board} is currently discussing the introduction of a new \textit{affinity} feature to make the runtime system aware of the affinities between the tasks and the data they access. 

In this paper, we present how we control task and data placement inside our OpenMP runtime system, implementing an \emph{affinity} clause which syntax is very close to the one currently discussed by the ARB. We also explain how we manage such information at runtime in order to improve the execution of task-based OpenMP programs on NUMA systems, with a particular focus on the scheduling data structure and the scheduling algorithm. 

The contribution of this paper is threefold:
\begin{itemize}
\item We propose an OpenMP-friendly \emph{affinity} extension to the Clang-3.8 compiler able to express affinities between tasks and memory and pass this information along to the runtime system ;
\item We describe an extension to our task-based OpenMP runtime system to guide the scheduling of tasks according to such information to reach better performance on NUMA systems ;
\item We present some preliminary experimental results on running OpenMP benchmarks with tasks dependencies on a 192-core NUMA system, with and without using \emph{affinity}.
\end{itemize}

%Francois : je sais pas s'il faut le mettre dans l'intro ça.
%Some preliminary experimental results demonstrate the capacity to manage finer information such as handle by affinity of task and memory. By analyzing our experimental results, we note OpenMP depend tasks based programs, with assumption that all accessed memory regions are encoded in dependencies, can be automatically scheduled with the same level of performance without adding clause to specify affinity.

The remainder of this paper is organized as follows : blablabla.

\begin{figure}[tc]
%\begin{center}
\begin{minipage}[c]{0.4\linewidth}
\begin{lstlisting}[frame=none,style=smaller,showlines=true,mathescape=true]{bicgstab}
 $\textbf{Compute}$ $r_0 = b - Ax_{0}$
 $r^{*}_{0}$ arbitrary
 $p_0 = r_0$
 $j=0$
 do {
	$\alpha_j = \dfrac{(r_j, r^*_0)}{(Ap_j,r^*_0)}$
	$s_j = r_j - \alpha_j A p_j$
	$ \omega_j = (As_j, s_j) / (As_j, As_j)$
	$x_{j+1} = x_j + \alpha_j p_j + \omega_j s_j$
	$r_{j+1} = s_j - \omega_j A s_j$
	$\beta_j = \dfrac{(r_{j+1},r^*_0)}{(r_j,r^*_0)} \times \dfrac{\alpha_j}{\omega_j}$
	$p_{j+1} = r_{j+1} + \beta_j ( p_j - \omega_j A p_j )$
	$j = j + 1$
 } while ( !convergence)
\end{lstlisting}
\centerline{BiCGStab algorithm}
\end{minipage}\hfill
\begin{minipage}[c]{0.5\linewidth}
\begin{lstlisting}[frame=none,style=smaller,showlines=true,mathescape=true,firstnumber=1]{bicgstab}
$\textbf{Compute}$ $r_0 = b - Ax_{0}$

#pragma omp parallel
#pragma omp single
do {
	$t_j = Ap_j$
	$s_j = r_j - \alpha_j A p_j$
	$ \omega_j = (As_j, s_j) / (As_j, As_j)$
	$x_{j+1} = x_j + \alpha_j p_j + \omega_j s_j$
	$r_{j+1} = s_j - \omega_j A s_j$
	$\beta_j = \dfrac{(r_{j+1},r^*_0)}{(r_j,r^*_0)} \times \dfrac{\alpha_j}{\omega_j}$
	$p_{j+1} = r_{j+1} + \beta_j ( p_j - \omega_j A p_j )$
	$j = j + 1$
while ( !convergence)
\end{lstlisting}
\centerline{Parallelization}
\end{minipage}
%\end{center}
\caption{BiCGStab Algorithm. Left: Mathematical formulation of the algorithm. Right: Sketch of the parallel program using a high level API interface.} \label{lst:bicgstab}
\end{figure}
\section{Motivation: examples where we need support}

The high memory throughput  of NUMA architectures~\cite{} have been introduced at the price of non uniformity in memory latency. Local memory access has lower latency than accessing data on remote memory bank. To get the most performance, computational works should ideally only access to local memory. 

A lot of projects in HPC projects are dealing with sparse linear solver as fundamental building block.
Let us consider the classical BiCGStab~\cite{} algorithm, a classical method for solving sparse linear algebra system. Listing of figure~\ref{lst:bicgstab} sketches the algorithm (left part). $A$ is the sparse matrice, $p_j$, $s_j$, $r_j$ are vectors. The algorithm is structured around a the main loop  that iterates until convergence. At each iteration, the algorithm performs global access to data in (sparse) matrix vector products ($A p_j$ at line 6 and $A s_j$ at line 8) as well in several dot products.

Parallelization of BiCGStab has been presented in several related works~\cite{,}. The algorithm of figure~\ref{lst:bicgstab} is parallelize using one parallel region surrounding the main loop.
\\Open
- trouver une application qui nécessite cela, typiquement une application itérative type stencil
  -> done, jacobi, on peut comparer taskdep+affinity vs taskdep vs for, affini est meilleurs
  -> dpot, si le runtime peut être malin, c'est tout aussi simple que le programmeur indique explicitement quelle dépendance est "importante"
- une clause qui permet de contrôler l’affinité tâche / donnée ou tâche / ressources est importante pour les perf
- le comité discute de cela

\section{Affinity in OpenMP directives}

2/ contribution du papier
- implémentation d’une clause affinity dans clang et modification du runtime associé
- scheduler affinity aware (europar)
- définition de stratégie de scheduling de tâches dans région parallèle pour permettre de gérer l’affinité en utilisant la politique first touch de l’OS

\section{Implem and results}

Ici on peut refaire tourner jacobi, dpotrf, dgeqrf, avec toute la couche "affinity" de clang.

\subsection{Parallel linear algebra algorithms}

\subsubsection{Sparse Matrix Vector product (SpMV operation)}

In this section, we present two major algorithms of linear algebra. We split data following matrix graph partitioning techniques \cite{Saad:2003:IMS:829576} while using automatic graph partitioner like Metis~\cite{metis} tools. In such a decomposition, a matrix A is split in several sub-domains of several rows. A sub-graph is assigned to each partition, where internal nodes refers to block diagonal entries and off diagonal elements are stored in external matrix data structure. To ensure an efficient data distribution on numa nodes, all the local data structures to a partition are allocated in parallel. Vectors are split following row permutations and splitting dictated by graph partitioning, and local parts of the vectors are distributed too. Sparse matrix are stored in CSR format.

Let us consider now the operation $A * x = b$, where A is a sparse matrix, x is the input vector and b the output result. Considering the above data partitioning, the multiplication operation can be written as following :

\begin{figure}[tc]
\begin{center}
\begin{minipage}[c]{0.6\linewidth}
\small{
\textbf{Compute} $Y =  A\times X$
$A$ matrix in CSR format\\
}
\begin{lstlisting}[frame=none,style=smaller,showlines=true,mathescape=true,firstnumber=1]{bicgstab}
for (i=0; i < $n_{partitions}$; ++i)
{
	A.mult_Internal(X[i], Y[i]); 
	A.mult_External(Y[i], X);
}
\end{lstlisting}
\end{minipage}\hfill
\end{center}
\caption{SpMV Algorithm} \label{lst:spmv}
\end{figure}

The algorithm can easily be parallelized, by assigning the computations of a partition to a core. To avoid bus contention on the external copy for each sub-domain from their neighbours, we give a local copy of the vector $x$ to each numa node before SpMV computations for this test. Each partition computations are encapsulated in an independent task, so the number of inserted tasks per SpMV operation is the number of partitions. In our experiment, we push $500$ iterates of SpMV operations and look at the computation time on $p$ threads, where $p= {2,4,8,10,12,16,32,48,64,92,128,160,192}$. Results are gathered in Figures \ref{figs:spmv:1000}, \ref{figs:spmv:2000}, \ref{figs:spmv:3000}, where the executions differs from the use of runtime systems. "XKaapi" stand for the use of XKaapi runtime system, and "Intel" the use of Intel runtime. For this experiment, we take matrices coming from the Finite Volume discretization of a 2D Laplace problem on several mesh sizes.  

\begin{figure}
  \centering
    \begin{tikzpicture}[scale=0.9]
      %\begin{loglogaxis}[legend style = { legend pos = north east}]
      \begin{axis}[%ybar,
                   symbolic x coords = {2,4,8,10,12,14,16,32,48,64,92,128,160,192},
                   legend style = {area legend,legend pos = north east,},
                   xtick = data,
                   xticklabel style={/pgf/number format/1000 sep=},
                   width=1\textwidth,
                   xlabel=\textsc{cores},
                   ylabel=\textsc{time (seconds)}]  
        \addplot table[x=th,y=1000]{data/RuntimeSystems/iWomp/Spmv_omp.dat.csv};
        \addplot table[x=th,y=1000]{data/RuntimeSystems/iWomp/Spmv_kaapi.dat.csv};
        \legend{Intel,XKaapi};
      \end{axis}
    \end{tikzpicture}
    \caption{Mesh size: 1000x1000}
    \label{figs:spmv:1000}
\end{figure}

\begin{figure}
  \centering
    \begin{tikzpicture}[scale=0.9]
      \begin{axis}[%ybar,
                   symbolic x coords = {2,4,8,10,12,14,16,32,48,64,92,128,160,192},
                   legend style = {area legend,legend pos = north east,},
                   xtick = data,
                   xticklabel style={/pgf/number format/1000 sep=},
                   width=1\textwidth,
                   xlabel=\textsc{cores},
                   ylabel=\textsc{time (seconds)}] 
        \addplot table[x=th,y=2000]{data/RuntimeSystems/iWomp/Spmv_omp.dat.csv};
        \addplot table[x=th,y=2000]{data/RuntimeSystems/iWomp/Spmv_kaapi.dat.csv};
        \legend{Intel,XKaapi};
      \end{axis}
    \end{tikzpicture}
    \caption{Mesh size: 2000x2000}
    \label{figs:spmv:2000}
\end{figure}

\begin{figure}
  \centering
    \begin{tikzpicture}[scale=0.9]
      %\begin{loglogaxis}[legend style = { legend pos = north east}]
      \begin{axis}[%ybar,
                   symbolic x coords = {2,4,8,10,12,14,16,32,48,64,92,128,160,192},
                   legend style = {area legend,legend pos = north east,},
                   xtick = data,
                   xticklabel style={/pgf/number format/1000 sep=},
                   width=1\textwidth,
                   xlabel=\textsc{cores},
                   ylabel=\textsc{time (seconds)}] 
        \addplot table[x=th,y=3000]{data/RuntimeSystems/iWomp/Spmv_omp.dat.csv};
        \addplot table[x=th,y=3000]{data/RuntimeSystems/iWomp/Spmv_kaapi.dat.csv};
        \legend{Intel,XKaapi};
      \end{axis}
    \end{tikzpicture}
    \caption{Mesh size: 3000x3000}
    \label{figs:spmv:3000}
\end{figure}

As we can see, for both executions the time spent to compute all the operations decreases. However, XKaapi runtime systems offers more powerful results because of the use of affinity clauses to place tasks on specific numa node where the output data is allocated to avoid massive data retrieving.



\section{Related work}

TODO


\section*{Acknowledgments}


This work is integrated and supported by the ELCI  project, a French FSN ("Fond pour la Société Numérique")
project that associates academic and industrial partners to design and provide software environment for very high performance
computing.
  \small \bibliographystyle{Styles/iplain}
%\nocite{*}
\bibliography{Bib/paper}

\end{document}
